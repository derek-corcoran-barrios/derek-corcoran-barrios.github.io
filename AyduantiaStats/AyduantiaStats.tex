\documentclass[]{book}
\usepackage{lmodern}
\usepackage{amssymb,amsmath}
\usepackage{ifxetex,ifluatex}
\usepackage{fixltx2e} % provides \textsubscript
\ifnum 0\ifxetex 1\fi\ifluatex 1\fi=0 % if pdftex
  \usepackage[T1]{fontenc}
  \usepackage[utf8]{inputenc}
\else % if luatex or xelatex
  \ifxetex
    \usepackage{mathspec}
  \else
    \usepackage{fontspec}
  \fi
  \defaultfontfeatures{Ligatures=TeX,Scale=MatchLowercase}
\fi
% use upquote if available, for straight quotes in verbatim environments
\IfFileExists{upquote.sty}{\usepackage{upquote}}{}
% use microtype if available
\IfFileExists{microtype.sty}{%
\usepackage{microtype}
\UseMicrotypeSet[protrusion]{basicmath} % disable protrusion for tt fonts
}{}
\usepackage[margin=1in]{geometry}
\usepackage{hyperref}
\hypersetup{unicode=true,
            pdftitle={Practicos de Bioestadística 2},
            pdfauthor={Derek Corcoran},
            pdfborder={0 0 0},
            breaklinks=true}
\urlstyle{same}  % don't use monospace font for urls
\usepackage{natbib}
\bibliographystyle{apalike}
\usepackage{color}
\usepackage{fancyvrb}
\newcommand{\VerbBar}{|}
\newcommand{\VERB}{\Verb[commandchars=\\\{\}]}
\DefineVerbatimEnvironment{Highlighting}{Verbatim}{commandchars=\\\{\}}
% Add ',fontsize=\small' for more characters per line
\usepackage{framed}
\definecolor{shadecolor}{RGB}{248,248,248}
\newenvironment{Shaded}{\begin{snugshade}}{\end{snugshade}}
\newcommand{\KeywordTok}[1]{\textcolor[rgb]{0.13,0.29,0.53}{\textbf{#1}}}
\newcommand{\DataTypeTok}[1]{\textcolor[rgb]{0.13,0.29,0.53}{#1}}
\newcommand{\DecValTok}[1]{\textcolor[rgb]{0.00,0.00,0.81}{#1}}
\newcommand{\BaseNTok}[1]{\textcolor[rgb]{0.00,0.00,0.81}{#1}}
\newcommand{\FloatTok}[1]{\textcolor[rgb]{0.00,0.00,0.81}{#1}}
\newcommand{\ConstantTok}[1]{\textcolor[rgb]{0.00,0.00,0.00}{#1}}
\newcommand{\CharTok}[1]{\textcolor[rgb]{0.31,0.60,0.02}{#1}}
\newcommand{\SpecialCharTok}[1]{\textcolor[rgb]{0.00,0.00,0.00}{#1}}
\newcommand{\StringTok}[1]{\textcolor[rgb]{0.31,0.60,0.02}{#1}}
\newcommand{\VerbatimStringTok}[1]{\textcolor[rgb]{0.31,0.60,0.02}{#1}}
\newcommand{\SpecialStringTok}[1]{\textcolor[rgb]{0.31,0.60,0.02}{#1}}
\newcommand{\ImportTok}[1]{#1}
\newcommand{\CommentTok}[1]{\textcolor[rgb]{0.56,0.35,0.01}{\textit{#1}}}
\newcommand{\DocumentationTok}[1]{\textcolor[rgb]{0.56,0.35,0.01}{\textbf{\textit{#1}}}}
\newcommand{\AnnotationTok}[1]{\textcolor[rgb]{0.56,0.35,0.01}{\textbf{\textit{#1}}}}
\newcommand{\CommentVarTok}[1]{\textcolor[rgb]{0.56,0.35,0.01}{\textbf{\textit{#1}}}}
\newcommand{\OtherTok}[1]{\textcolor[rgb]{0.56,0.35,0.01}{#1}}
\newcommand{\FunctionTok}[1]{\textcolor[rgb]{0.00,0.00,0.00}{#1}}
\newcommand{\VariableTok}[1]{\textcolor[rgb]{0.00,0.00,0.00}{#1}}
\newcommand{\ControlFlowTok}[1]{\textcolor[rgb]{0.13,0.29,0.53}{\textbf{#1}}}
\newcommand{\OperatorTok}[1]{\textcolor[rgb]{0.81,0.36,0.00}{\textbf{#1}}}
\newcommand{\BuiltInTok}[1]{#1}
\newcommand{\ExtensionTok}[1]{#1}
\newcommand{\PreprocessorTok}[1]{\textcolor[rgb]{0.56,0.35,0.01}{\textit{#1}}}
\newcommand{\AttributeTok}[1]{\textcolor[rgb]{0.77,0.63,0.00}{#1}}
\newcommand{\RegionMarkerTok}[1]{#1}
\newcommand{\InformationTok}[1]{\textcolor[rgb]{0.56,0.35,0.01}{\textbf{\textit{#1}}}}
\newcommand{\WarningTok}[1]{\textcolor[rgb]{0.56,0.35,0.01}{\textbf{\textit{#1}}}}
\newcommand{\AlertTok}[1]{\textcolor[rgb]{0.94,0.16,0.16}{#1}}
\newcommand{\ErrorTok}[1]{\textcolor[rgb]{0.64,0.00,0.00}{\textbf{#1}}}
\newcommand{\NormalTok}[1]{#1}
\usepackage{longtable,booktabs}
\usepackage{graphicx,grffile}
\makeatletter
\def\maxwidth{\ifdim\Gin@nat@width>\linewidth\linewidth\else\Gin@nat@width\fi}
\def\maxheight{\ifdim\Gin@nat@height>\textheight\textheight\else\Gin@nat@height\fi}
\makeatother
% Scale images if necessary, so that they will not overflow the page
% margins by default, and it is still possible to overwrite the defaults
% using explicit options in \includegraphics[width, height, ...]{}
\setkeys{Gin}{width=\maxwidth,height=\maxheight,keepaspectratio}
\IfFileExists{parskip.sty}{%
\usepackage{parskip}
}{% else
\setlength{\parindent}{0pt}
\setlength{\parskip}{6pt plus 2pt minus 1pt}
}
\setlength{\emergencystretch}{3em}  % prevent overfull lines
\providecommand{\tightlist}{%
  \setlength{\itemsep}{0pt}\setlength{\parskip}{0pt}}
\setcounter{secnumdepth}{5}
% Redefines (sub)paragraphs to behave more like sections
\ifx\paragraph\undefined\else
\let\oldparagraph\paragraph
\renewcommand{\paragraph}[1]{\oldparagraph{#1}\mbox{}}
\fi
\ifx\subparagraph\undefined\else
\let\oldsubparagraph\subparagraph
\renewcommand{\subparagraph}[1]{\oldsubparagraph{#1}\mbox{}}
\fi

%%% Use protect on footnotes to avoid problems with footnotes in titles
\let\rmarkdownfootnote\footnote%
\def\footnote{\protect\rmarkdownfootnote}

%%% Change title format to be more compact
\usepackage{titling}

% Create subtitle command for use in maketitle
\newcommand{\subtitle}[1]{
  \posttitle{
    \begin{center}\large#1\end{center}
    }
}

\setlength{\droptitle}{-2em}

  \title{Practicos de Bioestadística 2}
    \pretitle{\vspace{\droptitle}\centering\huge}
  \posttitle{\par}
    \author{Derek Corcoran}
    \preauthor{\centering\large\emph}
  \postauthor{\par}
      \predate{\centering\large\emph}
  \postdate{\par}
    \date{2019-03-13}

\usepackage{booktabs}

\begin{document}
\maketitle

{
\setcounter{tocdepth}{1}
\tableofcontents
}
\chapter*{Requerimientos}\label{requerimientos}
\addcontentsline{toc}{chapter}{Requerimientos}

Para comenzar el trabajo se necesita la última versión de R y RStudio
\citep{R-base}.También se requiere de los paquetes \emph{pacman},
\emph{rmarkdown}, \emph{tidyverse} y \emph{tinytex}. Si no se ha usado R
o RStudio anteriormente, el siguiente video muestra cómo instalar ambos
programas y los paquetes necesarios para este curso en el siguiente
\href{https://youtu.be/RtkCAKXsVbw}{link}.

El código para la instalación de esos paquetes es el siguiente:

\begin{Shaded}
\begin{Highlighting}[]
\KeywordTok{install.packages}\NormalTok{(}\StringTok{"pacman"}\NormalTok{, }\StringTok{"rmarkdown"}\NormalTok{, }\StringTok{"tidyverse"}\NormalTok{, }\StringTok{"tinytex"}\NormalTok{)}
\end{Highlighting}
\end{Shaded}

En caso de necesitar ayuda para la instalación, contactarse con el
instructor del curso.

\section{Antes de comenzar}\label{antes-de-comenzar}

Si nunca se ha trabajado con \texttt{R} antes de este curso, una buena
herramienta es provista por el paquete
\href{http://swirlstats.com/students.html}{Swirl} \citep{Kross2017}. Si
deseas estar más preparado para el curso, realiza los primeros 7 módulos
del programa \emph{R Programming: The basics of programming in R} que
incluye:

\begin{itemize}
\tightlist
\item
  Basic Building Blocks
\item
  Workspace and Files
\item
  Sequences of Numbers
\item
  Vectors
\item
  Missing Values
\item
  Subsetting Vectors
\item
  Matrices and Data Frames
\end{itemize}

El siguiente link muestra un video explicativo de cómo usar el paquete
swirl \href{https://youtu.be/w6L7Ye18yPE}{Video}

\section{Descripción del práctico}\label{descripcion-del-practico}

Los prácticos de este curso se enfocan en aprender a realizar de manera
práctica los conceptos enseñados en el cuso, pero además, usando
herramientas interactivas y/o programáticas, el profundizar el
entendimiento de ciertos conceptos teóricos y filosóficos del curso.

\section{Objetivos del práctico}\label{objetivos-del-practico}

\begin{enumerate}
\def\labelenumi{\arabic{enumi}.}
\item
  Aprender el uso de R como ambiente estadístico de limpieza,
  exploración, visualización de datos.
\item
  Conocer y aplicar de manera aplicada los conceptos enseñados en el
  curso de Bioestadística 2.
\item
  Aprender buenas prácticas de recolección y estandarización de bases de
  datos, con la finalidad de optimizar el análisis de datos y la
  revisión de éstas por pares.
\item
  Realizar análisis críticos de la naturaleza de los datos al realizar
  análisis exploratorios, que permitirán determinar la mejor forma de
  comprobar hipótesis asociadas a estas bases de datos.
\end{enumerate}

\section{Contenidos}\label{contenidos}

\begin{itemize}
\item
  Capítulo \ref{Explorando} \emph{Análisis exploratorio y el primer
  ANOVA}: En este capítulo se aprenderá a cómo explorar, resumir y
  visualizar una base de datos utilizando el paquete tidyverse
  \citep{WickhamTidy2017}, además se realizarán un análisis básico de
  ANOVA
\item
  Capítulo \ref{Supuestos} \emph{Supuestos de ANOVA y mínimos cuadrados}
\item
  Capítulo \ref{Poder} \emph{Análisis de poder y primera tarea}
\item
  Capítulo \ref{Refs} \emph{Referencias}
\item
  Capítulo \ref{t-student} \emph{T de student}
\end{itemize}

\section{Metodología}\label{metodologia}

Clases prácticas donde cada estudiante trabajará con datos entregados
para desarrollar análisis de datos. Además, se deberán generar informes,
en base al trabajo con sus datos.

\section{Evaluación}\label{evaluacion}

El trabajo práctico de este ramo es un 20\% de la nota final del curso,
y es obligatorio ir a todos los trabajos prácticos para pasar el ramo.

Durante los primeros 15 minutos se tomará un control. Pasado ese
período, no se acepta la entrega de controles, recibiendo calificación
1. La ausencia a los trabajos prácticos puede ser causal de reprobación
del curso. Ademas de los controles habrán trabajos de investigación.

La nota final de los practicos se evaluará de la siguiente forma:

\begin{itemize}
\tightlist
\item
  Tests de entrada: 60\%
\item
  Trabajos: 40\%
\end{itemize}

\section{Presentación de
introducción}\label{presentacion-de-introduccion}

Para la introducción de los prácticos seguiremos un a presentación que
se encuentra en este
\href{http://www.derek-corcoran-barrios.com/AyudantiaStatsPres/Clase1/Clase1.html}{link}

\chapter{Exploración de datos y tu primer ANOVA}\label{Explorando}

\section{Actividad 1 Educación en
Chile}\label{actividad-1-educacion-en-chile}

En esta actividad exploraremos los resultados de la PSU en Chile para el
año 2017. Pueden encontrar la base de datos original en
\href{https://es.datachile.io/geo/chile\#education}{Data Chile}.

Trataremos de determinar, usando el puntaje de la PSU como medida, si
existen brechas en la educación chilena por tipo de institución. Para
ello, primero trabajaremos realizando análisis exploratorios en base a
gráficos y tablas resumen usando funciones del paquete \emph{tidyverse}
\citep{WickhamTidy2017} en R.

La base de datos \emph{EducacionChile.csv} se encuentra disponible en
webcursos o en \url{https://es.datachile.io/geo/chile\#education}.

\subsection{Tablas resumen de los
datos:}\label{tablas-resumen-de-los-datos}

Lo primero que deben hacer es generar una tabla resumen usando el
\emph{tidyverse} usando las funciones \emph{group\_by} para agrupar por
variables y \emph{summarize} para resumir los datos, dentro de summarize
podemos usar variables como:

\begin{itemize}
\tightlist
\item
  \textbf{mean()} promedio
\item
  \textbf{sd()} desviación estándar
\item
  \textbf{n()} número de muestras
\end{itemize}

a modo de ejemplo vemos la tabla \ref{tab:MediaIris} mostrando la media
y número de muestras con la base de datos iris:

\begin{Shaded}
\begin{Highlighting}[]
\KeywordTok{data}\NormalTok{(}\StringTok{"iris"}\NormalTok{)}
\NormalTok{Table <-}\StringTok{ }\KeywordTok{group_by}\NormalTok{(iris, Species) }\OperatorTok\StringTok{ }\KeywordTok{summarize}\NormalTok{(}\DataTypeTok{Promedio =} \KeywordTok{mean}\NormalTok{(Petal.Length), }\DataTypeTok{N =} \KeywordTok{n}\NormalTok{())}
\end{Highlighting}
\end{Shaded}

\begin{Shaded}
\begin{Highlighting}[]
\NormalTok{knitr}\OperatorTok{::}\KeywordTok{kable}\NormalTok{(Table)}
\end{Highlighting}
\end{Shaded}

\begin{table}

\caption{\label{tab:MediaIris}Resumen con la media y número de muestras del largo de pétalo de las flores de tres especies del género Iris}
\centering
\begin{tabular}[t]{lrr}
\toprule
Species & Promedio & N\\
\midrule
setosa & 1.462 & 50\\
versicolor & 4.260 & 50\\
virginica & 5.552 & 50\\
\bottomrule
\end{tabular}
\end{table}

Basado en el resumen ¿Qué podemos decir de estos datos de educación en
Chile?

\subsection{Visualización de datos con ggplot2
(tidyverse)}\label{visualizacion-de-datos-con-ggplot2-tidyverse}

El paquete \emph{ggplot2} \citep{WickhamGG2016} es una poderosa
herramienta para graficar datos. Si desean ahondar en el uso de este
paquete, pueden ver el siguiente link
\url{http://zevross.com/blog/2014/08/04/beautiful-plotting-in-r-a-ggplot2-cheatsheet-3/}.
En este caso, aprenderemos a graficar \emph{boxplots} y
\emph{jitterplots}, dos opciones para visualizar una variable categórica
versus una cuantitativa.

\subsubsection{Uso del ggplot2}\label{uso-del-ggplot2}

Su función principal es \emph{ggplot}, luego de cada función usaremos el
símbolo \textbf{+} como usábamos el pipeline (\%\textgreater{}\%).

Primero usamos la función ggplot para determinar la base de datos y
variables, acá las variables siempre van dentro de la función aes

\begin{Shaded}
\begin{Highlighting}[]
\KeywordTok{ggplot}\NormalTok{(MiBaseDeDatos, }\KeywordTok{aes}\NormalTok{(}\DataTypeTok{x =}\NormalTok{ VariableX, }\DataTypeTok{y =}\NormalTok{ VariableY)) }
\end{Highlighting}
\end{Shaded}

Luego agregamos el tipo de gráfico que queremos para nuestra figura
usando el \textbf{+} como pipeline

\begin{Shaded}
\begin{Highlighting}[]
\KeywordTok{ggplot}\NormalTok{(MiBaseDeDatos, }\KeywordTok{aes}\NormalTok{(}\DataTypeTok{x =}\NormalTok{ VariableX, }\DataTypeTok{y =}\NormalTok{ VariableY)) }\OperatorTok{+}\StringTok{ }\KeywordTok{geom_boxplot}\NormalTok{()}
\end{Highlighting}
\end{Shaded}

\subsubsection{Ejemplo usando la base de datos
iris}\label{ejemplo-usando-la-base-de-datos-iris}

\paragraph{Boxplot}\label{boxplot}

El siguiente código muestra como graficar un boxplot para la base de
datos iris, la cual esta en R. En este caso graficaremos el largo del
pétalo para cada especie (Figura \ref{fig:BoxplotIris}).

\begin{Shaded}
\begin{Highlighting}[]
\KeywordTok{data}\NormalTok{(}\StringTok{"iris"}\NormalTok{)}
\KeywordTok{ggplot}\NormalTok{(iris, }\KeywordTok{aes}\NormalTok{(}\DataTypeTok{x =}\NormalTok{ Species, }\DataTypeTok{y =}\NormalTok{ Petal.Length)) }\OperatorTok{+}\StringTok{ }\KeywordTok{geom_boxplot}\NormalTok{()}
\end{Highlighting}
\end{Shaded}

\begin{figure}
\centering
\includegraphics{AyduantiaStats_files/figure-latex/BoxplotIris-1.pdf}
\caption{\label{fig:BoxplotIris}Box plot del largo de petalo de tres
especies del género Iris}
\end{figure}

En los Box Plots tenemos 4 visualizaciones:

\begin{itemize}
\tightlist
\item
  Mediana (linea gruesa)
\item
  Caja (Cuantiles 25\% y 75\%)
\item
  Bigotes (intervalo de confianza del 95\%)
\item
  Puntos Outlayers
\end{itemize}

Realice un boxplot de los datos de la educación de Chile, ¿Qué nos dice
esto de los datos?

\paragraph{Jitter plot}\label{jitter-plot}

El jitter plot suma un punto por cada observación, lo cual nos permite
entender un poco más la naturaleza de los datos. En general se le agrega
a un box plot para tener mayor claridad en los datos (Figura
\ref{fig:JitterIris}).

\begin{Shaded}
\begin{Highlighting}[]
\KeywordTok{data}\NormalTok{(}\StringTok{"iris"}\NormalTok{)}
\KeywordTok{ggplot}\NormalTok{(iris, }\KeywordTok{aes}\NormalTok{(}\DataTypeTok{x =}\NormalTok{ Species, }\DataTypeTok{y =}\NormalTok{ Petal.Length)) }\OperatorTok{+}\StringTok{ }\KeywordTok{geom_boxplot}\NormalTok{() }\OperatorTok{+}\StringTok{ }\KeywordTok{geom_jitter}\NormalTok{(}\KeywordTok{aes}\NormalTok{(}\DataTypeTok{color =}\NormalTok{ Species))}
\end{Highlighting}
\end{Shaded}

\begin{figure}
\centering
\includegraphics{AyduantiaStats_files/figure-latex/JitterIris-1.pdf}
\caption{\label{fig:JitterIris}Box plot y jitter plot juntos para el largo
de petalo de tres especies del género Iris}
\end{figure}

\section{Actividad 2 Captación de CO2 en
plantas}\label{actividad-2-captacion-de-co2-en-plantas}

Utilizaremos base de datos \(CO_2\) \citep{potvin1990statistical}
enviada al curso. Esta base de datos, también presente en R, tiene las
siguientes variables

\begin{itemize}
\tightlist
\item
  \textbf{Plant}: Identidad de cada planta
\item
  \textbf{Type}: Variedad de la planta (subespecie Quebec o Mississippi)
\item
  \textbf{Treatment}: Tratamiento de la planta, algunas fueron enfriadas
  la noche anterior (Chilled)
\item
  \textbf{conc}: Concentración ambiental de \(CO_2\)
\item
  \textbf{Uptake}: Captación de \(CO_2\) para cada planta en cada día
\end{itemize}

¿Hay diferencias entre la captación de \(CO_2\) en plantas tratadas y no
tratadas?

\begin{itemize}
\tightlist
\item
  Genere tablas resumenes que le permitan explorar esta pregunta

  \begin{itemize}
  \tightlist
  \item
    ¿Existen variables que puedan confundir el resultado? ¿como trataría
    los datos para lidiar con esto?
  \end{itemize}
\item
  Genere gráficos exploratorios para contestar esta pregunta
\end{itemize}

\section{Actividad 3 Mi primer ANOVA}\label{actividad-3-mi-primer-anova}

\subsection{antes de empezar a entender el
ANOVA}\label{antes-de-empezar-a-entender-el-anova}

\subsection{Como hacer un ANOVA en R}\label{como-hacer-un-anova-en-r}

En \emph{R} todos los modelos tienen la siguiente estructura
\textbf{Funcion(y \textasciitilde{} x1 + x2 + \ldots{} + xn, data =
MisDatos)}, donde la \textbf{Funcion} dice el modelo que queremos
realizar (por ejemplo ANOVA, regresión lineal, modelos mixtos, etc.),
\textbf{y} es la variable que queremos explicar, \textbf{x1} a
\textbf{xn} son las variables explicativas, \textbf{\textasciitilde{}}
es un símbolo que debe ser leído como explicado por y finalmente
\textbf{data} es la base de datos que queremos utilizar, en un ANOVA
(análisis de varianza), la función en cuestión es aov.

En el siguiente código vemos si el largo del pétalo de las flores del
género \emph{Iris}, pueden ser explicados por la especie a la que estas
plantas pertenecen, por lo que generamos un modelo llamado
\emph{Primer.Anova} con la función \textbf{aov}.

\begin{Shaded}
\begin{Highlighting}[]
\NormalTok{Primer.Anova <-}\StringTok{ }\KeywordTok{aov}\NormalTok{(Petal.Length }\OperatorTok{~}\StringTok{ }\NormalTok{Species, }\DataTypeTok{data =}\NormalTok{ iris)}
\end{Highlighting}
\end{Shaded}

Para acceder a la tabla de resultados utilizamos la función
\textbf{summary}

\begin{verbatim}
##              Df Sum Sq Mean Sq F value Pr(>F)    
## Species       2  437.1  218.55    1180 <2e-16 ***
## Residuals   147   27.2    0.19                   
## ---
## Signif. codes:  0 '***' 0.001 '**' 0.01 '*' 0.05 '.' 0.1 ' ' 1
\end{verbatim}

Si establecemos el valor de alfa en 0.05 y al ver en la tabla que el
valor de p es menor a alfa, rechazamos la hipótesis nula de que las
medias son iguales, y decidimos que la media del largo de pétalo es
distinta entre las especies.

\subsection{Ejercicio}\label{ejercicio}

Determine si para la base de datos \textbf{CO2} la captación de \(CO_2\)
es distinto entre plantas con tratamiento de enfriamiento y sin
enfriamiento.

\subsubsection{Simulador de ANOVA}\label{simulador-de-anova}

\chapter{Supuestos de ANOVA y mínimos cuadrados}\label{Supuestos}

\section{Objetivos de este práctico}\label{objetivos-de-este-practico}

\begin{itemize}
\tightlist
\item
  Entender los supuestos de un ANOVA de una vía (independencia,
  aleatoriedad, homocedasticidad y normalidad)
\item
  Entender el concepto de mínimos cuadrados
\item
  Saber cuando realizar un ANOVA e interpretar sus resultados
\end{itemize}

\section{Actividad 1 Sueño en
mamíferos}\label{actividad-1-sueno-en-mamiferos}

En esta actividad intentaremos ver si hay diferencias en horas de sueño
en mamíferos por Orden o dieta. Los datos fueron extraídos del trabajo
de \citet{savage2007quantitative} y están incorporados en la base de
datos de \emph{ggplot2} con el nombre de \emph{msleep}, pero estarán en
webcursos en formato csv de todas formas. Para la guía los ejemplos se
generarán en base a la base de datos \emph{InsectSprays} que está en
\emph{R} y que fue extraída de \citet{beall1942transformation}, en la
cual se testean la efectividad de insecticidas en Spray en la abundancia
de insectos en plantaciones. Y en la base de datos \emph{iris} que ya
fue entregada, en la que se miden distintas características florales de
especies del genero \emph{Iris} \citep{anderson1935irises}.

\subsection{Homogeneidad de varianza}\label{homogeneidad-de-varianza}

\subsubsection{Inspección visual}\label{inspeccion-visual}

Lo primero que intentaremos explorar de forma visual y a partir de tests
si es que hay homogeneidad de varianza, para esto usaremos boxplots, y
jitter plots (Figura \ref{fig:Visual}), lo cual ya hemos hecho
anteriormente:

\begin{Shaded}
\begin{Highlighting}[]
\KeywordTok{ggplot}\NormalTok{(InsectSprays, }\KeywordTok{aes}\NormalTok{(}\DataTypeTok{x =}\NormalTok{ spray, }\DataTypeTok{y =}\NormalTok{ count)) }\OperatorTok{+}\StringTok{ }\KeywordTok{geom_boxplot}\NormalTok{() }\OperatorTok{+}\StringTok{ }\KeywordTok{geom_jitter}\NormalTok{(}\KeywordTok{aes}\NormalTok{(}\DataTypeTok{color =}\NormalTok{ spray)) }
\end{Highlighting}
\end{Shaded}

\begin{figure}
\centering
\includegraphics{AyduantiaStats_files/figure-latex/Visual-1.pdf}
\caption{\label{fig:Visual}Cuenta de insectos según tipo de insecticida}
\end{figure}

Para explorar visualmente si existe homogeneidad de varianza, se
compraran las cajas y bigotes de los boxplots y se espera que tengan
(Mas o menos distintos tamaños).

\subsubsection{Test de Bartlett}\label{test-de-bartlett}

Para realizar un test de homogeneidad de varianza se realiza el test de
bartlett \citep{bartlett1937properties}, en este se usa nuestra conocida
formula \emph{y \textasciitilde{} x}, esto es, y explicado por x junto a
la función \emph{bartlett.test}. Para nuestro caso usaríamos:

\begin{verbatim}
## 
##  Bartlett test of homogeneity of variances
## 
## data:  count by spray
## Bartlett's K-squared = 25.96, df = 5, p-value = 9.085e-05
\end{verbatim}

Como en este caso, no el valor de p es menor a 0.05, decimos que no hay
homogeneidad de varianza, por lo que no podemos hacer el test.

\subsection{Normalidad de los
residuales}\label{normalidad-de-los-residuales}

En el caso de la base de datos \emph{iris}, demostraremos inmediatamente
que si hay homogeneidad de varianza en el ancho del sépalo (Figura
\ref{fig:IrisBox}):

\begin{figure}
\centering
\includegraphics{AyduantiaStats_files/figure-latex/IrisBox-1.pdf}
\caption{\label{fig:IrisBox}Ancho de sépalo según especie del género Iris}
\end{figure}

\begin{verbatim}
## 
##  Bartlett test of homogeneity of variances
## 
## data:  Sepal.Width by Species
## Bartlett's K-squared = 2.0911, df = 2, p-value = 0.3515
\end{verbatim}

Debido a ello, podemos testar si los residuales tienen una distribución
normalidad de los residuales, para esto lo primero que debemos hacer es
un ANOVA, como fue explicado en el práctico anterior y guardar este
objeto con un nombre:

\subsubsection{Extracción de los residuales del
modelo}\label{extraccion-de-los-residuales-del-modelo}

Para extraer los residuales, podemos hacerlo de dos formas, si solo
queremos un vector de sus valores, podemos extraerlo desde el modelo
mismo utilizando \emph{\$residuals}. Si queremos guardarlo en un
dataframe mas completo podemos utilizar la función \emph{augment} del
paquete \emph{broom}.

La segunda opción nos entregará más información que podremos utilizar
más tarde, pero ambas sirven para testear normalidad, la siguiente tabla
muestra las primeras 6 observaciones generadas por la función
\emph{augment}, donde \emph{resid}, son los residuales (Ver tabla
\ref{tab:TabResid}.

\begin{table}

\caption{\label{tab:TabResid}primeras 6 observaciones del dataframe resultante de augment}
\centering
\begin{tabular}[t]{r|l|r|r|r|r|r|r|r}
\hline
Sepal.Width & Species & .fitted & .se.fit & .resid & .hat & .sigma & .cooksd & .std.resid\\
\hline
3.5 & setosa & 3.428 & 0.048 & 0.072 & 0.02 & 0.341 & 0.000 & 0.214\\
\hline
3.0 & setosa & 3.428 & 0.048 & -0.428 & 0.02 & 0.339 & 0.011 & -1.273\\
\hline
3.2 & setosa & 3.428 & 0.048 & -0.228 & 0.02 & 0.340 & 0.003 & -0.678\\
\hline
3.1 & setosa & 3.428 & 0.048 & -0.328 & 0.02 & 0.340 & 0.006 & -0.975\\
\hline
3.6 & setosa & 3.428 & 0.048 & 0.172 & 0.02 & 0.341 & 0.002 & 0.511\\
\hline
3.9 & setosa & 3.428 & 0.048 & 0.472 & 0.02 & 0.339 & 0.013 & 1.404\\
\hline
\end{tabular}
\end{table}

\subsubsection{Inspección visual de los
residuales}\label{inspeccion-visual-de-los-residuales}

Existen dos formas de visualizar los residuales para determinar si la
distribución de estos es o no es normal, histogramas y el qqplot.

\paragraph{Histograma}\label{histograma}

Los histogramas nos darán una representación visual para tratar de
entender si la distribución es normal, para esto, solo necesitamos usar
el comando \emph{hist}, seguido del vector de los residuales, este es el
comando para hacer el histograma (Figura \ref{fig:histogram}) con
cualquiera de las dos bases de datos, el resultado debiera ser el mismo:

\begin{Shaded}
\begin{Highlighting}[]
\KeywordTok{hist}\NormalTok{(Residuales)}
\KeywordTok{hist}\NormalTok{(Resultados}\OperatorTok{$}\NormalTok{.resid)}
\end{Highlighting}
\end{Shaded}

\begin{figure}
\centering
\includegraphics{AyduantiaStats_files/figure-latex/histogram-1.pdf}
\caption{\label{fig:histogram}Histograma de los resiudales del modelo ANOVA}
\end{figure}

\paragraph{QQplot}\label{qqplot}

El qq plot es otra forma visual de establecer si los residuales son o no
son normales, para esto, lo esperado es que la gráfica resultante sea
una diagonal lo mas recta posible, para esto usaremos la función
\emph{qqnorm}, con nuestros residuales, de nuevo, podemos usar
cualquiera de las dos versiones de nuestros datos:

\begin{Shaded}
\begin{Highlighting}[]
\KeywordTok{qqnorm}\NormalTok{(Residuales)}
\KeywordTok{qqnorm}\NormalTok{(Resultados}\OperatorTok{$}\NormalTok{.resid)}
\end{Highlighting}
\end{Shaded}

\begin{figure}
\centering
\includegraphics{AyduantiaStats_files/figure-latex/unnamed-chunk-19-1.pdf}
\caption{\label{fig:unnamed-chunk-19}qqplot de los resiudales del modelo
ANOVA}
\end{figure}

\subsubsection{Test de Shapiro para determinar
normalidad}\label{test-de-shapiro-para-determinar-normalidad}

La forma más sencilla de determinar normalidad es usando el test de
Shapiro-Wilk de normalidad \citep{royston1995remark}. Al igual que el
test de Bartlett, si el valor de p es menor a 0.05, determinamos que la
distribución de los datos no son normales, la función en \emph{R} para
este test es \emph{shapiro.test}, y al igual que en los casos anteriores
de \emph{hist} y \emph{qqpot}, solo necesitamos de usar un vector de
residuales para ver el resultado del test. En nuestro caso:

\begin{Shaded}
\begin{Highlighting}[]
\KeywordTok{shapiro.test}\NormalTok{(Residuales)}
\KeywordTok{shapiro.test}\NormalTok{(Resultados}\OperatorTok{$}\NormalTok{.resid)}
\end{Highlighting}
\end{Shaded}

\begin{verbatim}
## 
##  Shapiro-Wilk normality test
## 
## data:  Residuales
## W = 0.98948, p-value = 0.323
\end{verbatim}

Ya que el valor de p es menor a 0.05, podemos decir que la distribución
de nuestros residuales es normal, y por lo tanto el test cumple con los
supuestos, y esto hace que sea valido el ANOVA, por lo que podemos ver
nuestros resultados. La homogeneidad de Varianza es mas importante que
la normalidad de residuales para estos casos, para ejemplos de lo que se
debe hacer si se viola la normalidad ver \citet{lix1996consequences}

\section{Actividad 2 Suma de
cuadrados}\label{actividad-2-suma-de-cuadrados}

Tanto los ANOVAS como las regresiones lineales se basan en minimizar la
suma de cuadrados, es la suma de los cuadrados de los errores o
residuales.

\subsection{¿Que es el error? ¿Por qué al
cuadrado??}\label{que-es-el-error-por-que-al-cuadrado}

\begin{figure}
\centering
\includegraphics{AyduantiaStats_files/figure-latex/unnamed-chunk-22-1.pdf}
\caption{\label{fig:unnamed-chunk-22}Errores de una regresión lineal
ejemplificados con la linea entre el valor predicho y el observado}
\end{figure}

En la figura y en la formula vemos ejemplificado que es el error,
también conocido como residual, este es simplemente el valor observado

\[Observado - Predicho\]

El objetivo de todo modelo es el de minimizar estos errores, al ajustar
el mejor modelo posible.

Los errores siempre se calculan al cuadrado, discutiremos por que en
clase

\[\sum_{i=1}^{n} (Observado - Predicho)^2\]

\section{Referencias}\label{referencias}

\chapter{Análisis de poder y primera tarea}\label{Poder}

\section{Obejtivos del práctico}\label{obejtivos-del-practico}

\begin{itemize}
\tightlist
\item
  Entender cálculos de poder en base a matriz de confusión
\item
  Primera tarea de práctico
\end{itemize}

\section{Matriz de confusión}\label{matriz-de-confusion}

La matriz de confusión es una herramienta de toma de decisiones, en el
caso especial de la toma de decisiones tenemos la siguiente matriz de
confusión (Tabla \ref{tab:errores})

\begin{table}

\caption{\label{tab:errores}Tabla de confusión de errores}
\centering
\begin{tabular}[t]{l|l|l}
\hline
  & Hipótesis nula cierta & Hipótesis alternativa cierta\\
\hline
Acepto hipótesis nula & No hay error & Error tipo 2\\
\hline
Acepto hipótesis alternativa & Error tipo 1 & No hay error\\
\hline
\end{tabular}
\end{table}

Esto puede ser fácilmente ejemplificado con el problema de una alarma de
humo (tabla\ref{tab:Confucion}), en este caso cuando la alarma suena y
no hay fuego y suena la alarma tenemos un error de tipo 1, en cambio si
hay fuego y la alarma no suena tenemos un error de tipo 2

\begin{table}

\caption{\label{tab:Confucion}Matriz de confusión de una alarma de incendio}
\centering
\begin{tabular}[t]{lll}
\toprule
  & No hay fuego & Hay fuego\\
\midrule
No suena alarma & No hay error & Error tipo 2\\
Suena alarma & Error tipo 1 & No hay error\\
\bottomrule
\end{tabular}
\end{table}

\subsection{Poder y matriz de
confusión}\label{poder-y-matriz-de-confusion}

\begin{itemize}
\tightlist
\item
  Probabilidad de que suene la alarma cuando no hay fuego

  \begin{itemize}
  \tightlist
  \item
    \(\alpha\) usualmente 5\%
  \item
    una de cada 20 alarmas es falsa
  \item
    ¿Cuál es el \(\alpha\) de una alarma de auto?
  \end{itemize}
\item
  Probabilidad de que no suene la alarma cuando hay fuego

  \begin{itemize}
  \tightlist
  \item
    \(\beta\) si es 10\% uno de cada 10 fuegos no es detectado
  \item
    poder es \(1-\beta\) confianza de que fuegos son detectados
  \end{itemize}
\end{itemize}

\section{Calculo de poder en R}\label{calculo-de-poder-en-r}

Para hacer cálculos de poder en ANOVAS de una y dos vías en \emph{R},
utilizamos el paquete \emph{pwr2} \citep{Pengcheng2017}. En este paquete
podemos utilizar la función \emph{pwr.1way} para determinar el poder de
un ANOVA de una vía, los argumentos de esta función son:

\begin{itemize}
\tightlist
\item
  \emph{K}: El número de grupos a testear
\item
  \emph{n}: Número de individuos por grupo
\item
  \emph{Alpha}: Nivel de significancia
\item
  \emph{Delta}: Valor mínimo a detectar
\item
  \emph{Sigma}: Desviación estándar de la muestra
\end{itemize}

Para cálculos precisos de n necesarios para muestras usar la siguiente
app

\section{Tarea}\label{tarea}

\subsection{El problema}\label{el-problema}

Una compañía que genera pesticidas descarga parte de sus desechos a un
río. La ONG \textbf{RioSano}, dice que ha notado una alza en la
mortalidad de los patos cortacorriente (\emph{Merganneta armata}) del
río.

Ante esto la empresa contrata un científico, el cual hace una estimación
de la mortalidad de patos en 10 zonas del río en que descargan sus
desechos, y lo compara con otros dos ríos no contaminados. Este
científico dice que no hay diferencias significativas en la mortalidad
de los patos de los ríos con desechos y sin desechos con una confianza
del 95\%. Para esto muestra como evidencia la figura 1 y tabla 3 e
incluso hace públicos sus datos en el archivo \emph{MuestraPatos.csv}.

\begin{figure}
\centering
\includegraphics{AyduantiaStats_files/figure-latex/unnamed-chunk-28-1.pdf}
\caption{\label{fig:unnamed-chunk-28}Mortalidades calculadas en 10 zonas de
tres ríos}
\end{figure}

\begin{table}

\caption{\label{tab:unnamed-chunk-29}Tabla de ANOVA de una vía de la mortalidad de patos de los tres ríos}
\centering
\begin{tabular}[t]{l|r|r|r|r|r}
\hline
term & df & sumsq & meansq & statistic & p.value\\
\hline
rio & 2 & 301.2531 & 150.62656 & 2.359899 & 0.1136188\\
\hline
Residuals & 27 & 1723.3436 & 63.82754 & NA & NA\\
\hline
\end{tabular}
\end{table}

La ONG \emph{RioSano} lo contrata para determinar la validez del estudio
y si es necesario generar un estudio extra. Ante esto:

\begin{enumerate}
\def\labelenumi{\arabic{enumi}.}
\item
  Genere una matriz de confusión del problema y explique en este
  contexto que significaría el alfa y beta para este problema, y cual
  consideraría más relevante.
\item
  Diseñe el estudio que le gustaría hacer, determinando cuantas áreas
  debe muestrear por río, estime un delta mínimo que le gustaría
  determinar y el beta con el que se siente seguro y determine el
  \emph{n} mínimo necesario para ese estudio. Justifique su respuesta
\item
  Dado este \emph{n} mínimo realice lo siguiente

  \begin{itemize}
  \tightlist
  \item
    Realice un muestreo de n muestras por tipo de río del archivo
    \emph{Patos.csv}
  \item
    Genere gráficos y tablas exploratorias de los datos de su muestreo y
    describalas
  \item
    Revise los supuestos del ANOVA para su base de datos tanto
    gráficamente como con tests y determine si se puede realizar el
    anova
  \item
    Diga si según su diseño hay diferencias significativas en la
    mortalidad de patos entre los ríos
  \end{itemize}
\item
  Cada zona a muestrear requiere de un monitoreo exhaustivo, que tiene
  un costo de 500.000 pesos (esto es 1.500.000 de pesos si consideramos
  los 3 ríos). La ONG \emph{RioSano} consiguió 20.000.000 de pesos para
  este estudio. Dadas esas limitaciones, genere un balance de
  \(\alpha\), \(\beta\) y \(n\) dada esa limitación para hacer el mejor
  estudio posible dadas las consecuencias, justifique su respuesta.
\end{enumerate}

Genere un informe para la ONG \emph{RioSano} incorporando estos 5 puntos
e incluya una introducción, metodología, resultados,
discusión-conclusión y bibliografía, envíe el script de como generó los
resultados

\chapter{Prueba t de Student}\label{t-student}

Puedes encontrar una versión interactiva de esta guía
\href{http://admin.derek-corcoran-barrios.com/shiny/rstudio/sample-apps/Interactivo5/}{aquí}.

La prueba t de student fue desarrollada por Gosset cuando trabajaba para
la cervecería Guinness \citep{student1908probable}. Esta prueba permite
comparar las medias de una muestra con la media teórica de una
población, o comparar dos poblaciones. Una de las características de la
prueba de student, es que permite la alternativa de ver si dos medias
son diferentes o, si uno busca más confianza determinar si una media es
mayor, o menor que otra. Para la prueba t de Student, se determina un
valor de t, usando la siguiente formula (ecuación \eqref{eq:tStud}):

\begin{equation} 
  t = \frac{(\bar{x} - \mu)/(\frac{\sigma}{\sqrt{n}})}{s}
  \label{eq:tStud}
\end{equation}

El estadístico \(t\) posee un valor de p asociado dependiendo de los
grados de libertad de la prueba.

\subsection{Pruebas de una muestra}\label{pruebas-de-una-muestra}

Las pruebas de una muestra nos permiten poner a prueba si la media de
una población son distintas a una media teórica. Como ejemplo veremos el
caso de las erupciones del géiser \emph{Old Faithful}, localizado en el
Parque Nacional Yellowstone. Un guarda-parque del lugar dice que este
géiser erupta cada 1 hora. Por suerte \emph{R} posee una base de datos
de \citet{azzalini1990look} llamada \emph{faithfull}, la cual
utilizaremos para determinar si esto es cierto o no usando la función
\texttt{t.test}. Esta base de datos tiene dos columnas \emph{eruptions},
que muestra la duración en minutos de cada erupción y \emph{waiting} que
presenta la espera en minutos entre erupciones.

Cuando usamos esta función con una muestra necesitamos llenar 2
argumentos:

\begin{itemize}
\tightlist
\item
  \textbf{x:} Un vector con los valores numéricos de a poner a prueba
\item
  \textbf{mu:} La media teórica a poner a prueba
\item
  \textbf{alternative:} Puede ser ``two.sided'', ``less'' o ``greater'',
  dependiendo de si uno quiere probar que la muestra posee una media
  distinta, menor o mayor que la media teórica.
\end{itemize}

En este caso haríamos lo siguiente

\begin{Shaded}
\begin{Highlighting}[]
\KeywordTok{data}\NormalTok{(}\StringTok{"faithful"}\NormalTok{)}
\KeywordTok{t.test}\NormalTok{(}\DataTypeTok{x =}\NormalTok{ faithful}\OperatorTok{$}\NormalTok{waiting, }\DataTypeTok{mu =} \DecValTok{60}\NormalTok{, }\DataTypeTok{alternative =} \StringTok{"two.sided"}\NormalTok{)}
\end{Highlighting}
\end{Shaded}

\begin{verbatim}
## 
##  One Sample t-test
## 
## data:  faithful$waiting
## t = 13.22, df = 271, p-value < 2.2e-16
## alternative hypothesis: true mean is not equal to 60
## 95 percent confidence interval:
##  69.27418 72.51994
## sample estimates:
## mean of x 
##  70.89706
\end{verbatim}

En este caso el valor de p nos dice que la media es diferente a 60.

\subsubsection{Ejercicio 1}\label{ejercicio-1}

La base de datos \emph{airquality} (incorporada como ejemplo en
\textbf{R}), muestra entre otras variables las partículas de ozono en
Nueva York, cada día de Mayo a Septiembre de 1973 entre las 13:00 y las
15:00 \citep{chambers35graphical}. Supongamos que ustedes están a cargo
de una agencia ambiental, y están estudiando en que meses deben reducir
la actividad vehicular de Nueva York. Para esto planean disminuir a la
mitad los pasajes del metro de Nueva York todos los meses que en
promedio tengan sobre 55 ppb. Para esto deben comprobar estadisticamente
que el mes en que harán esto tiene promedios sobre 55.

\subsection{Pruebas de dos muestras}\label{pruebas-de-dos-muestras}

Las pruebas de dos muestras nos permiten ver si hay diferencias
significativas entre las medias de dos muestras. En la base de datos
\texttt{mtcars}, hay una columna que determina si los vehículos son de
cambios manuales o automáticos. En este caso 0 significa automático y 1
significa manual. En la figura \ref{fig:autom} podemos ver una
inspección gráfica de las posibles diferencias.

\begin{figure}
\centering
\includegraphics{AyduantiaStats_files/figure-latex/autom-1.pdf}
\caption{\label{fig:autom}Comparación de eficiencia entre vehiculos
automaticos y manuales}
\end{figure}

Para hacer la comparación debemos agregar el argumento
\texttt{var.equal} el cual en este caso asumiremos que es verdad, ya que
en la próxima sección veremos los supuestos de la prueba t y las
consecuencias de las violaciones de estos supuestos. En este caso
podemos usar el símbolo \texttt{\textasciitilde{}} a ser leído como
explicado por para la prueba t de dos muestras.

\begin{Shaded}
\begin{Highlighting}[]
\KeywordTok{t.test}\NormalTok{(mpg }\OperatorTok{~}\StringTok{ }\NormalTok{am, }\DataTypeTok{data =}\NormalTok{ mtcars, }\DataTypeTok{var.equal =}\OtherTok{TRUE}\NormalTok{)}
\end{Highlighting}
\end{Shaded}

\begin{verbatim}
## 
##  Two Sample t-test
## 
## data:  mpg by am
## t = -4.1061, df = 30, p-value = 0.000285
## alternative hypothesis: true difference in means is not equal to 0
## 95 percent confidence interval:
##  -10.84837  -3.64151
## sample estimates:
## mean in group 0 mean in group 1 
##        17.14737        24.39231
\end{verbatim}

En este caso se determinaría que los vehículos manuales (am = 1), son
más eficientes que sus contra-partes automáticas.

\subsubsection{Ejercicio 2}\label{ejercicio-2}

Para el siguiente ejercicio usaremos la base de datos \texttt{BeerDark}
disponible en webcursos o en el siguiente
\href{https://archive.org/download/BeerDark/BeerDark.csv}{link}. Esta
base de datos posee 7 columnas, pero usaremos solo 4 de ellas:

\begin{itemize}
\tightlist
\item
  \textbf{Estilo:} Separa las cervezas entre Porters y Stouts
\item
  \textbf{Grado\_Alcoholico:} El grado alcohólico de las cervezas
\item
  \textbf{Amargor:} Valor IBU (International Bittering Units), a mayor
  valor más amarga la cerveza
\item
  \textbf{Color:} A mayor valor más oscura la cerveza.
\end{itemize}

Determinar si las cervezas Porter y Stouts son distintas en grado
alcohólico, amargor y/o color.

\section{Supuestos de la prueba de t y
alternativas}\label{supuestos-de-la-prueba-de-t-y-alternativas}

Los supuestos de la t de student son las siguientes
\citep{boneau1960effects}

\begin{itemize}
\tightlist
\item
  Independencia de las observaciones
\item
  Distribución normal de los datos en cada grupo
\item
  Homogeneidad de varianza
\end{itemize}

\subsection{Prueba de una muestra}\label{prueba-de-una-muestra}

Como siempre la independencia de las muestras es algo que solo puede
determinarse en base a el diseño del muestreo, y por otro lado, al haber
solo una muestra, la homogeneidad de varianza no es un problema, en este
caso solo podemos ver si la distribución es normal. Volviendo a nuestro
ejemplo de una muestra, con la base de datos \texttt{faithfull}, veamos
en base a un histograma (figura \ref{fig:Hist}), qqplot (figura
\ref{fig:QQ}) y test de shapiro, si los datos son normales o no:

\begin{Shaded}
\begin{Highlighting}[]
\KeywordTok{hist}\NormalTok{(faithful}\OperatorTok{$}\NormalTok{waiting, }\DataTypeTok{xlab =} \StringTok{"Minutos de espera entre erupciones"}\NormalTok{)}
\end{Highlighting}
\end{Shaded}

\begin{figure}
\centering
\includegraphics{AyduantiaStats_files/figure-latex/Hist-1.pdf}
\caption{\label{fig:Hist}Histograma de los minutos de espera de el géiser
Old Fiathful}
\end{figure}

\begin{Shaded}
\begin{Highlighting}[]
\KeywordTok{qqnorm}\NormalTok{(faithful}\OperatorTok{$}\NormalTok{waiting)}
\end{Highlighting}
\end{Shaded}

\begin{figure}
\centering
\includegraphics{AyduantiaStats_files/figure-latex/QQ-1.pdf}
\caption{\label{fig:QQ}QQplot de los minutos de espera de el géiser Old
Fiathful}
\end{figure}

\begin{Shaded}
\begin{Highlighting}[]
\KeywordTok{shapiro.test}\NormalTok{(faithful}\OperatorTok{$}\NormalTok{waiting)}
\end{Highlighting}
\end{Shaded}

\begin{verbatim}
## 
##  Shapiro-Wilk normality test
## 
## data:  faithful$waiting
## W = 0.92215, p-value = 1.015e-10
\end{verbatim}

Como vemos en la figura 2, los datos no se ven normales, incluso se ven
bimodales, lo cual significa que tiene 2 picos, en este caso uno al
rededor de los 52 minutos y otro al rededor de los 85 minutos de espera
(recordemos que la función \texttt{hist}, automáticamente usa el
algoritmo de \citet{sturges1926choice}, para determinar como dividir los
datos y obtener el mejor histograma). Nuestras sospechas de no
normalidad son confirmadas al ver el qqplot, que no sigue para nada la
diagonal, y es reafirmado por el test de shapiro, cuyo valor mucho menor
a 0.05, nos dice que la distribución no es normal. Dado esto, debemos
apelar a un test de distribución libre como el de \emph{Mann-Whitney},
la cual se realiza con la función \texttt{wilcox.test}, de la misma
forma que es utilizada la función \texttt{t.test}, por lo tanto para
nuestro ejemplo usamos:

\begin{Shaded}
\begin{Highlighting}[]
\KeywordTok{data}\NormalTok{(}\StringTok{"faithful"}\NormalTok{)}
\KeywordTok{wilcox.test}\NormalTok{(}\DataTypeTok{x =}\NormalTok{ faithful}\OperatorTok{$}\NormalTok{waiting, }\DataTypeTok{mu =} \DecValTok{60}\NormalTok{, }\DataTypeTok{alternative =} \StringTok{"two.sided"}\NormalTok{)}
\end{Highlighting}
\end{Shaded}

\begin{verbatim}
## 
##  Wilcoxon signed rank test with continuity correction
## 
## data:  faithful$waiting
## V = 31048, p-value < 2.2e-16
## alternative hypothesis: true location is not equal to 60
\end{verbatim}

Que en este caso nos lleva a la misma conclusión que nuestro ejemplo
anterior.

\subsection{Prueba de dos muestras}\label{prueba-de-dos-muestras}

Para una prueba de dos muestras, podemos testear tanto la homogeneidad
de varianza como la normalidad, para ver las dos cosas al mismo tiempo
podemos usar un gráfico de violín (figura \ref{fig:Violin}). En este
caso, las distribuciones no se ven muy diferentes a la normalidad, pero
las varianzas se ven un tanto distintas, podemos seguir explorando esto
visualmente usando la función \texttt{hist} previamente generando dos
data frames, uno para autos automático y otro para manuales.

\begin{Shaded}
\begin{Highlighting}[]
\KeywordTok{data}\NormalTok{(}\StringTok{"mtcars"}\NormalTok{)}
\NormalTok{mt <-}\StringTok{ }\NormalTok{mtcars}
\NormalTok{mt}\OperatorTok{$}\NormalTok{am <-}\StringTok{ }\KeywordTok{ifelse}\NormalTok{(mtcars}\OperatorTok{$}\NormalTok{am  }\OperatorTok{==}\StringTok{ }\DecValTok{0}\NormalTok{, }\StringTok{"automatico"}\NormalTok{, }\StringTok{"manual"}\NormalTok{)}
\NormalTok{mt <-}\StringTok{ }\KeywordTok{as.data.frame}\NormalTok{(mt)}
\end{Highlighting}
\end{Shaded}

\begin{Shaded}
\begin{Highlighting}[]
\KeywordTok{ggplot}\NormalTok{(mt, }\KeywordTok{aes}\NormalTok{(}\DataTypeTok{x =}\NormalTok{ am, }\DataTypeTok{y =}\NormalTok{ mpg)) }\OperatorTok{+}\StringTok{ }\KeywordTok{geom_violin}\NormalTok{()}
\end{Highlighting}
\end{Shaded}

\begin{figure}
\centering
\includegraphics{AyduantiaStats_files/figure-latex/Violin-1.pdf}
\caption{\label{fig:Violin}Comparación de distribuciones y varianzas de los
vehiculos automáticos}
\end{figure}

En este caso, las distribuciones no se ven muy diferentes a la
normalidad, pero las varianzas se ven un tanto distintas, podemos seguir
explorando esto separando los datos en vehículos automáticos y manuales
para hacer histogramas, en este caso es importante que los ejes sean
iguales, para eso en el histograma usaremos los parámetros ylim y xlim.

\begin{Shaded}
\begin{Highlighting}[]
\KeywordTok{hist}\NormalTok{(manuales}\OperatorTok{$}\NormalTok{mpg, }\DataTypeTok{xlim =} \KeywordTok{c}\NormalTok{(}\DecValTok{10}\NormalTok{,}\DecValTok{35}\NormalTok{), }\DataTypeTok{ylim =} \KeywordTok{c}\NormalTok{(}\DecValTok{0}\NormalTok{,}\DecValTok{5}\NormalTok{))}
\end{Highlighting}
\end{Shaded}

\begin{figure}
\centering
\includegraphics{AyduantiaStats_files/figure-latex/Manual-1.pdf}
\caption{\label{fig:Manual}Histograma de vehiculos manuales}
\end{figure}

\begin{Shaded}
\begin{Highlighting}[]
\KeywordTok{hist}\NormalTok{(autos}\OperatorTok{$}\NormalTok{mpg, }\DataTypeTok{xlim =} \KeywordTok{c}\NormalTok{(}\DecValTok{10}\NormalTok{,}\DecValTok{35}\NormalTok{), }\DataTypeTok{ylim =} \KeywordTok{c}\NormalTok{(}\DecValTok{0}\NormalTok{,}\DecValTok{5}\NormalTok{))}
\end{Highlighting}
\end{Shaded}

\begin{figure}
\centering
\includegraphics{AyduantiaStats_files/figure-latex/Auto-1.pdf}
\caption{\label{fig:Auto}Histograma de vehiculos automáticos}
\end{figure}

Como vemos, los vehículos manuales no parecen tener distribución normal
como se ve en la figura \ref{fig:Manual}, esto podemos comprobarlo con
el qqlot de los mismos datos (figura \ref{fig:qqManual})

\begin{Shaded}
\begin{Highlighting}[]
\KeywordTok{qqnorm}\NormalTok{(manuales}\OperatorTok{$}\NormalTok{mpg)}
\end{Highlighting}
\end{Shaded}

\begin{figure}
\centering
\includegraphics{AyduantiaStats_files/figure-latex/qqManual-1.pdf}
\caption{\label{fig:qqManual}QQplot de eficiencia de vehiculos con cambios
manuales}
\end{figure}

\subsubsection{Ejercicio 3}\label{ejercicio-3}

Como siempre la independencia de las muestras es algo que solo puede
determinarse en base a el diseño del muestreo, y por otro lado, al haber
solo una muestra, la homogeneidad de varianza no es un problema, en este
caso solo podemos ver si la distribución es normal. Volviendo a nuestro
ejercicio de una muestra, con la base de datos \texttt{airquality},
evalúe basado en histograma, qqplot y test de shapiro si se debe
revaluar la hipótesis para los meses de julio y agosto

Para una prueba de dos muestras, podemos testear tanto la homogeneidad
de varianza como la normalidad, para ver las dos cosas al mismo tiempo
podemos usar un gráfico de violín \texttt{geom\_violin} en
\emph{ggplot2}, lo cual puede seguir siendo explorando esto visualmente
usando la función \texttt{hist} generando dos data frames, uno por cada
clase de datos.

Evalúe si es necesario revaluar la hipótesis de que el amargor es
distinto entre ambos estilos de cerveza

\section{Bibliografía}\label{bibliografia}

\bibliography{book.bib}


\end{document}
