\documentclass[]{book}
\usepackage{lmodern}
\usepackage{amssymb,amsmath}
\usepackage{ifxetex,ifluatex}
\usepackage{fixltx2e} % provides \textsubscript
\ifnum 0\ifxetex 1\fi\ifluatex 1\fi=0 % if pdftex
  \usepackage[T1]{fontenc}
  \usepackage[utf8]{inputenc}
\else % if luatex or xelatex
  \ifxetex
    \usepackage{mathspec}
  \else
    \usepackage{fontspec}
  \fi
  \defaultfontfeatures{Ligatures=TeX,Scale=MatchLowercase}
\fi
% use upquote if available, for straight quotes in verbatim environments
\IfFileExists{upquote.sty}{\usepackage{upquote}}{}
% use microtype if available
\IfFileExists{microtype.sty}{%
\usepackage{microtype}
\UseMicrotypeSet[protrusion]{basicmath} % disable protrusion for tt fonts
}{}
\usepackage[margin=1in]{geometry}
\usepackage{hyperref}
\hypersetup{unicode=true,
            pdftitle={BIO 4022. Manipulación de datos e investigación reproducible en R},
            pdfauthor={Derek Corcoran},
            pdfborder={0 0 0},
            breaklinks=true}
\urlstyle{same}  % don't use monospace font for urls
\usepackage{natbib}
\bibliographystyle{apalike}
\usepackage{color}
\usepackage{fancyvrb}
\newcommand{\VerbBar}{|}
\newcommand{\VERB}{\Verb[commandchars=\\\{\}]}
\DefineVerbatimEnvironment{Highlighting}{Verbatim}{commandchars=\\\{\}}
% Add ',fontsize=\small' for more characters per line
\usepackage{framed}
\definecolor{shadecolor}{RGB}{248,248,248}
\newenvironment{Shaded}{\begin{snugshade}}{\end{snugshade}}
\newcommand{\AlertTok}[1]{\textcolor[rgb]{0.94,0.16,0.16}{#1}}
\newcommand{\AnnotationTok}[1]{\textcolor[rgb]{0.56,0.35,0.01}{\textbf{\textit{#1}}}}
\newcommand{\AttributeTok}[1]{\textcolor[rgb]{0.77,0.63,0.00}{#1}}
\newcommand{\BaseNTok}[1]{\textcolor[rgb]{0.00,0.00,0.81}{#1}}
\newcommand{\BuiltInTok}[1]{#1}
\newcommand{\CharTok}[1]{\textcolor[rgb]{0.31,0.60,0.02}{#1}}
\newcommand{\CommentTok}[1]{\textcolor[rgb]{0.56,0.35,0.01}{\textit{#1}}}
\newcommand{\CommentVarTok}[1]{\textcolor[rgb]{0.56,0.35,0.01}{\textbf{\textit{#1}}}}
\newcommand{\ConstantTok}[1]{\textcolor[rgb]{0.00,0.00,0.00}{#1}}
\newcommand{\ControlFlowTok}[1]{\textcolor[rgb]{0.13,0.29,0.53}{\textbf{#1}}}
\newcommand{\DataTypeTok}[1]{\textcolor[rgb]{0.13,0.29,0.53}{#1}}
\newcommand{\DecValTok}[1]{\textcolor[rgb]{0.00,0.00,0.81}{#1}}
\newcommand{\DocumentationTok}[1]{\textcolor[rgb]{0.56,0.35,0.01}{\textbf{\textit{#1}}}}
\newcommand{\ErrorTok}[1]{\textcolor[rgb]{0.64,0.00,0.00}{\textbf{#1}}}
\newcommand{\ExtensionTok}[1]{#1}
\newcommand{\FloatTok}[1]{\textcolor[rgb]{0.00,0.00,0.81}{#1}}
\newcommand{\FunctionTok}[1]{\textcolor[rgb]{0.00,0.00,0.00}{#1}}
\newcommand{\ImportTok}[1]{#1}
\newcommand{\InformationTok}[1]{\textcolor[rgb]{0.56,0.35,0.01}{\textbf{\textit{#1}}}}
\newcommand{\KeywordTok}[1]{\textcolor[rgb]{0.13,0.29,0.53}{\textbf{#1}}}
\newcommand{\NormalTok}[1]{#1}
\newcommand{\OperatorTok}[1]{\textcolor[rgb]{0.81,0.36,0.00}{\textbf{#1}}}
\newcommand{\OtherTok}[1]{\textcolor[rgb]{0.56,0.35,0.01}{#1}}
\newcommand{\PreprocessorTok}[1]{\textcolor[rgb]{0.56,0.35,0.01}{\textit{#1}}}
\newcommand{\RegionMarkerTok}[1]{#1}
\newcommand{\SpecialCharTok}[1]{\textcolor[rgb]{0.00,0.00,0.00}{#1}}
\newcommand{\SpecialStringTok}[1]{\textcolor[rgb]{0.31,0.60,0.02}{#1}}
\newcommand{\StringTok}[1]{\textcolor[rgb]{0.31,0.60,0.02}{#1}}
\newcommand{\VariableTok}[1]{\textcolor[rgb]{0.00,0.00,0.00}{#1}}
\newcommand{\VerbatimStringTok}[1]{\textcolor[rgb]{0.31,0.60,0.02}{#1}}
\newcommand{\WarningTok}[1]{\textcolor[rgb]{0.56,0.35,0.01}{\textbf{\textit{#1}}}}
\usepackage{longtable,booktabs}
\usepackage{graphicx,grffile}
\makeatletter
\def\maxwidth{\ifdim\Gin@nat@width>\linewidth\linewidth\else\Gin@nat@width\fi}
\def\maxheight{\ifdim\Gin@nat@height>\textheight\textheight\else\Gin@nat@height\fi}
\makeatother
% Scale images if necessary, so that they will not overflow the page
% margins by default, and it is still possible to overwrite the defaults
% using explicit options in \includegraphics[width, height, ...]{}
\setkeys{Gin}{width=\maxwidth,height=\maxheight,keepaspectratio}
\IfFileExists{parskip.sty}{%
\usepackage{parskip}
}{% else
\setlength{\parindent}{0pt}
\setlength{\parskip}{6pt plus 2pt minus 1pt}
}
\setlength{\emergencystretch}{3em}  % prevent overfull lines
\providecommand{\tightlist}{%
  \setlength{\itemsep}{0pt}\setlength{\parskip}{0pt}}
\setcounter{secnumdepth}{5}
% Redefines (sub)paragraphs to behave more like sections
\ifx\paragraph\undefined\else
\let\oldparagraph\paragraph
\renewcommand{\paragraph}[1]{\oldparagraph{#1}\mbox{}}
\fi
\ifx\subparagraph\undefined\else
\let\oldsubparagraph\subparagraph
\renewcommand{\subparagraph}[1]{\oldsubparagraph{#1}\mbox{}}
\fi

%%% Use protect on footnotes to avoid problems with footnotes in titles
\let\rmarkdownfootnote\footnote%
\def\footnote{\protect\rmarkdownfootnote}

%%% Change title format to be more compact
\usepackage{titling}

% Create subtitle command for use in maketitle
\newcommand{\subtitle}[1]{
  \posttitle{
    \begin{center}\large#1\end{center}
    }
}

\setlength{\droptitle}{-2em}

  \title{BIO 4022. Manipulación de datos e investigación reproducible en R}
    \pretitle{\vspace{\droptitle}\centering\huge}
  \posttitle{\par}
    \author{Derek Corcoran}
    \preauthor{\centering\large\emph}
  \postauthor{\par}
      \predate{\centering\large\emph}
  \postdate{\par}
    \date{2018-11-28}

\usepackage{booktabs}

\begin{document}
\maketitle

{
\setcounter{tocdepth}{1}
\tableofcontents
}
\hypertarget{requerimientos}{%
\chapter*{Requerimientos}\label{requerimientos}}
\addcontentsline{toc}{chapter}{Requerimientos}

Para comenzar el trabajo se necesita la última versión de R y RStudio
\citep{R-base}.También se requiere de los paquetes \emph{pacman},
\emph{rmarkdown}, \emph{tidyverse} y \emph{tinytex}. Si no se ha usado R
o RStudio anteriormente, el siguiente video muestra cómo instalar ambos
programas y los paquetes necesarios para este curso en el siguiente
\href{https://youtu.be/RtkCAKXsVbw}{link}.

El código para la instalación de esos paquetes es el siguiente:

\begin{Shaded}
\begin{Highlighting}[]
\KeywordTok{install.packages}\NormalTok{(}\StringTok{"pacman"}\NormalTok{, }\StringTok{"rmarkdown"}\NormalTok{, }\StringTok{"tidyverse"}\NormalTok{, }\StringTok{"tinytex"}\NormalTok{)}
\end{Highlighting}
\end{Shaded}

En caso de necesitar ayuda para la instalación, contactarse con el
instructor del curso.

\hypertarget{antes-de-comenzar}{%
\section{Antes de comenzar}\label{antes-de-comenzar}}

Si nunca se ha trabajado con \texttt{R} antes de este curso, una buena
herramienta es provista por el paquete
\href{http://swirlstats.com/students.html}{Swirl} \citep{Kross2017}.
Para comenzar la práctica, realizar los primeros 7 modulos del programa
\emph{R Programming: The basics of programming in R} que incluye:

\begin{itemize}
\tightlist
\item
  Basic Building Blocks
\item
  Workspace and Files
\item
  Sequences of Numbers
\item
  Vectors
\item
  Missing Values
\item
  Subsetting Vectors
\item
  Matrices and Data Frames
\end{itemize}

El siguiente link muestra un video explicativo de cómo usar el paquete
swirl \href{https://youtu.be/w6L7Ye18yPE}{Video}

\hypertarget{descripcion-del-curso}{%
\section{Descripción del curso}\label{descripcion-del-curso}}

Este curso está enfocado en entregar principios básicos de investigación
reproducible en R, con énfasis en la recopilación y/o lectura de datos
de forma reproducible y automatizada. Para esto se trabajará con bases
de datos complejas, las cuales deberán ser transformadas y organizadas
para optimizar su análisis. Se generarán documentos reproducibles
integrando en un documento: código, bibliografía, exploración y análisis
de datos. Se culminará el curso con la generación de un manuscrito, una
presentación y/o un documento interactivo reproducible.

\hypertarget{objetivos-del-curso}{%
\section{Objetivos del curso}\label{objetivos-del-curso}}

\begin{enumerate}
\def\labelenumi{\arabic{enumi}.}
\item
  Conocer y entender el concepto de investigación reproducible como una
  forma y filosofía de trabajo que permite que las investigaciones sean
  más ordenadas y replicables, desde la toma de datos hasta la escritura
  de resultados.
\item
  Conocer y aplicar el concepto de pipeline, el cual permite generar una
  modularidad desde la toma de datos hasta la escritura de resultados,
  donde la corrección independiente de un paso tiene un efecto cascada
  sobre el resultado final.
\item
  Aprender buenas prácticas de recolección y estandarización de bases de
  datos, con la finalidad de optimizar el análisis de datos y la
  revisión de éstas por pares.
\item
  Realizar análisis críticos de la naturaleza de los datos al realizar
  análisis exploratorios, que permitirán determinar la mejor forma de
  comprobar hipótesis asociadas a estas bases de datos.
\end{enumerate}

\hypertarget{contenidos}{%
\section{Contenidos}\label{contenidos}}

\begin{itemize}
\item
  Capítulo \ref{tidydata} \emph{Tidy Data}: En este capítulo se
  aprenderá a cómo optimizar una de base de datos, sobre la limpieza y
  transformación de bases de datos, qué es una base de datos \emph{tidy}
  y cómo manipular estas bases de datos con el paquete \emph{dplyr}
  \citep{R-dplyr}.
\item
  Capítulo \ref{reproducible} \emph{Investigación reproducible}: En este
  capítulo se trabajará en la confección de un documento que combine
  códigos de \texttt{R} y texto para generar documentos reproducibles
  utilizando el paquete \emph{rmarkdown} \citep{Allaire2018}. Además, se
  verá cómo al usar RStudio se pueden guardar los proyectos en un
  repositorio de github.
\item
  Capítulo \ref{tidyverso} \emph{El tidyverso} y el concepto de
  pipeline:En este capítulo se aprenderá sobre la limpieza de datos
  complejos.
\item
  Capítulo \ref{visualizacion} \emph{Visualización de datos} visualizar
  datos vs.~visualizar modelos. Insertar gráficos con leyenda en un
  documento Rmd
\item
  Capítulo \ref{modelos} \emph{Modelos en R} Aprender a generar modelos
  en R, desde ANOVA a GLM.
\item
  Capítulo \ref{loops} \emph{Loops}. Generación de funciones propias en
  R y loops
\end{itemize}

\begin{enumerate}
\def\labelenumi{\arabic{enumi}.}
\setcounter{enumi}{5}
\item
  Escritura de manuscritos en R, transformación de documentos Rmd en un
  manuscrito
\item
  Presentaciones en R y generar documentos interactivos. Transformación
  de datos en una presentación o en una Shiny app. Realizar una
  presentación o aplicación en R.
\end{enumerate}

\hypertarget{metodologia}{%
\section{Metodología}\label{metodologia}}

Todas las clases estarán divididas en dos partes: I. Clases expositivas
de principios y herramientas, donde se presentarán los principios de
investigación reproducible y tidy data, junto con las herramientas
actuales más utilizadas, y II. Clases prácticas donde cada estudiante
trabajará con datos propios para desarrollar un documento reproducible.
Los estudiantes que no cuenten con datos propios podrán acceder a sets
de datos para su trabajo o podrán simularlos, dependiendo del caso.

Además, se deberán generar informes y presentaciones siguiendo los
principios de investigación reproducible, en base al trabajo con sus
datos. Se realizará un informe final, en el cual se espera un trabajo
que compile los conociminetos adquiridos durante el curso.

\hypertarget{evaluacion}{%
\section{Evaluación}\label{evaluacion}}

\begin{itemize}
\tightlist
\item
  Evaluación 1: Informe exploratorio de base de datos 25\%
\item
  Evaluación 2: Presentación 25\%
\item
  Evaluación 3: Informe final 50\%
\end{itemize}

\hypertarget{libros-de-consulta}{%
\section{Libros de consulta}\label{libros-de-consulta}}

Los principios de este curso están explicados en los siguientes libros
gratuitos.

\begin{itemize}
\tightlist
\item
  Gandrud, Christopher. Reproducible Research with R and R Studio. CRC
  Press, 2013. Available for free in the following
  \href{https://englianhu.files.wordpress.com/2016/01/reproducible-research-with-r-and-studio-2nd-edition.pdf}{link}
\item
  Stodden, Victoria, Friedrich Leisch, and Roger D. Peng, eds.
  Implementing reproducible research. CRC Press, 2014. Available for
  free in the following
  \href{http://web.stanford.edu/~vcs/papers/ijclp-STODDEN-2009.pdf}{link}
\end{itemize}

\hypertarget{bibliografia}{%
\section{Bibliografía}\label{bibliografia}}

\hypertarget{tidydata}{%
\chapter{Tidy Data y manipulación de datos}\label{tidydata}}

\hypertarget{paquetes-necesarios-para-este-capitulo}{%
\section{Paquetes necesarios para este
capítulo}\label{paquetes-necesarios-para-este-capitulo}}

Para este capitulo necesitas tener instalado el paquete \emph{tidyverse}

En este capítulo se explicará qué es una base de datos \emph{tidy}
\citep{wickham2014tidy} y se aprenderá a usar funciones del paquete
\emph{dplyr} \citep{R-dplyr} para manipular datos.

Dado que este libro es un apoyo para el curso BIO4022, esta clase del
curso puede también ser seguida en este
\href{https://derek-corcoran-barrios.github.io/Clase1/Clase1TidyData}{link}.
El video de la clase se encuentra disponible en este
\href{https://youtu.be/vQKWd02HB90}{link}.

\hypertarget{tidy-data}{%
\section{Tidy data}\label{tidy-data}}

Una base de datos tidy es una base de datos en la cuál (modificado de
\citep{leek2015elements}):

\begin{itemize}
\tightlist
\item
  Cada vararible que se medida debe estar en una columna.
\item
  Cada observación distinta de esa variable debe estar en una fila
  diferente.
\end{itemize}

En general, la forma en que representaríamos una base de datos
\emph{tidy} en \texttt{R} es usando un \emph{data frame}.

\hypertarget{dplyr}{%
\section{dplyr}\label{dplyr}}

El paquete \emph{dplyr} es definido por sus autores como una gramática
para la manipulación de datos. De este modo sus funciones son conocidas
como verbos. Un resumen útil de muchas de estas funciones se encuentra
en este
\href{https://www.rstudio.com/wp-content/uploads/2015/02/data-wrangling-cheatsheet.pdf}{link}.

Este paquete tiene un gran número de verbos y sería difícil ver todos en
una clase, en este capítulo nos enfocaremos en sus funciones más
utilizadas, las cuales son:

\begin{itemize}
\tightlist
\item
  \emph{group\_by} (agrupa datos)
\item
  \emph{summarize} (resume datos agrupados)
\item
  \emph{mutate} (genera variables nuevas)
\item
  \emph{\%\textgreater{}\%} (pipeline)
\item
  \emph{filter} (encuentra filas con ciertas condiciones)
\item
  \emph{select} junto a \emph{starts\_with}, \emph{ends\_with} o
  \emph{contains}
\end{itemize}

\hypertarget{summarize}{%
\subsection{summarize}\label{summarize}}

La función \texttt{summarize} toma los datos de un data frame y los
resume. Para usar esta función, el primer argumento que tomaríamos sería
un data frame, se continúa del nombre que queremos darle a una variable
resumen, seguida del signo = y luego la fórmula a aplicar a una o mas
columnas. COmo un ejemplo se utilizará la base de datos \texttt{iris}
\citep{anderson1935irises} que viene en \texttt{R} y de las cual podemos
ver parte de sus datos en la tabla \ref{tab:iris}

\label{tab:iris}una tabla con 10 filas de la base de datos iris.

Sepal.Length

Sepal.Width

Petal.Length

Petal.Width

Species

5.8

4.0

1.2

0.2

setosa

4.7

3.2

1.6

0.2

setosa

5.1

3.8

1.9

0.4

setosa

5.2

2.7

3.9

1.4

versicolor

6.4

2.9

4.3

1.3

versicolor

5.5

2.5

4.0

1.3

versicolor

6.5

3.0

5.8

2.2

virginica

6.0

2.2

5.0

1.5

virginica

6.1

2.6

5.6

1.4

virginica

5.9

3.0

5.1

1.8

virginica

Si quisieramos resumir esa tabla y generar un par de variables que
fueran la media y la desviación estándar del largo del pétalo, lo
haríamos con el siguiente código:

\begin{Shaded}
\begin{Highlighting}[]
\KeywordTok{library}\NormalTok{(tidyverse)}
\NormalTok{Summary.Petal <-}\StringTok{ }\KeywordTok{summarize}\NormalTok{(iris, }\DataTypeTok{Mean.Petal.Length =} \KeywordTok{mean}\NormalTok{(Petal.Length), }
    \DataTypeTok{SD.Petal.Length =} \KeywordTok{sd}\NormalTok{(Petal.Length))}
\end{Highlighting}
\end{Shaded}

El resultado se puedde ver en la tabla \ref{tab:SummaryPetaltab}, en el
cuál se obtienen los promedios y desviaciones estándar de los largos de
los pétalos. Es importante notar que al usar summarize, todas las otras
variables desapareceran de la tabla.

\label{tab:SummaryPetaltab}Resumen del promedio y desviación estándar del
largo de pétalo de las flores del generi Iris.

Mean.Petal.Length

SD.Petal.Length

3.758

1.765298

\hypertarget{group_by}{%
\subsection{group\_by}\label{group_by}}

La función \texttt{group\_by} por si sola no genera cambios visibles en
las bases de datos. Sin embargo, al ser utilizada en conjunto con
\texttt{summarize} permite resumir una variable agrupada (usualmente)
basada en una o más variables categóricas.

Se puede ver que para el ejemplo con el caso de las plantas del género
\emph{Iris}, el resumen que se obtiene en el caso de la tabla
\ref{tab:SummaryPetaltab} no es tan útil considerando que tenemos tres
especies presentes. Si se quiere ver el promedio del largo del pétalo
por especie, se debe ocupar la función \texttt{group\_by} de la
siguiente forma:

\begin{Shaded}
\begin{Highlighting}[]
\NormalTok{BySpecies <-}\StringTok{ }\KeywordTok{group_by}\NormalTok{(iris, Species)}
\NormalTok{Summary.Byspecies <-}\StringTok{ }\KeywordTok{summarize}\NormalTok{(BySpecies, }\DataTypeTok{Mean.Petal.Length =} \KeywordTok{mean}\NormalTok{(Petal.Length), }
    \DataTypeTok{SD.Petal.Length =} \KeywordTok{sd}\NormalTok{(Petal.Length))}
\end{Highlighting}
\end{Shaded}

Esto dá como resultado la tabla \ref{tab:SummaryBySpecies}, con la cuál
se puede ver que \emph{Iris setosa} tiene pétalos mucho más cortos que
las otras dos especies del mismo género.

\label{tab:SummaryBySpecies}Resumen del promedio y desviación estándar del
largo de pétalo de las flores del generi Iris.

Species

Mean.Petal.Length

SD.Petal.Length

setosa

1.462

0.1736640

versicolor

4.260

0.4699110

virginica

5.552

0.5518947

\hypertarget{group_by-en-mas-de-una-variable}{%
\subsubsection{group\_by en más de una
variable}\label{group_by-en-mas-de-una-variable}}

Se puede usar la función \texttt{group\_by} en más de una variable, y
esto generaría un resumen anidado. Como ejemplo se usará la base de
datos \texttt{mtcars} presente en R \citep{henderson1981building}. Esta
base de datos presenta una variable llamada \emph{mpg} (miles per
gallon) y una medida de eficiencia de combustible. Se resumirá la
información en base a la variable \emph{am} (que se refiere al tipo de
transmisión, donde 0 es automático y 1 es manual) y al número de
cilindros del motor. Para eso se utilizará el siguiente código:

\begin{Shaded}
\begin{Highlighting}[]
\NormalTok{Grouped <-}\StringTok{ }\KeywordTok{group_by}\NormalTok{(mtcars, cyl, am)}
\NormalTok{Eficiencia <-}\StringTok{ }\KeywordTok{summarize}\NormalTok{(Grouped, }\DataTypeTok{Eficiencia =} \KeywordTok{mean}\NormalTok{(mpg))}
\end{Highlighting}
\end{Shaded}

Como puede verse en la tabla \ref{tab:Eficienciatab}, en todos los casos
los autos con cambios manuales tienen mejor eficiencia de combustible.
Se podría probar el cambiar el orden de las variables con las cuales
agrupar y observar los distintos resultados que se pueden obtener.

\label{tab:Eficienciatab}Millas por galón promedio en vehiculos automáticos
(am = 0) y manuales (am = 1), con los distintos tipos de cilindros

cyl

am

Eficiencia

4

0

22.90000

4

1

28.07500

6

0

19.12500

6

1

20.56667

8

0

15.05000

8

1

15.40000

\hypertarget{mutate}{%
\subsection{mutate}\label{mutate}}

Esta función tiene como objetivo crear variables nuevas basadas en otras
variables. Es muy facil de usar, como argumento se usa el nombre de la
variable nueva que se quiere crear y se realiza una operación con
variables que ya estan ahí. Por ejemplo, si se continúa el trabajo con
la base de datos \emph{Iris}, al crear una nueva variable que sea la
razón entre el largo del pétalo y el del sépalo, resulta lo siguiente:

\begin{Shaded}
\begin{Highlighting}[]
\NormalTok{DF <-}\StringTok{ }\KeywordTok{mutate}\NormalTok{(iris, }\DataTypeTok{Petal.Sepal.Ratio =}\NormalTok{ Petal.Length}\OperatorTok{/}\NormalTok{Sepal.Length)}
\end{Highlighting}
\end{Shaded}

El resultado de esta operación es la tabla \ref{tab:Mutate}. Siempre la
variable que se acaba de crear aparecerá al final del data frame.

\label{tab:Mutate}Tabla con diez de las observaciones de la nueva base de
datos con la variable nueva creada con mutate

Sepal.Length

Sepal.Width

Petal.Length

Petal.Width

Species

Petal.Sepal.Ratio

5.8

4.0

1.2

0.2

setosa

0.21

4.7

3.2

1.6

0.2

setosa

0.34

5.1

3.8

1.9

0.4

setosa

0.37

5.2

2.7

3.9

1.4

versicolor

0.75

6.4

2.9

4.3

1.3

versicolor

0.67

5.5

2.5

4.0

1.3

versicolor

0.73

6.5

3.0

5.8

2.2

virginica

0.89

6.0

2.2

5.0

1.5

virginica

0.83

6.1

2.6

5.6

1.4

virginica

0.92

5.9

3.0

5.1

1.8

virginica

0.86

\hypertarget{pipeline}{%
\subsection{Pipeline (\%\textgreater{}\%)}\label{pipeline}}

El pipeline es un simbolo operatorio \texttt{\%\textgreater{}\%} que
sirve para realizar varias operaciones de forma secuencial sin recurrir
a parentesis anidados o a sobrescribir muúltiples bases de datos.

Para ver como funciona esto como un vector, supongamos que se tiene una
variable a la cual se quiere primero obtener su logaritmo, luego su raíz
cuadrada y finalmente su promedio con dos cifras significativas. Para
realizar esto se debe seguir lo siguiente:

\begin{Shaded}
\begin{Highlighting}[]
\NormalTok{x <-}\StringTok{ }\KeywordTok{c}\NormalTok{(}\DecValTok{1}\NormalTok{, }\DecValTok{4}\NormalTok{, }\DecValTok{6}\NormalTok{, }\DecValTok{8}\NormalTok{)}
\NormalTok{y <-}\StringTok{ }\KeywordTok{round}\NormalTok{(}\KeywordTok{mean}\NormalTok{(}\KeywordTok{sqrt}\NormalTok{(}\KeywordTok{log}\NormalTok{(x))), }\DecValTok{2}\NormalTok{)}
\end{Highlighting}
\end{Shaded}

Si se utiliza pipeline, el código sería mucho más ordenado. En ese caso,
se partiría por el objeto a procesar y luego cada una de las funciones
con sus argumentos si es necesario:

\begin{Shaded}
\begin{Highlighting}[]
\NormalTok{x <-}\StringTok{ }\KeywordTok{c}\NormalTok{(}\DecValTok{1}\NormalTok{, }\DecValTok{4}\NormalTok{, }\DecValTok{6}\NormalTok{, }\DecValTok{8}\NormalTok{)}
\NormalTok{y <-}\StringTok{ }\NormalTok{x }\OperatorTok\StringTok{ }\KeywordTok{log}\NormalTok{() }\OperatorTok\StringTok{ }\KeywordTok{sqrt}\NormalTok{() }\OperatorTok\StringTok{ }\KeywordTok{mean}\NormalTok{() }\OperatorTok\StringTok{ }\KeywordTok{round}\NormalTok{(}\DecValTok{2}\NormalTok{)}
\end{Highlighting}
\end{Shaded}

\begin{verbatim}
## [1] 0.99
\end{verbatim}

El código con pipeline es mucho más fácil de interpretar a primera vista
ya que se lee de izquierda a derecha y no de adentro hacia afuera. EL
uso de pipeli se hace aun más importante cuando se usa con un \emph{Data
frame}, como se ve en el siguiente ejemplo:

\hypertarget{el-pipeline-en-data-frames}{%
\subsubsection{El pipeline en data
frames}\label{el-pipeline-en-data-frames}}

POr ejemplo se quiere resumir la variable recien creada de la razón
entre el sépalo y el petalo. Para hacer esto, si se partiera desde la
base de datos original, tomaría varias líneas de código y la creación de
múltiples bases de datos intermedias

\begin{Shaded}
\begin{Highlighting}[]
\NormalTok{DF <-}\StringTok{ }\KeywordTok{mutate}\NormalTok{(iris, }\DataTypeTok{Petal.Sepal.Ratio =}\NormalTok{ Petal.Length}\OperatorTok{/}\NormalTok{Sepal.Length)}
\NormalTok{BySpecies <-}\StringTok{ }\KeywordTok{group_by}\NormalTok{(DF, Species)}
\NormalTok{Summary.Byspecies <-}\StringTok{ }\KeywordTok{summarize}\NormalTok{(BySpecies, }\DataTypeTok{MEAN =} \KeywordTok{mean}\NormalTok{(Petal.Sepal.Ratio), }
    \DataTypeTok{SD =} \KeywordTok{sd}\NormalTok{(Petal.Sepal.Ratio))}
\end{Highlighting}
\end{Shaded}

Otra opción es usar paréntesis anidados, lo que se traduce en el
siguiente código:

\begin{Shaded}
\begin{Highlighting}[]
\NormalTok{Summary.Byspecies <-}\StringTok{ }\KeywordTok{summarize}\NormalTok{(}\KeywordTok{group_by}\NormalTok{(}\KeywordTok{mutate}\NormalTok{(iris, }\DataTypeTok{Petal.Sepal.Ratio =}\NormalTok{ Petal.Length}\OperatorTok{/}\NormalTok{Sepal.Length), }
\NormalTok{    Species), }\DataTypeTok{MEAN =} \KeywordTok{mean}\NormalTok{(Petal.Sepal.Ratio), }\DataTypeTok{SD =} \KeywordTok{sd}\NormalTok{(Petal.Sepal.Ratio))}
\end{Highlighting}
\end{Shaded}

Esto se simplifica mucho más al usar el pipeline, lo cual permite partir
en un \emph{Data Frame} y luego usar el pipeline. Esto permite obtener
el mismo resultado que en las operaciones anteriores con el siguiente
código:

\begin{Shaded}
\begin{Highlighting}[]
\NormalTok{Summary.Byspecies <-}\StringTok{ }\NormalTok{iris }\OperatorTok\StringTok{ }\KeywordTok{mutate}\NormalTok{(}\DataTypeTok{Petal.Sepal.Ratio =}\NormalTok{ Petal.Length}\OperatorTok{/}\NormalTok{Sepal.Length) }\OperatorTok\StringTok{ }
\StringTok{    }\KeywordTok{group_by}\NormalTok{(Species) }\OperatorTok\StringTok{ }\KeywordTok{summarize}\NormalTok{(}\DataTypeTok{MEAN =} \KeywordTok{mean}\NormalTok{(Petal.Sepal.Ratio), }
    \DataTypeTok{SD =} \KeywordTok{sd}\NormalTok{(Petal.Sepal.Ratio))}
\end{Highlighting}
\end{Shaded}

Estos tres códigos son correctos (tabla \ref{tab:Pipe}), pero
definitivamente el uso del pipeline da el código más conciso y fácil de
interpretar sin pasos intermedios.

\label{tab:Pipe}Razón pétalo sépalo promedio para las tres especies de Iris

Species

MEAN

SD

setosa

0.2927557

0.0347958

versicolor

0.7177285

0.0536255

virginica

0.8437495

0.0438064

\hypertarget{filter}{%
\subsection{filter}\label{filter}}

Esta función permite seleccionar filas que cumplen con ciertas
condiciones, como tener un valor mayor a un umbral o pertenecer a cierta
clase Los símbolos más típicos a usar en este caso son los que se ven en
la tabla \ref{tab:Logicas}.

\label{tab:Logicas}Símbolos lógicos de R y su significado

simbolo

significado

simbolo\_cont

significado\_cont

\textgreater{}

Mayor que

!=

distinto a

\textless{}

Menor que

\%in\%

dentro del grupo

==

Igual a

is.na

es NA

\textgreater{}=

mayor o igual a

!is.na

no es NA

\textless{}=

menor o igual a

\textbar{} \&

o, y

Por ejemplo si se quiere estudiar las características florales de las
plantas del género \emph{Iris}, pero no tomar en cuenta a la especie
\emph{Iris versicolor} se deberá usar el siguiente código:

\begin{Shaded}
\begin{Highlighting}[]
\KeywordTok{data}\NormalTok{(}\StringTok{"iris"}\NormalTok{)}
\NormalTok{DF <-}\StringTok{ }\NormalTok{iris }\OperatorTok\StringTok{ }\KeywordTok{filter}\NormalTok{(Species }\OperatorTok{!=}\StringTok{ "versicolor"}\NormalTok{) }\OperatorTok\StringTok{ }\KeywordTok{group_by}\NormalTok{(Species) }\OperatorTok\StringTok{ }
\StringTok{    }\KeywordTok{summarise_all}\NormalTok{(mean)}
\end{Highlighting}
\end{Shaded}

De esta forma se obtiene como resultado la tabla
\ref{tab:MenosVersicolor}. En este caso se introduce la función
\texttt{summarize\_all} de \texttt{summarize}, la cual aplica la función
que se le da como argumento a todas las variables de la base de datos.

\label{tab:MenosVersicolor}Resumen de la media de todas las características
florales de las especies Iris setosa e Iris virginica

Species

Sepal.Length

Sepal.Width

Petal.Length

Petal.Width

setosa

5.006

3.428

1.462

0.246

virginica

6.588

2.974

5.552

2.026

Por otro lado si se quiere estudiar cuántas plantas de cada especie
tienen un largo de pétalo mayor a 4 y un largo de sépalo mayor a 5 se
deberá usar el siguiente código:

\begin{Shaded}
\begin{Highlighting}[]
\NormalTok{DF <-}\StringTok{ }\NormalTok{iris }\OperatorTok\StringTok{ }\KeywordTok{filter}\NormalTok{(Petal.Length }\OperatorTok{>=}\StringTok{ }\DecValTok{4} \OperatorTok{&}\StringTok{ }\NormalTok{Sepal.Length }\OperatorTok{>=}\StringTok{ }\DecValTok{5}\NormalTok{) }\OperatorTok\StringTok{ }
\StringTok{    }\KeywordTok{group_by}\NormalTok{(Species) }\OperatorTok\StringTok{ }\KeywordTok{summarise}\NormalTok{(}\DataTypeTok{N =} \KeywordTok{n}\NormalTok{())}
\end{Highlighting}
\end{Shaded}

En la tabla tabla \ref{tab:Numero} se ve que con este filtro desaparecen
de la base de datos todas las plantas de \emph{Iris setosa} y que todas
menos una planta de \emph{Iris virginica} tienen ambas características.

\label{tab:Numero}Número de plantas de cada especie con un largo de pétalo
mayor a 4 y un largo de sépalo mayor a 5 centímetros

Species

N

versicolor

39

virginica

49

\hypertarget{select}{%
\subsection{select}\label{select}}

Esta función permite seleccionar las variables a utilizar dado que en
muchos casos nos encontraremos con bases de datos con demasiadas
variables y por lo tanto, se querrá reducirlas para solo trabajar en una
tabla con las variables necesarias.

Con select hay varias formas de trabajar, por un lado se puede escribir
las variables que se utilizarán, o restar las que no. En ese sentido
estos cuatro códigos dan exactamente el mismo resultado. Esto se puede
ver en la tabla \ref{tab:Selected}

\begin{Shaded}
\begin{Highlighting}[]
\NormalTok{iris }\OperatorTok\StringTok{ }\KeywordTok{group_by}\NormalTok{(Species) }\OperatorTok\StringTok{ }\KeywordTok{select}\NormalTok{(Petal.Length, Petal.Width) }\OperatorTok\StringTok{ }
\StringTok{    }\KeywordTok{summarize_all}\NormalTok{(mean)}
\end{Highlighting}
\end{Shaded}

\begin{Shaded}
\begin{Highlighting}[]
\NormalTok{iris }\OperatorTok\StringTok{ }\KeywordTok{group_by}\NormalTok{(Species) }\OperatorTok\StringTok{ }\KeywordTok{select}\NormalTok{(}\OperatorTok{-}\NormalTok{Sepal.Length, }\OperatorTok{-}\NormalTok{Sepal.Width) }\OperatorTok\StringTok{ }
\StringTok{    }\KeywordTok{summarize_all}\NormalTok{(mean)}
\end{Highlighting}
\end{Shaded}

\begin{Shaded}
\begin{Highlighting}[]
\NormalTok{iris }\OperatorTok\StringTok{ }\KeywordTok{group_by}\NormalTok{(Species) }\OperatorTok\StringTok{ }\KeywordTok{select}\NormalTok{(}\KeywordTok{contains}\NormalTok{(}\StringTok{"Petal"}\NormalTok{)) }\OperatorTok\StringTok{ }
\StringTok{    }\KeywordTok{summarize_all}\NormalTok{(mean)}
\end{Highlighting}
\end{Shaded}

\begin{Shaded}
\begin{Highlighting}[]
\NormalTok{iris }\OperatorTok\StringTok{ }\KeywordTok{group_by}\NormalTok{(Species) }\OperatorTok\StringTok{ }\KeywordTok{select}\NormalTok{(}\OperatorTok{-}\KeywordTok{contains}\NormalTok{(}\StringTok{"Sepal"}\NormalTok{)) }\OperatorTok\StringTok{ }
\StringTok{    }\KeywordTok{summarize_all}\NormalTok{(mean)}
\end{Highlighting}
\end{Shaded}

\label{tab:Selected}Promedio de largo de pétalo y ancho de pétalo para las
especies del genero Iris

Species

Petal.Length

Petal.Width

setosa

1.462

0.246

versicolor

4.260

1.326

virginica

5.552

2.026

\hypertarget{joins}{%
\subsection{Joins}\label{joins}}

Los ejemplos a continuación se basan en el código generado por Garrick
Aden-Buie en su repositorio de animaciones de verbos del tidyverse
\citep{AdenBuie2018}. El paquete \emph{dplyr}, tiene una serie de
funciones de apellido join: \texttt{anti\_join}, \texttt{full\_join},
\texttt{inner\_join}, \texttt{left\_join}, \texttt{right\_join} y
\texttt{semi\_join}, en general no son tan fáciles de entender a primera
vista, por lo que se trabajará con dos tablas muy simples (Tabla
\ref{tab:DosTablas}), las cuales tienen dos columnas cada una

\begin{table}
\caption{\label{tab:DosTablas}Dos tablas para unir.}

\centering
\begin{tabular}[t]{rl}
\toprule
id & x\\
\midrule
1 & x1\\
2 & x2\\
3 & x3\\
\bottomrule
\end{tabular}
\centering
\begin{tabular}[t]{rl}
\toprule
id & y\\
\midrule
1 & y1\\
2 & y2\\
4 & y4\\
\bottomrule
\end{tabular}
\end{table}

\hypertarget{left-join}{%
\section{left join}\label{left-join}}

Como vemos en la figura \ref{fig:leftjoin}

Entonces

\hypertarget{ejercicios}{%
\subsection{Ejercicios}\label{ejercicios}}

\hypertarget{ejercicio-1}{%
\subsubsection{Ejercicio 1}\label{ejercicio-1}}

Usando la base de datos \texttt{storms} del paquete \emph{dplyr},
calcular la velocidad promedio y diámetro promedio (hu\_diameter) de las
tormentas que han sido declaradas huracanes para cada año.

\hypertarget{ejercicio-2}{%
\subsubsection{Ejercicio 2}\label{ejercicio-2}}

La base de datos \texttt{mpg} del paquete ggplot2 tiene datos de
eficiencia vehicular en millas por galón en ciudad (\emph{cty}) en
varios vehículos. Obtener los datos de vehículos del año 2004 en
adelante que sean compactos y transformar la eficiencia Km/litro (1
milla = 1.609 km; 1 galón = 3.78541 litros)

Las soluciones a estos ejercicios se encuentran en el capítulo
\ref{soluciones}

\hypertarget{reproducible}{%
\chapter{Investigación reproducible}\label{reproducible}}

\hypertarget{paquetes-necesarios-para-este-capitulo-1}{%
\section{Paquetes necesarios para este
capítulo}\label{paquetes-necesarios-para-este-capitulo-1}}

Para este capítulo se necesita tener instalado los paquetes
\emph{rmarkdown}, \emph{knitr} y \emph{stargazer}

En este capítulo se explicará qué es investigación reproducible, cómo
aplicarla usando github más los paquetes \emph{rmarkdown}
\citep{Allaire2018} y \emph{knitr} \citep{xie2015}. Además, se aprenderá
a usar tablas usando \emph{knitr} \citep{xie2015} y \emph{stargazer}
\citep{hlavak2018}

Recuerda que este libro es un apoyo para el curso BIO4022, puedes seguir
la clase de este curso en este
\href{https://derek-corcoran-barrios.github.io/Clase2/Clase2InvestigacionReproducible}{link},
y en cuanto el video de la clase encontrarás un link aca.

\hypertarget{investigacion-reproducible}{%
\section{Investigación reproducible}\label{investigacion-reproducible}}

La investigación reproducible no es lo mismo que la investigación
replicable. La replicabilidad implica que experimentos o estudios
llevados a cabo en condiciones similares nos llevarán a conclusiones
similares. La investigación reproducible implica que desde los mismos
datos y/o el mismo código se generarán los mismos resultados.

\begin{figure}

{\centering \includegraphics[width=0.8\linewidth]{Reproducible} 

}

\caption{Continuo de reproducibilidad (extraido de Peng 2011)}\label{fig:reproducible}
\end{figure}

En la figura \ref{fig:reproducible} vemos el continuo de
reproducibilidad \citep{peng2011reproducible}. En este continuo tenemos
el ejemplo de no reproducibilidad como una publicación sin código. Se
pasa de menos a más reproducible por la publicación y el código que
generó los resultados y gráficos; seguido por la publicación, el código
y los datos que generan los resultados y gráficos; y por último código,
datos y texto entrelazados de forma tal que al correr el código
obtenemos exactamente la mismma publicación que leímos.

Esto tiene muchas ventajas, incluyendo el que es más fácil aplicar
exactamente los mismos métodos a otra base de datos. Basta poner la
nueva base de datos en el formato que tenía el autor de la primera
publicación y podremos comparar los resultados.

Además en un momento en que la ciencia está basada cada vez más en bases
de datos, se puede poner en el código la recolección y/o muestreo de
datos.

\hypertarget{guardando-nuestro-proyecto-en-github}{%
\section{Guardando nuestro proyecto en
github}\label{guardando-nuestro-proyecto-en-github}}

\hypertarget{que-es-github}{%
\subsection{Que es github?}\label{que-es-github}}

Github es una suerte de dropbox o google drive pensado para la
investigación reproducible, en donde cada proyecto es un
\emph{repositorio}. La mayoría de los investigadores que trabajan en
investigación reproducible dejan todo su trabajo documentado en sus
repositorios, lo cual permite interactuar con otros autores.

\hypertarget{creando-un-proyecto-de-github-en-rstudio}{%
\subsection{creando un proyecto de github en
RStudio}\label{creando-un-proyecto-de-github-en-rstudio}}

Para crear un proyecto en github presionamos \textbf{start a project} en
la página inicial de nuestra cuenta, como vemos en la figura
\ref{fig:Start}

\begin{figure}

{\centering \includegraphics[width=0.8\linewidth]{StartAProject} 

}

\caption{Para empezar un projecto en github, debes presionar Start a project en tu página de inicio}\label{fig:Start}
\end{figure}

Luego se debe crear un nombre único, y sin cambiar nada más presiona
\textbf{create repository} en el botón verde como vemos en la figura
\ref{fig:Name}.

\begin{figure}

{\centering \includegraphics[width=0.8\linewidth]{NombreRepo} 

}

\caption{Crea el nombre de tu repositorio y apreta el boton create repository}\label{fig:Name}
\end{figure}

Esto te llevará a una página donde aparecerá una url de tu nuevo
repositorio como en la figura \ref{fig:ssh}

\begin{figure}

{\centering \includegraphics[width=0.8\linewidth]{GitAdress} 

}

\caption{El contenido del cuadro en el cual dice ssh es la url de tu repisitorio}\label{fig:ssh}
\end{figure}

Para incorporar tu proyecto en tu repositorio, lo primero que debes
hacer es generar un proyecto en RStudio. Para esto debes ir en el menú
superior de \emph{Rstudio} a \emph{File \textgreater{} New Project
\textgreater{} Git} como se ve en las figuras \ref{fig:NewProject} y
\ref{fig:NewProject}.

\begin{figure}

{\centering \includegraphics[width=0.8\linewidth]{NewProject} 

}

\caption{Menú para crear un proyecto nuevo}\label{fig:NewProject}
\end{figure}

\begin{figure}

{\centering \includegraphics[width=0.8\linewidth]{Git} 

}

\caption{Seleccionar git dentro de las opciones}\label{fig:Git}
\end{figure}

Luego seleccionar la ubicación del proyecto nuevo y pegar el url que
aparece en la figura \ref{fig:ssh} en el espacio que dice
\textbf{Repository URL:}, como muestra en la figura
\ref{fig:GitRstudio}.

\begin{figure}

{\centering \includegraphics[width=0.8\linewidth]{GitRstudio} 

}

\caption{Pegar el url del repositorio en el cuadro de dialogo Repository URL:}\label{fig:GitRstudio}
\end{figure}

Cuando tu proyecto de R ya este siguiendo los cambios en github, te
aparecerá una pestaña git dentro de la ventana superior derecha de tu
sesión de RStudio, tal como vemos en la figura \ref{fig:GitPan}

\begin{figure}

{\centering \includegraphics[width=0.8\linewidth]{GitPan} 

}

\caption{Al incluir tu repositorio en tu sesión de Rstudio, aparecera la pestaña git en la ventana superior derecha}\label{fig:GitPan}
\end{figure}

\hypertarget{los-tres-principales-pasos-de-un-repositorio}{%
\subsection{Los tres principales pasos de un
repositorio}\label{los-tres-principales-pasos-de-un-repositorio}}

Github es todo un mundo, existen muchas funciones y hay expertos en el
uso de github. En este curso, nos enfocaremos en los 3 pasos principales
de un repositorio: \emph{add}, \emph{commit} y \emph{push}. Para
entender bien qué significa cada uno de estos pasos, tenemos que
entender que existen dos repositorios en todo momento: uno local (en tu
computador) y otro remoto (en github.com). Los dos primeros pasos
\emph{add} y \emph{commit}, solo generan cambios en tu repositorio
local. Mientras que \emph{push}, salva los cambios al repositorio
remoto.

\hypertarget{git-add}{%
\subsubsection{git add}\label{git-add}}

Esta función es la que agrega archivos a tu repositorio local. Solo
estos archivos serán guardados en github. Github tienen un límite de
tamaño de repositorio de 1 GB y de archivos de 100 MB, ya que si bien te
dan repositorios ilimitados, el espacio de cada uno no lo es, en
particular en cuanto a bases de datos. Para adicionar un archivo a tu
repositorio tan solo debes selecionar los archivos en la pestaña git. Al
hacer eso una letra A verde aparecerá en vez de los dos signos de
interrogación amarillos, como vemos en la figura \ref{fig:Add}. En este
caso solo adicionamos al repositorio el archivo \emph{Analisis.r} pero
no el resto.

\begin{figure}

{\centering \includegraphics[width=0.8\linewidth]{GitAdd} 

}

\caption{Al incluir tu repositorio en tu sesión de Rstudio, aparecera la pestaña git en la ventana superior derecha}\label{fig:Add}
\end{figure}

\hypertarget{git-commit}{%
\subsubsection{git commit}\label{git-commit}}

Cuando ocupas el comando \emph{commit} estas guardando los cambios de
los archivos que adicionaste en tu repositorio local. Para hacer esto en
Rstudio, en la misma pestaña de git, debes presionar el botón
\emph{commit} como vemos en la figura \ref{fig:Commit}.

\begin{figure}

{\centering \includegraphics[width=0.8\linewidth]{Commit} 

}

\caption{Para guardar los cambios en tu repositorio apretar commit en la pestaña git de la ventana superior derecha}\label{fig:Commit}
\end{figure}

Al presionar \emph{commit}, se abrirá una ventana emergente, donde
deberás escribir un mensaje que describa lo que guardarás. Una vez echo
eso, presiona \emph{commit} nuevamente en la ventana emergente como
aparece en la figura \ref{fig:MensajeCommit}.

\begin{figure}

{\centering \includegraphics[width=0.8\linewidth]{MensajeCommit} 

}

\caption{Escribir un mensaje que recuerde los cambios que hiciste en la ventana emergente}\label{fig:MensajeCommit}
\end{figure}

\hypertarget{git-push}{%
\subsubsection{git push}\label{git-push}}

Finalmente, \emph{push} te permitirá guardar los cambios en tu
repositorio remoto, lo cual asegura tus datos en la nube y además lo
hace disponible a otros investigadores. Luego de apretar \emph{commit}
en la ventana emergente (figura \ref{fig:MensajeCommit}), podemos
presionar \emph{push} en la flecha verde de la ventana emergente como se
ve el a figura \ref{fig:push}. Luego se nos pedirá nuestro nombre de
usuario y contraseña, y ya podemos revisar que nuestro repositorio esta
online entrando a nuestra sesión de github.

\begin{figure}

{\centering \includegraphics[width=0.8\linewidth]{Push} 

}

\caption{Para guardar en el repositorio remoto apretar push en la ventana emergente}\label{fig:push}
\end{figure}

\hypertarget{reproducibilidad-en-r}{%
\section{Reproducibilidad en R}\label{reproducibilidad-en-r}}

Existen varios paquetes que permiten que hagamos investigación
reproducible en \texttt{R}, pero sin duda los más relevantes son
\emph{rmarkdown} y \emph{knitr}. Ambos paquetes funcionan en conjunto
cuando generamos un archivo \emph{Rmd} (Rmarkdown), en el cual ocupamos
al mismo tiempo texto, código de R y otros elementos para generar un
documento word, pdf, página web, presentación y/o aplicación web (fig
\ref{fig:Rmark}).

\begin{figure}

{\centering \includegraphics[width=0.8\linewidth]{Rmark} 

}

\caption{El objetivo de Rmarkdown es el unir código de r con texto y datos para generar un documento reproducible}\label{fig:Rmark}
\end{figure}

\hypertarget{creando-un-rmarkdown}{%
\subsection{Creando un Rmarkdown}\label{creando-un-rmarkdown}}

Para crear un archivo Rmarkdown, simplemente ve a el menu \emph{File
\textgreater{} New file \textgreater{} Rmarkdown} y con eso habrás
creado un nuevo archivo \emph{Rmd}. Veremos algunos de los elementos más
típicos de un archivo Rmarkdown.

\hypertarget{markdown}{%
\subsubsection{Markdown}\label{markdown}}

El markdown es la parte del archivo en que simplemente escribimos texto,
aunque tiene algunos detalles para el formato como generar texto en
negrita, cursiva, títulos y subtitulos.

Para hacer que un texto este en \textbf{negrita}, se debe poner entre
dos asteriscos \texttt{**negrita**}, para que un texto aparezca en
\emph{cursiva} debe estar entre asteriscos \texttt{*cursiva*}. Otros
ejemplos son los títulos de distintos niveles, los cuales se denotan con
distintos números de \texttt{\#}, así los siguientes 4 títulos o
subtítulos:

\hypertarget{subtitulo-1}{%
\section*{subtitulo 1}\label{subtitulo-1}}
\addcontentsline{toc}{section}{subtitulo 1}

\hypertarget{subtitulo-2}{%
\subsection*{subtítulo 2}\label{subtitulo-2}}
\addcontentsline{toc}{subsection}{subtítulo 2}

\hypertarget{subtitulo-3}{%
\subsubsection*{subtítulo 3}\label{subtitulo-3}}
\addcontentsline{toc}{subsubsection}{subtítulo 3}

\hypertarget{subtitulo-4}{%
\paragraph{subtítulo 4}\label{subtitulo-4}}
\addcontentsline{toc}{paragraph}{subtítulo 4}

se vería de la siguiente manera en el código

\begin{Shaded}
\begin{Highlighting}[]
\NormalTok{## subtitulo 1}

\NormalTok{### subtítulo 2}

\NormalTok{#### subtítulo 3}

\NormalTok{##### subtítulo 4}
\end{Highlighting}
\end{Shaded}

\hypertarget{chunks}{%
\subsubsection{Chunks}\label{chunks}}

Los \emph{chunks} son una de las partes más importantes del un
Rmarkdown. En estos es donde se agrega el código de R (u otros lenguajes
de programación). Lo cual permíte que el producto de nuestro código no
sea sólo un escrito con resultados pegados, sino que efectivamente
generados en el mismo documento que nuestro escrito. La forma más fácil
de agregar un chunk es apretando el botón de \emph{insert chunk} en
Rstudio, este boton se encuentra en la ventana superior izquierda de
nuestra sesión de RStudio, tal como se muestra en la figura
\ref{fig:Insertchunk}

\begin{figure}

{\centering \includegraphics[width=0.8\linewidth]{Insertchunk} 

}

\caption{Al apretar el botón insert chunk, aparecera un espacio en el cuál insertar código}\label{fig:Insertchunk}
\end{figure}

Al apretar este botón aparecera un espacio, ahí se puede agregar un
código como el que aparece a continuación, y ver a continuación los
resultados.

\begin{Shaded}
\begin{Highlighting}[]
\NormalTok{```\{r\}}
\NormalTok{library(tidyverse)}
\NormalTok{iris %>% group_by(Species) %>% summarize(Petal.Length = mean(Petal.length))}
\NormalTok{```}
\end{Highlighting}
\end{Shaded}

\begin{verbatim}
## # A tibble: 3 x 2
##   Species    Petal.Length
##   <fct>             <dbl>
## 1 setosa             1.46
## 2 versicolor         4.26
## 3 virginica          5.55
\end{verbatim}

\hypertarget{opciones-de-los-chunks}{%
\paragraph{Opciones de los chunks}\label{opciones-de-los-chunks}}

Existen muchas opciones para los chunks, una documentación completa
podemos encontrarle en el siguiente
\href{https://yihui.name/knitr/options/}{link}, pero acá mostraremos los
más comunes:

\begin{itemize}
\tightlist
\item
  \emph{echo} = T o F muestro o no el código, respectivamente
\item
  \emph{message} = T o F muestra mensajes de paquetes, respectivamente
\item
  \emph{warning} = T o F muestra advertencias, respectivamente
\item
  \emph{eval} = T o F evaluar o no el código, respectivamente
\item
  \emph{cache} = T o F guarda o no el resultado, respectivamente
\end{itemize}

\hypertarget{inline-code}{%
\subsubsection{inline code}\label{inline-code}}

Los \emph{inline codes} son útiles para agregar algún valor en el texto,
como por ejemplo el valor de p o la media. Para usarlo, se debe poner un
backtick (comilla simple hacia atrás), r, el código en cuestion y otro
backtick como se ve a continuación
\texttt{\textasciigrave{}r\ R\_código\textasciigrave{}}. No podemos
poner cualquier cosa en un \emph{inline code}, ya que sólo puede generar
vectores, lo cuál muchas veces requiere de mucha creatividad para lograr
lo que queremos. Por ejemplo si se quisiera poner el promedio del largo
del sépalo de la base da dato \texttt{iris} en un \emph{inline code}
pondríamos
\texttt{\textasciigrave{}r\ mean(iris\$Sepal.Length)\textasciigrave{}},
lo cual resultaría en 5.8433333. Como en un texto se vería extraño un
número con 7 cifras significativas, querríamos usar ademas la función
\texttt{round}, para que tenga 2 cifras significativas, para eso ponemos
el siguiente inline code
\texttt{\textasciigrave{}r\ round(mean(iris\$Sepal.Length),2)\textasciigrave{}}
que da como resultado 5.84. Esto se puede complejizar más aún si se
quiere trabajar con una tabla resumen. Por ejemplo, si quisieramos
listar el promedio del tamaño de sépalo usaríamos \texttt{summarize} de
\emph{dplyr}, pero esto nos daría como resultado un data.frame, el cual
no aparece si intentamos hacer un inline code. Partamos por ver como se
vería el código donde obtuvieramos la media del tamaño del sépalo.

\begin{Shaded}
\begin{Highlighting}[]
\NormalTok{iris }\OperatorTok\StringTok{ }\KeywordTok{group_by}\NormalTok{(Species) }\OperatorTok\StringTok{ }\KeywordTok{summarize}\NormalTok{(}\DataTypeTok{Mean =} \KeywordTok{mean}\NormalTok{(Sepal.Length))}
\end{Highlighting}
\end{Shaded}

El resultado de ese código lo veríamos \ref{tab:SummarySepaltab}

\label{tab:SummarySepaltab}Resumen del promedio del largo de sépalo de las
flores del genero Iris.

Species

Mean

setosa

5.006

versicolor

5.936

virginica

6.588

Para sacar de este data frame el vector de la media podríamos
subsetearlo con el signo \texttt{\$}. Entonces si queremos sacar como
vector la columna \emph{Mean} del data frame que creamos, haríamos lo
siguiente
\texttt{\textasciigrave{}r\ (iris\ \%\textgreater{}\%\ group\_by(Species)\ \%\textgreater{}\%\ summarize(Mean\ =\ mean(Sepal.Length)))\$Mean\textasciigrave{}}.
Esto daría como resultado 5.006, 5.936, 6.588.

\hypertarget{ejercicios-1}{%
\subsection{Ejercicios}\label{ejercicios-1}}

\hypertarget{ejercicio-1-1}{%
\subsubsection{Ejercicio 1}\label{ejercicio-1-1}}

Usando la base de datos \emph{iris}, crea un inline code que diga cuál
es la media del largo del pétalo de la especie \emph{Iris virginica}

La solución a este ejercicio se encuentra en el capítulo
\ref{soluciones}

\hypertarget{tablas-en-rmarkdown}{%
\subsubsection{Tablas en Rmarkdown}\label{tablas-en-rmarkdown}}

La función más típica para generar tablas en un archivo \emph{rmd} es
\texttt{kable} del paquete \emph{knitr}, que en su forma más simple se
incluye un dataframe como único argumento. Además de esto, podemos
agregar algunos parámetros como \emph{caption}, que nos permite poner un
título a la tabla o \emph{row.names}, que si se pone como se ve en el
código (FALSE) no mostrará en la tabla los nombres de las filas, tal
como se ve en la tabla \ref{tab:SummaryMeans}.

\begin{Shaded}
\begin{Highlighting}[]
\NormalTok{DF <-}\StringTok{ }\NormalTok{iris }\OperatorTok\StringTok{ }\KeywordTok{group_by}\NormalTok{(Species) }\OperatorTok\StringTok{ }\KeywordTok{summarize_all}\NormalTok{(mean)}
\KeywordTok{kable}\NormalTok{(DF, }\DataTypeTok{caption =} \StringTok{"Promedio por especie de todas las variables de la base de datos iris."}\NormalTok{, }
    \DataTypeTok{row.names =} \OtherTok{FALSE}\NormalTok{)}
\end{Highlighting}
\end{Shaded}

\label{tab:SummaryMeans}Promedio por especie de todas las variables de la
base de datos iris.

Species

Sepal.Length

Sepal.Width

Petal.Length

Petal.Width

setosa

5.006

3.428

1.462

0.246

versicolor

5.936

2.770

4.260

1.326

virginica

6.588

2.974

5.552

2.026

\hypertarget{tidyverso}{%
\chapter{El Tidyverso y tidyr}\label{tidyverso}}

\hypertarget{paquetes-necesarios-para-este-capitulo-2}{%
\section{Paquetes necesarios para este
capítulo}\label{paquetes-necesarios-para-este-capitulo-2}}

Para este capítulo necesitas tener instalado el paquete \emph{tidyverse}
y el paquete \emph{dismo} para uno de los ejercicios.

En este capítulo se explicará qué es el paquete \emph{tidyverse}
\citep{Wickhamtidyverse} y cuales son sus componentes. Además veremos
las funciones del paqute \emph{tidyr} \citep{Wickhamtidy} con sus dos
funciones \texttt{gather} y \texttt{spread}.

Dado que este libro es un apoyo para el curso BIO4022, esta clase puede
también ser seguida en este
\href{https://derek-corcoran-barrios.github.io/Clase3/Clase3Hadleyverso}{link}.
El video de la clase se encontrará disponible en este
\href{https://www.youtube.com/watch?v=UhmHsx5X9Ug\&feature=youtu.be}{link}.

\hypertarget{el-tidyverso}{%
\section{El tidyverso}\label{el-tidyverso}}

El tidiverso se refiere al paquete
\href{https://www.tidyverse.org/}{tidiverse}, el cual es una colección
de paquetes coehrentes, que tienen una gramática, filosofía y estructura
similar. Todos se basan en la idea de tidy data propuesta por Hadley
Wickham \citep{wickham2014tidy}.

Los paquetes que forman parte del tidyverso son:

\begin{itemize}
\tightlist
\item
  readr (ya la estamos usando)
\item
  dplyr (Clase anterior)
\item
  tidyr (Hoy)
\item
  ggplot2 (Próxima clase)
\item
  purrr (En clase sobre loops)
\item
  forcats (Para variables categóricas)
\item
  stringr (Para carácteres, Palabras)
\end{itemize}

\hypertarget{readr}{%
\subsection{readr}\label{readr}}

El paquete \emph{readr} \citep{Wickhamreadr} tiene como función el
importar (leer) y exportar archivos. Dado que en general nosotros
usaremos archivos del tipo \emph{csv}, para este tipo de archivos,
\emph{readr} tiene la función \texttt{read\_csv}. Para exportar un
archivo ocupamos la función \texttt{write\_csv}. Ambas funciones son 10
veces más rápidas que las versiones de r base. Para más información
sobre este revisar su \href{https://readr.tidyverse.org/}{página
oficial}.

\hypertarget{dplyr-1}{%
\subsection{dplyr}\label{dplyr-1}}

Este paquete sirve para modificar variables y sus detalles los vimos en
el capítulo \ref{tidydata}. Los cinco verbos principales que tiene son
\texttt{mutate} para generar nuevas variables y que vienen de variables
ya existentes, \texttt{select} para seleccionar variables basadas en su
nombre, \texttt{filter} para seleccionar filas de acuerdo a si cumplen o
no con condiciones en una o mas variables, \texttt{summarize} para
resumir las variables, y \texttt{arrange} para reordenar las filas de
acuerdo a alguna variable. Para más información sobre este paquete
revisar su \href{https://dplyr.tidyverse.org/}{página oficial}.

\hypertarget{tidyr}{%
\subsection{tidyr}\label{tidyr}}

Con sólo dos funciones: \texttt{gather} y \texttt{spread}. El paquete
\emph{tidyr} \citep{Wickhamtidy} tiene como finalidad el tomar bases de
datos no tidy y transformalas en tidy (datos limpios y ordenados). Para
esto, \texttt{gather} transforma tablas anchas en largas y
\texttt{spread} transforma tablas anchas en larga. En este capítulo
explicaremos en más detalle estos dos verbos. Para más información sobre
este paquete revisar su \href{https://tidyr.tidyverse.org/}{página
oficial}.

\hypertarget{ggplot2}{%
\subsection{ggplot2}\label{ggplot2}}

Una vez que una base de datos está en formato tidy, podemos usar
\emph{ggplot2} \citep{Wickhamggplot} para visualizar estos datos. Los
datos pueden ser categóricos, continuos e incluso espaciales en conjunto
con el paquete \emph{sf}. Este paquete es el más antiguo del
\emph{tidyverse} y por ello posee una gramática un poco diferente.
Hablaremos más de este paquete en el capítulo \ref{visualizacion}. Por
ahora si se quiere aprender más sobre \emph{ggplot2} se puede revisar la
\href{https://ggplot2.tidyverse.org/}{página oficial}

\hypertarget{purrr}{%
\subsection{purrr}\label{purrr}}

\emph{Purrr} \citep{HenryPurrr} permite formular loops de una forma más
sencilla e intuitiva que los \textbf{for} loops. Utilizando sus
funciones \texttt{map}, \texttt{map2}, \texttt{walk} y \texttt{reduce}
podemos realizar loops dentro de la gramática del tidyverse.
Trabajaremos en este paquete en el capítulo \ref{loops}. Como siempre
puedes encontras más información en su
\href{https://purrr.tidyverse.org/}{página oficial}

\hypertarget{forcats}{%
\subsection{forcats}\label{forcats}}

Trabajar con factores es una de las labores más complejas en R, es por
eso que se creó el paquete \emph{forcats} \citep{Wickhamforcats}. Si
bien no hay un capítulo en este libro en el cuál se trabajará
exclusivamente con este paquete, se utilizará al menos una función en el
capítulo \ref{visualizacion}

\hypertarget{stringr}{%
\subsection{stringr}\label{stringr}}

El modíficar variables de texto para hacer que las variables tengan
sentido humano es algo muy importante, para este tipo de modificaciones
se utiliza el paquete \emph{stringr} \citep{Wickhamstringr}. En este
capítulo, para algunos ejercicios, introduciremos algunas
funcionalidades de este paquete. Para más información revisar su
\href{https://stringr.tidyverse.org/}{página oficial}.

\hypertarget{tidyr-1}{%
\section{tidyr}\label{tidyr-1}}

Este paquete como ya fue explicado en la sección anterior, solo posee
dos funciones: \texttt{gather} y \texttt{spread}. Estas funciones sirven
para pasar de tablas anchas a largas y viceversa, pero ¿qué significa
que la misma información sea presentada en una tabla larga o en una
tabla ancha?

Tomemos por ejemplo dos tablas. En la tabla \ref{tab:TablaAncha} vemos
una tabla ancha y en la tabla \ref{tab:TablaLarga} una tabla larga.

\label{tab:TablaAncha}Tabla ancha.

Sepal.Length

Sepal.Width

Petal.Length

Petal.Width

Species

5.1

3.5

1.4

0.2

setosa

7.0

3.2

4.7

1.4

versicolor

6.3

3.3

6.0

2.5

virginica

\label{tab:TablaLarga}Tabla larga.

Species

Atributos\_florales

Medidas

setosa

Sepal.Length

5.1

versicolor

Sepal.Length

7.0

virginica

Sepal.Length

6.3

setosa

Sepal.Width

3.5

versicolor

Sepal.Width

3.2

virginica

Sepal.Width

3.3

setosa

Petal.Length

1.4

versicolor

Petal.Length

4.7

virginica

Petal.Length

6.0

setosa

Petal.Width

0.2

versicolor

Petal.Width

1.4

virginica

Petal.Width

2.5

\hypertarget{dato}{%
\subsubsection{DATO}\label{dato}}

Usualmente las tablas anchas son mejores para ser mostradas ya que se
distinguen más facilmente las variables trabajadas, mientras que las
tablas largas son mejores para programar y hacer análisis.

\hypertarget{gather}{%
\subsection{gather}\label{gather}}

Esta función nos permite pasar de una tabla ancha a una larga. En muchos
casos esto es necesario para generar una base de datos \emph{tidy}, y en
otras ocaciones es importante para generación de gráficos que
necesitamos tal como veremos en el capítulo \ref{visualizacion}. En esta
función partimos con un data frame y luego tenemos 3 argumentos: en el
primero \texttt{key}, ponemos el nombre de la variable que va a llevar
como observaciones los nombres de las columnas; luego en el argumento
\emph{value}, ponemos el nombre de la columna que llevará los valores de
cada columna al transformarse en una columna larga; Por último hay un
argumento (sin nombre), en el cual ponemos las columas que queremos que
sean ``\emph{alargadas}'', o con un signo negativo, las que no queremos
que sean parte de esta transformación. Todo esto quedará más claro en el
siguiente ejemplo.

\hypertarget{ejemplo-de-los-censos}{%
\subsubsection{Ejemplo de los censos}\label{ejemplo-de-los-censos}}

Supongamos que un estudiante de biología va a realizar un censo en un
parque nacional por tres días y genera la siguiente tabla (el código a
continuación es el que permite generar el data frame obervado en la
tabla \ref{tab:Censo})

\begin{Shaded}
\begin{Highlighting}[]
\NormalTok{df_cuentas <-}\StringTok{ }\KeywordTok{data.frame}\NormalTok{(}\DataTypeTok{dia =} \KeywordTok{c}\NormalTok{(}\StringTok{"Lunes"}\NormalTok{, }\StringTok{"Martes"}\NormalTok{, }\StringTok{"Miercoles"}\NormalTok{), }
    \DataTypeTok{Lobo =} \KeywordTok{c}\NormalTok{(}\DecValTok{2}\NormalTok{, }\DecValTok{1}\NormalTok{, }\DecValTok{3}\NormalTok{), }\DataTypeTok{Liebre =} \KeywordTok{c}\NormalTok{(}\DecValTok{20}\NormalTok{, }\DecValTok{25}\NormalTok{, }\DecValTok{30}\NormalTok{), }\DataTypeTok{Zorro =} \KeywordTok{c}\NormalTok{(}\DecValTok{4}\NormalTok{, }\DecValTok{4}\NormalTok{, }
        \DecValTok{4}\NormalTok{))}
\end{Highlighting}
\end{Shaded}

\label{tab:Censo}Abundancia detectada por especie en tres días de muestreo

dia

Lobo

Liebre

Zorro

Lunes

2

20

4

Martes

1

25

4

Miercoles

3

30

4

Claramente esta base de datos no es tidy, ya que deberíamos tener una
columna para la variable día, otra para especie y por último una para la
abundancia de cada especie en cadad día. Antes de mostrar como
realizaríamos esto con \texttt{gather}, veamos sus efectos para
entenderlo mejor. La forma más básica de usar esta función sería el solo
darle un nombre a la columna \emph{key} (que tendrá el nombre de las
columnas) y otro a \emph{value}, que tendría el valor de las celdas.
Veamos que ocurre si hacermos eso en el siguiente código y tabla
\ref{tab:Larga1}.

\begin{Shaded}
\begin{Highlighting}[]
\KeywordTok{library}\NormalTok{(tidyverse)}
\NormalTok{DF_largo <-}\StringTok{ }\NormalTok{df_cuentas }\OperatorTok\StringTok{ }\KeywordTok{gather}\NormalTok{(}\DataTypeTok{key =}\NormalTok{ Columnas, }\DataTypeTok{value =}\NormalTok{ Valores)}
\end{Highlighting}
\end{Shaded}

\label{tab:Larga1}Abundancia detectada por especie en tres días de muestreo

Columnas

Valores

dia

Lunes

dia

Martes

dia

Miercoles

Lobo

2

Lobo

1

Lobo

3

Liebre

20

Liebre

25

Liebre

30

Zorro

4

Zorro

4

Zorro

4

Como vemos en la tabla \ref{tab:Larga1}, en la columna llamada
\emph{Columnas}, tenemos sólo los nombres de las columnas de la tabla
\ref{tab:Censo}, y en la columna \emph{Valores}, tenemos los valores
encontrados en la tabla \ref{tab:Censo}. Sin embargo, para tener las
tres columnas que desearíamos tener (día, especie y abundancia),
necesitamos que la variable día no participe de este
``\emph{alargamiento}'', para esto lo que haríamos sería los siguiente:

\begin{Shaded}
\begin{Highlighting}[]
\NormalTok{DF_largo <-}\StringTok{ }\NormalTok{df_cuentas }\OperatorTok\StringTok{ }\KeywordTok{gather}\NormalTok{(}\DataTypeTok{key =}\NormalTok{ Columnas, }\DataTypeTok{value =}\NormalTok{ Valores, }
    \OperatorTok{-}\NormalTok{dia)}
\end{Highlighting}
\end{Shaded}

Al agregar \texttt{-día} como tercer argumento quitamos esa variable del
día en el ``\emph{alargamiento}'', en ese caso obtenemos la tabla
\ref{tab:Larga2}. Ahora sólo falta arreglar los nombres.

\label{tab:Larga2}Abundancia detectada por especie en tres días de muestreo

dia

Columnas

Valores

Lunes

Lobo

2

Martes

Lobo

1

Miercoles

Lobo

3

Lunes

Liebre

20

Martes

Liebre

25

Miercoles

Liebre

30

Lunes

Zorro

4

Martes

Zorro

4

Miercoles

Zorro

4

Para cambiar los nombres de las columnas que nos faltan, sólo cambiamos
los valores de los argumentos \texttt{key} y \texttt{value} como se ve a
continuación y en la tabla \ref{tab:Larga3}.

\begin{Shaded}
\begin{Highlighting}[]
\NormalTok{DF_largo <-}\StringTok{ }\NormalTok{df_cuentas }\OperatorTok\StringTok{ }\KeywordTok{gather}\NormalTok{(}\DataTypeTok{key =}\NormalTok{ Especie, }\DataTypeTok{value =}\NormalTok{ Abundancia, }
    \OperatorTok{-}\NormalTok{dia)}
\end{Highlighting}
\end{Shaded}

\label{tab:Larga3}Abundancia detectada por especie en tres días de muestreo

dia

Especie

Abundancia

Lunes

Lobo

2

Martes

Lobo

1

Miercoles

Lobo

3

Lunes

Liebre

20

Martes

Liebre

25

Miercoles

Liebre

30

Lunes

Zorro

4

Martes

Zorro

4

Miercoles

Zorro

4

\hypertarget{spread}{%
\subsection{spread}\label{spread}}

\texttt{spread} es la función inversa a \texttt{gather}, esto es, toma
una tabla de datos en formato ancho y la trnasforma en una base de datos
de formato largo. Esta función tiene dos argumentos básicos. \emph{key}
que es el nombre de la variable que pasará a ser nombres de columna y
\emph{value}, que es el nombre de la columna con los valores que
llenarán estas columnas.

\hypertarget{continuacion-ejemplo-de-censos}{%
\subsubsection{Continuación ejemplo de
censos}\label{continuacion-ejemplo-de-censos}}

Volvamos al ejemplo de los censos donde quedamos, en nuestro último
ejercicio creamos el data frame \emph{DF\_largo} que vemos en la tabla
\ref{tab:Larga3}. Veremos algunos ejemplos de como podemos cambiar este
data frame en una tabla ancha:

\begin{Shaded}
\begin{Highlighting}[]
\NormalTok{DF_ancho <-}\StringTok{ }\NormalTok{DF_largo }\OperatorTok\StringTok{ }\KeywordTok{spread}\NormalTok{(}\DataTypeTok{key =}\NormalTok{ dia, }\DataTypeTok{value =}\NormalTok{ Abundancia)}
\end{Highlighting}
\end{Shaded}

Con el código anterior generamos la \ref{tab:Ancha1}, la cuál es
distinta a la original en la tabla \ref{tab:Censo}), en esta los días
quedaron como nombres de columnas, y las especies pasaron a ser una
variable.

\label{tab:Ancha1}Abundancia detectada por especie en tres días de muestreo

Especie

Lunes

Martes

Miercoles

Liebre

20

25

30

Lobo

2

1

3

Zorro

4

4

4

En la tabla \ref{tab:Ops} se ven todas las opciones de como generar una
tabla ancha en base a el data frame \emph{DF\_largo}, pruebe opciones
hasta entender la función, algunas de estas opciones darán errores.

\label{tab:Ops}Todas las opciones a probar para generar una tabla ancha

Key

Value

Especie

dia

Abundancia

dia

dia

Especie

Abundancia

Especie

dia

Abundancia

Especie

Abundancia

\hypertarget{ejercicios-2}{%
\section{Ejercicios}\label{ejercicios-2}}

\hypertarget{ejercicio-1-2}{%
\subsection{Ejercicio 1}\label{ejercicio-1-2}}

Utilizando el siguiente código usando el paquete dismo bajaras la base
de datos del \emph{GBIF} (Global Biodiversity Information Facility) de
presencias conocidas del huemul (\emph{Hippocamelus bisulcus}):

\begin{Shaded}
\begin{Highlighting}[]
\KeywordTok{library}\NormalTok{(dismo)}
\NormalTok{Huemul <-}\StringTok{ }\KeywordTok{gbif}\NormalTok{(}\StringTok{"Hippocamelus"}\NormalTok{, }\StringTok{"bisulcus"}\NormalTok{, }\DataTypeTok{down =} \OtherTok{TRUE}\NormalTok{)}
\KeywordTok{colnames}\NormalTok{(Huemul)}
\end{Highlighting}
\end{Shaded}

Tomando la base de datos generada:

\begin{enumerate}
\def\labelenumi{\alph{enumi}.}
\item
  Quedarse con solo las observaciones que tienen coordenadas geograficas
\item
  Determinar cuantas observaciones son de observacion humana y cuantas
  de especimen de museo
\end{enumerate}

\hypertarget{ejercicio-2-1}{%
\subsection{Ejercicio 2}\label{ejercicio-2-1}}

Entrar a
\href{http://www.ine.cl/estadisticas/medioambiente/series-cronologicas-vba}{INE
ambiental} y bajar la base de datos de Dimensión Aire.

\begin{enumerate}
\def\labelenumi{\alph{enumi}.}
\tightlist
\item
  Generar una base de datos \textbf{tidy} con las siguientes 5 columnas
\end{enumerate}

\begin{itemize}
\tightlist
\item
  El nombre de la localidad donde se encuntra la estación
\item
  El año en que se tomo la medida
\item
  El mes en que se tomo la medida
\item
  La temperatura media de ese mes
\item
  La media del mp25 de ese mes
\item
  Humedad relativa media mensual
\end{itemize}

\begin{enumerate}
\def\labelenumi{\alph{enumi}.}
\setcounter{enumi}{1}
\tightlist
\item
  De la base de datos anterior obterner un segundo data frame en la cual
  calculen para cada variable y estación la media y desviación estandar
  para cada mes
\end{enumerate}

\hypertarget{visualizacion}{%
\chapter{Visualización de datos}\label{visualizacion}}

\hypertarget{paquetes-necesarios-para-este-capitulo-3}{%
\section{Paquetes necesarios para este
capítulo}\label{paquetes-necesarios-para-este-capitulo-3}}

Para este capítulo necesitas tener instalado el paquete
\emph{tidyverse}.

En este capítulo se explicará qué es el paquete \emph{ggplot2}
\citep{Wickhamggplot} y cómo utilizarlo para visualizar datos.

Dado que este libro es un apoyo para el curso BIO4022, esta clase puede
también ser seguida en este
\href{https://derek-corcoran-barrios.github.io/Clase4/Clase4Visualizacion}{link}.
El video de la clase se encontrará disponible en este
\href{https://youtu.be/YKEiqSDz-c8}{link}

\hypertarget{el-esqueleto}{%
\section{El esqueleto}\label{el-esqueleto}}

El esqueleto de una visualización usando \emph{ggplot2} es la siguiente

\begin{Shaded}
\begin{Highlighting}[]
\KeywordTok{ggplot}\NormalTok{(data.frame, }\KeywordTok{aes}\NormalTok{(nombres_de_columna)) }\OperatorTok{+}\StringTok{ }\KeywordTok{geom_algo}\NormalTok{(argumentos, }
    \KeywordTok{aes}\NormalTok{(columnas)) }\OperatorTok{+}\StringTok{ }\KeywordTok{theme_algo}\NormalTok{()}
\end{Highlighting}
\end{Shaded}

Como ejemplo para discutir usaremos el siguiente código que genera la
figura \ref{fig:ejemplo1-ggplot}:

\begin{Shaded}
\begin{Highlighting}[]
\KeywordTok{library}\NormalTok{(tidyverse)}
\KeywordTok{data}\NormalTok{(}\StringTok{"diamonds"}\NormalTok{)}
\KeywordTok{ggplot}\NormalTok{(diamonds, }\KeywordTok{aes}\NormalTok{(}\DataTypeTok{x =}\NormalTok{ carat, }\DataTypeTok{y=}\NormalTok{price)) }\OperatorTok{+}\StringTok{ }\KeywordTok{geom_point}\NormalTok{(}\KeywordTok{aes}\NormalTok{(}\DataTypeTok{color =}\NormalTok{ cut)) }\OperatorTok{+}\StringTok{ }\KeywordTok{theme_classic}\NormalTok{()}
\end{Highlighting}
\end{Shaded}

\begin{figure}

{\centering \includegraphics[width=0.8\linewidth]{Libro_files/figure-latex/ejemplo1-ggplot-1} 

}

\caption{Gráfico en el cual gráficamos los quilates de diamantes versus su precio, con el corte del diamante representado por el color}\label{fig:ejemplo1-ggplot}
\end{figure}

En este caso general, lo primero que ponemos después de ggplot es el
data.frame desde el cual graficaremos algo. En el ejemplo de la figura
\ref{fig:ejemplo1-ggplot} usamos la base de datos \emph{diamonds} del
paquete \emph{ggplot2} \citep{Wickhamggplot}, luego dentro de
\texttt{aes} ponemos las columnas que graficaremos como \emph{x} y/o
\emph{y}. En nuestro ejemplo dentro de aes ponemos como eje \emph{x} los
quilates de los diamantes (caret) y como \emph{y} el precio de los
mismos (price). Ojo que existe la necesidad de poner \texttt{aes} en
ggplot2 (algo que no había sido necesario cuando usamos \emph{dplyr} o
\emph{tidyr}).

\hypertarget{por-que-usamos-aes-y}{%
\section{Por que usamos aes() y +}\label{por-que-usamos-aes-y}}

Al ser el primer paquete creado en el tidyverse, \emph{ggplot2} tiene un
par de convenciones distintas. Por un lado, cada vez que usamos el
nombre de una columna que está en un data frame debemos usarlo dentro de
la función \texttt{aes}. Además, cuando se creó el paquete
\emph{ggplot2} no existia el pipeline (\texttt{\%\textgreater{}\%}), por
lo que se utilizaba el signo \texttt{+} con la misma función.

\hypertarget{geom_algo}{%
\section{geom\_algo}\label{geom_algo}}

Luego de especificar una base de datos, se debe continuar con un
\texttt{geom\_algo}, esto nos indicará que tipo de gráfico usaremos. Los
gráficos pueden ser combinados como veremos en ejemplos futuros.

\hypertarget{una-variable-categorica-una-continua}{%
\subsection{Una variable categórica una
continua}\label{una-variable-categorica-una-continua}}

Primero veremos algunos de los \emph{geom} que podemos utilizar con una
variable categórica y una continua

\hypertarget{geom_boxplot}{%
\subsubsection{geom\_boxplot}\label{geom_boxplot}}

En la figura \ref{fig:boxplot}, generado a partir del código a
continuación con la base de datos iris presente en \texttt{R}
\citep{anderson1935irises}.

\begin{Shaded}
\begin{Highlighting}[]
\KeywordTok{data}\NormalTok{(}\StringTok{"iris"}\NormalTok{)}
\KeywordTok{ggplot}\NormalTok{(iris, }\KeywordTok{aes}\NormalTok{(}\DataTypeTok{x =}\NormalTok{ Species, }\DataTypeTok{y =}\NormalTok{ Sepal.Length)) }\OperatorTok{+}\StringTok{ }\KeywordTok{geom_boxplot}\NormalTok{()}
\end{Highlighting}
\end{Shaded}

\begin{figure}

{\centering \includegraphics[width=0.8\linewidth]{Libro_files/figure-latex/boxplot-1} 

}

\caption{Boxplot que representa los largos del sépalo de tres especies del género Iris}\label{fig:boxplot}
\end{figure}

Los boxplots muestran una línea gruesa central (la mediana), una caja,
que delimita el primer y tercer cuartil y los bigotes, los cuales se
extienden hasta los valores extremos. En el caso que estos valores estén
por sobre 1.5 veces la distancia entre el primer y tercer cuartil, estos
serán representados por puntos (siendo considerados outlyers). En la
figura \ref{fig:boxplot}, sólo \emph{Iris virginica} presenta un
outlayer en cuanto a las medidas del largo del sépalo.

Los boxplots, como todos los gráficos pueden ser personalizados usando
otros argumentos, los que mostraremos en esta sección los iremos
introduciendo de a poco. Si quisieramos por ejemplo que el color de las
cajas del \emph{boxplot} fueran dea cuerdo a la especie, cambiamos el
llenado (\textbf{fill}) de la caja, como vemos en el siguiente ejemplo y
figura \ref{fig:boxplot2}

\begin{Shaded}
\begin{Highlighting}[]
\KeywordTok{ggplot}\NormalTok{(iris, }\KeywordTok{aes}\NormalTok{(}\DataTypeTok{x =}\NormalTok{ Species, }\DataTypeTok{y =}\NormalTok{ Sepal.Length)) }\OperatorTok{+}\StringTok{ }\KeywordTok{geom_boxplot}\NormalTok{(}\KeywordTok{aes}\NormalTok{(}\DataTypeTok{fill =}\NormalTok{ Species))}
\end{Highlighting}
\end{Shaded}

\begin{figure}

{\centering \includegraphics[width=0.8\linewidth]{Libro_files/figure-latex/boxplot2-1} 

}

\caption{Boxplot que representa los largos del sépalo de tres especies del género Iris, en este caso el color de la caja representa la especie}\label{fig:boxplot2}
\end{figure}

Dos cosas a notar en este ejemplo, por un lado la leyenda se genera de
forma automática, y por otro lado, vemos que es necesario poner
\emph{Species} dentro de \texttt{aes}, esto es debido a que Species es
una columna y como se explicó al principio de este capítulo, todas las
columnas deben ser incuidas dentro de la función \texttt{aes} para poder
ser referenciadas.

\hypertarget{geom_jitter}{%
\subsubsection{geom\_jitter}\label{geom_jitter}}

Utilizando la misma base de datos, podemos crear un gráfico del tipo
\emph{jitter}. En este caso hay un punto por cada observación, lo cual
puede ayudar a entender mejor los datos que tenemos.

\begin{Shaded}
\begin{Highlighting}[]
\KeywordTok{ggplot}\NormalTok{(iris, }\KeywordTok{aes}\NormalTok{(}\DataTypeTok{x =}\NormalTok{ Species, }\DataTypeTok{y =}\NormalTok{ Sepal.Length)) }\OperatorTok{+}\StringTok{ }\KeywordTok{geom_jitter}\NormalTok{(}\KeywordTok{aes}\NormalTok{(}\DataTypeTok{color =}\NormalTok{ Species))}
\end{Highlighting}
\end{Shaded}

\begin{figure}

{\centering \includegraphics[width=0.8\linewidth]{Libro_files/figure-latex/jitter-1} 

}

\caption{jitter plot que representa los largos del sépalo de tres especies del género Iris, en este caso el color de los puntos representan la especie}\label{fig:jitter}
\end{figure}

En la figura \ref{fig:jitter} vemos los mismos datos que en la figura
\ref{fig:boxplot}, el agregar el \texttt{color\ =\ Species} dentro del
\texttt{aes} nos permite que el color de cada punto este determinado por
la especie a la que pertenece.

\hypertarget{otros-geom-categoricos}{%
\subsubsection{Otros geom categóricos}\label{otros-geom-categoricos}}

Otros geom categóricos que podemos explorar con esta base de datos son:

\begin{itemize}
\tightlist
\item
  geom\_violin
\item
  geom\_bar
\item
  geom\_col
\end{itemize}

\hypertarget{combinando-geoms}{%
\section{Combinando geoms}\label{combinando-geoms}}

Uno puede escribir varios geoms para formar un gráfico combinado. Por
ejemplo, podríamos generar un gráfico con un boxplot y un jitter plot,
como vemos en la figura \ref{fig:boxjitter}

\begin{Shaded}
\begin{Highlighting}[]
\KeywordTok{ggplot}\NormalTok{(iris, }\KeywordTok{aes}\NormalTok{(}\DataTypeTok{x =}\NormalTok{ Species, }\DataTypeTok{y =}\NormalTok{ Sepal.Length)) }\OperatorTok{+}\StringTok{ }\KeywordTok{geom_boxplot}\NormalTok{() }\OperatorTok{+}\StringTok{ }
\StringTok{    }\KeywordTok{geom_jitter}\NormalTok{(}\KeywordTok{aes}\NormalTok{(}\DataTypeTok{color =}\NormalTok{ Species))}
\end{Highlighting}
\end{Shaded}

\begin{figure}

{\centering \includegraphics[width=0.8\linewidth]{Libro_files/figure-latex/boxjitter-1} 

}

\caption{Boxplot y jitter plot combinados que representa los largos del sépalo de tres especies del género Iris.}\label{fig:boxjitter}
\end{figure}

\hypertarget{el-orden-importa}{%
\subsection{El orden importa}\label{el-orden-importa}}

Si bien se pueden combinar los geom, el orden de estos importa, ya que
\emph{ggplot2} genera las figuras por capas. Esto es ilustrado en la
figura \ref{fig:jitterbox}, en la cual al crear primero el jitter y
luego el boxplot, las cajas del boxplot tapan los puntos, a diferencia
de la figura \ref{fig:boxjitter}

\begin{Shaded}
\begin{Highlighting}[]
\KeywordTok{ggplot}\NormalTok{(iris, }\KeywordTok{aes}\NormalTok{(}\DataTypeTok{x =}\NormalTok{ Species, }\DataTypeTok{y =}\NormalTok{ Sepal.Length)) }\OperatorTok{+}\StringTok{ }\KeywordTok{geom_jitter}\NormalTok{(}\KeywordTok{aes}\NormalTok{(}\DataTypeTok{color =}\NormalTok{ Species)) }\OperatorTok{+}\StringTok{ }
\StringTok{    }\KeywordTok{geom_boxplot}\NormalTok{()}
\end{Highlighting}
\end{Shaded}

\begin{figure}

{\centering \includegraphics[width=0.8\linewidth]{Libro_files/figure-latex/jitterbox-1} 

}

\caption{Boxplot y jitter plot combinados que representa los largos del sépalo de tres especies del género Iris, en este caso al llamar al jitter antes del boxplot, las cajas tapan los puntos.}\label{fig:jitterbox}
\end{figure}

\hypertarget{dos-variables-continuas}{%
\subsection{Dos variables continuas}\label{dos-variables-continuas}}

Algunos de los geoms que podemos usar para dos variables continuas son:

\begin{itemize}
\tightlist
\item
  geom\_point
\item
  geom\_smooth
\item
  geom\_line
\item
  geom\_hex
\item
  geom\_rug
\end{itemize}

Ahora veremos algunos de ellos:

\hypertarget{geom_point}{%
\subsubsection{geom\_point}\label{geom_point}}

Este geom es el que nos permite hacer un gráfico de dispersión en R.
Para esto tenemos que poner variables continuas en x e y en ggplot y
agregar la función \texttt{geom\_point}, como vemos en el siguiente
código y en la figura \ref{fig:scatter}.

\begin{Shaded}
\begin{Highlighting}[]
\KeywordTok{data}\NormalTok{(}\StringTok{"ChickWeight"}\NormalTok{)}
\KeywordTok{ggplot}\NormalTok{(ChickWeight, }\KeywordTok{aes}\NormalTok{(}\DataTypeTok{x =}\NormalTok{ Time, }\DataTypeTok{y =}\NormalTok{ weight)) }\OperatorTok{+}\StringTok{ }\KeywordTok{geom_point}\NormalTok{()}
\end{Highlighting}
\end{Shaded}

\begin{figure}

{\centering \includegraphics[width=0.8\linewidth]{Libro_files/figure-latex/scatter-1} 

}

\caption{Gráfico en el cual vemos el peso de pollos en el tiempo}\label{fig:scatter}
\end{figure}

Si quisieramos que el color de cada punto estuviera separado por dieta,
podríamos agregarle \texttt{aes(color\ =\ Diet)} a geom\_point. Sin
embargo, deberíamos transformar Diet en factor, ya sea antes de usar
ggplot o dentro de ggplot tal como vemos en el siguiente código y en la
figura \ref{fig:scatterColor}.

\begin{Shaded}
\begin{Highlighting}[]
\KeywordTok{data}\NormalTok{(}\StringTok{"ChickWeight"}\NormalTok{)}
\KeywordTok{ggplot}\NormalTok{(ChickWeight, }\KeywordTok{aes}\NormalTok{(}\DataTypeTok{x =}\NormalTok{ Time, }\DataTypeTok{y =}\NormalTok{ weight)) }\OperatorTok{+}\StringTok{ }\KeywordTok{geom_point}\NormalTok{(}\KeywordTok{aes}\NormalTok{(}\DataTypeTok{color =} \KeywordTok{factor}\NormalTok{(Diet)))}
\end{Highlighting}
\end{Shaded}

\begin{figure}

{\centering \includegraphics[width=0.8\linewidth]{Libro_files/figure-latex/scatterColor-1} 

}

\caption{Gráfico en el cual vemos el peso de pollos en el tiempo, con colores distintos según el tipo de dieta}\label{fig:scatterColor}
\end{figure}

\hypertarget{geom_smooth-y-stat_smooth}{%
\subsubsection{geom\_smooth y
stat\_smooth}\label{geom_smooth-y-stat_smooth}}

\hypertarget{geom_smooth}{%
\paragraph{geom\_smooth}\label{geom_smooth}}

Estas funciones nos permiten generar líneas de tendencias con intervalos
de confianza. Así si quisieramos ver líneas de tendencias para nuestro
scatterplot, dependiendo de la dieta, usaríamos el siguiente código, el
cual nos da la figura \ref{fig:scatterLoess}.

\begin{Shaded}
\begin{Highlighting}[]
\KeywordTok{ggplot}\NormalTok{(ChickWeight, }\KeywordTok{aes}\NormalTok{(}\DataTypeTok{x =}\NormalTok{ Time, }\DataTypeTok{y =}\NormalTok{ weight)) }\OperatorTok{+}\StringTok{ }\KeywordTok{geom_point}\NormalTok{(}\KeywordTok{aes}\NormalTok{(}\DataTypeTok{color =} \KeywordTok{factor}\NormalTok{(Diet))) }\OperatorTok{+}\StringTok{ }
\StringTok{    }\KeywordTok{geom_smooth}\NormalTok{(}\KeywordTok{aes}\NormalTok{(}\DataTypeTok{fill =} \KeywordTok{factor}\NormalTok{(Diet)))}
\end{Highlighting}
\end{Shaded}

\begin{verbatim}
## `geom_smooth()` using method = 'loess' and formula 'y ~ x'
\end{verbatim}

\begin{figure}

{\centering \includegraphics[width=0.8\linewidth]{Libro_files/figure-latex/scatterLoess-1} 

}

\caption{Gráfico en el cual vemos el peso de pollos en el tiempo, con colores distintos según el tipo de dieta, con líneas de tendencia e intervalos de confianza basados en el método loess}\label{fig:scatterLoess}
\end{figure}

Por defecto, la función \texttt{geom\_smooth} generará una tendencia
basada en \emph{loess}, lo cual es una correlación local. En general, es
mejor hacer una línea de tendencia basado en modelos que uno puede
explicar mejor como un modelo lineal. Para esto, cambiamos el argumento
method a lm como en el siguiente código y la figura \ref{fig:scatterLM}.

\begin{Shaded}
\begin{Highlighting}[]
\KeywordTok{ggplot}\NormalTok{(ChickWeight, }\KeywordTok{aes}\NormalTok{(}\DataTypeTok{x =}\NormalTok{ Time, }\DataTypeTok{y =}\NormalTok{ weight)) }\OperatorTok{+}\StringTok{ }\KeywordTok{geom_point}\NormalTok{(}\KeywordTok{aes}\NormalTok{(}\DataTypeTok{color =} \KeywordTok{factor}\NormalTok{(Diet))) }\OperatorTok{+}\StringTok{ }
\StringTok{    }\KeywordTok{geom_smooth}\NormalTok{(}\KeywordTok{aes}\NormalTok{(}\DataTypeTok{fill =} \KeywordTok{factor}\NormalTok{(Diet)), }\DataTypeTok{method =} \StringTok{"lm"}\NormalTok{)}
\end{Highlighting}
\end{Shaded}

\begin{figure}

{\centering \includegraphics[width=0.8\linewidth]{Libro_files/figure-latex/scatterLM-1} 

}

\caption{Gráfico en el cual vemos el peso de pollos en el tiempo, con colores distintos según el tipo de dieta, con líneas de tendencia e intervalos de confianza basados en modelos lineales}\label{fig:scatterLM}
\end{figure}

\hypertarget{stat_smooth}{%
\paragraph{stat\_smooth}\label{stat_smooth}}

La función \texttt{stat\_smooth} es más flexible que
\texttt{geom\_smooth}. La gran diferencia es que nos permite incluir una
fórmula para expresar la relación entre \(x\) e \(y\). Por ejemplo, si
pensaramos que en el caso de la base de datos \texttt{ChickWeight} la
relación entre el peso y el tiempo se expresa mejor con una ecuación
cuadrática (ver ecuación \eqref{eq:quad}) tendríamos el siguiente código
que genera la figura \ref{fig:scatterQuad}.

\begin{Shaded}
\begin{Highlighting}[]
\KeywordTok{ggplot}\NormalTok{(ChickWeight, }\KeywordTok{aes}\NormalTok{(}\DataTypeTok{x =}\NormalTok{ Time, }\DataTypeTok{y =}\NormalTok{ weight)) }\OperatorTok{+}\StringTok{ }\KeywordTok{geom_point}\NormalTok{(}\KeywordTok{aes}\NormalTok{(}\DataTypeTok{color =} \KeywordTok{factor}\NormalTok{(Diet))) }\OperatorTok{+}\StringTok{ }
\StringTok{    }\KeywordTok{stat_smooth}\NormalTok{(}\KeywordTok{aes}\NormalTok{(}\DataTypeTok{fill =} \KeywordTok{factor}\NormalTok{(Diet)), }\DataTypeTok{method =} \StringTok{"lm"}\NormalTok{, }\DataTypeTok{formula =}\NormalTok{ y }\OperatorTok{~}\StringTok{ }
\StringTok{        }\NormalTok{x }\OperatorTok{+}\StringTok{ }\KeywordTok{I}\NormalTok{(x}\OperatorTok{^}\DecValTok{2}\NormalTok{))}
\end{Highlighting}
\end{Shaded}

\begin{figure}

{\centering \includegraphics[width=0.8\linewidth]{Libro_files/figure-latex/scatterQuad-1} 

}

\caption{Gráfico en el cual vemos el peso de pollos en el tiempo, con colores distintos según el tipo de dieta, con líneas de tendencia e intervalos de confianza basados en modelos lineales con una relación cuadrática}\label{fig:scatterQuad}
\end{figure}

\begin{equation} 
  y = \beta_2 x^2 + \beta_1 x + c
  \label{eq:quad}
\end{equation}

\hypertarget{combinando-varios-graficos-con-facet_wrap}{%
\subsection{Combinando varios gráficos con
facet\_wrap}\label{combinando-varios-graficos-con-facet_wrap}}

Algunas veces, en particular si tenemos muchas variables categóricas, no
es recomendable generar una línea o punto de color distinto para cada
variable. Por ejemplo, si seguimos con el crecimiento de los pollos de
la base de datos \texttt{ChickWeight}, vemos que la variable
\emph{Chick} representa cada pollo. Dado que hay varios pollos por dieta
se vuelve confuso y poco informativo como se ve en la figura
\ref{fig:Pollos} generada con el siguiente código.

\begin{Shaded}
\begin{Highlighting}[]
\KeywordTok{ggplot}\NormalTok{(ChickWeight, }\KeywordTok{aes}\NormalTok{(}\DataTypeTok{x =}\NormalTok{ Time, }\DataTypeTok{y =}\NormalTok{ weight)) }\OperatorTok{+}\StringTok{ }\KeywordTok{geom_point}\NormalTok{(}\KeywordTok{aes}\NormalTok{(}\DataTypeTok{color =}\NormalTok{ Diet)) }\OperatorTok{+}\StringTok{ }
\StringTok{    }\KeywordTok{geom_line}\NormalTok{(}\KeywordTok{aes}\NormalTok{(}\DataTypeTok{color =}\NormalTok{ Diet, }\DataTypeTok{group =}\NormalTok{ Chick))}
\end{Highlighting}
\end{Shaded}

\begin{figure}

{\centering \includegraphics[width=0.8\linewidth]{Libro_files/figure-latex/Pollos-1} 

}

\caption{Gráfico en el cual vemos el peso de pollos en el tiempo, con colores distintos según el tipo de dieta y con líneas para cada pollo individual.}\label{fig:Pollos}
\end{figure}

Para aclarar este enredo, es mejor el generar un gráfico para cada
dieta, y es ahí donde entra la función \texttt{facet\_wrap}. Esta
función nos permite generar el gráfico deseado al agregar como argumento
dentro de la función el simbolo \texttt{\textasciitilde{}} seguido del
nombre de la variable a utilizar para separar los gráficos, tal como en
la figura \ref{fig:MasPollos} y su código correspondiente.

\begin{Shaded}
\begin{Highlighting}[]
\KeywordTok{ggplot}\NormalTok{(ChickWeight, }\KeywordTok{aes}\NormalTok{(}\DataTypeTok{x =}\NormalTok{ Time, }\DataTypeTok{y =}\NormalTok{ weight)) }\OperatorTok{+}\StringTok{ }\KeywordTok{geom_point}\NormalTok{(}\KeywordTok{aes}\NormalTok{(}\DataTypeTok{color =}\NormalTok{ Diet)) }\OperatorTok{+}\StringTok{ }
\StringTok{    }\KeywordTok{geom_line}\NormalTok{(}\KeywordTok{aes}\NormalTok{(}\DataTypeTok{color =}\NormalTok{ Diet, }\DataTypeTok{group =}\NormalTok{ Chick)) }\OperatorTok{+}\StringTok{ }\KeywordTok{facet_wrap}\NormalTok{(}\OperatorTok{~}\NormalTok{Diet)}
\end{Highlighting}
\end{Shaded}

\begin{figure}

{\centering \includegraphics[width=0.8\linewidth]{Libro_files/figure-latex/MasPollos-1} 

}

\caption{Gráfico en el cual vemos el peso de pollos en el tiempo, con colores y gráficos distintos según el tipo de dieta y con líneas para cada pollo individual.}\label{fig:MasPollos}
\end{figure}

Esta función siempre tendrá los mismos ejes y escala para todos los
gráficos. Además, intentará siempre dejar la disposición de los gráficos
de la forma más cuadrada posible, pero esto puede ser modificado
agregando el argumento \emph{ncol} y un número de columnas, así como
vemos en la figura \ref{fig:Columnas} y su código correspondiente.

\begin{Shaded}
\begin{Highlighting}[]
\KeywordTok{ggplot}\NormalTok{(ChickWeight, }\KeywordTok{aes}\NormalTok{(}\DataTypeTok{x =}\NormalTok{ Time, }\DataTypeTok{y =}\NormalTok{ weight)) }\OperatorTok{+}\StringTok{ }\KeywordTok{geom_point}\NormalTok{(}\KeywordTok{aes}\NormalTok{(}\DataTypeTok{color =}\NormalTok{ Diet)) }\OperatorTok{+}\StringTok{ }
\StringTok{    }\KeywordTok{geom_line}\NormalTok{(}\KeywordTok{aes}\NormalTok{(}\DataTypeTok{color =}\NormalTok{ Diet, }\DataTypeTok{group =}\NormalTok{ Chick)) }\OperatorTok{+}\StringTok{ }\KeywordTok{facet_wrap}\NormalTok{(}\OperatorTok{~}\NormalTok{Diet, }
    \DataTypeTok{ncol =} \DecValTok{3}\NormalTok{)}
\end{Highlighting}
\end{Shaded}

\begin{figure}

{\centering \includegraphics[width=0.8\linewidth]{Libro_files/figure-latex/Columnas-1} 

}

\caption{Gráfico en el cual vemos el peso de pollos en el tiempo, con colores y gráficos distintos según el tipo de dieta y con líneas para cada pollo individual.}\label{fig:Columnas}
\end{figure}

\hypertarget{modelos}{%
\chapter{Modelos en R}\label{modelos}}

\hypertarget{paquetes-necesarios-para-este-capitulo-4}{%
\section{Paquetes necesarios para este
capítulo}\label{paquetes-necesarios-para-este-capitulo-4}}

Para este capítulo necesitas tener instalado el paquete
\emph{tidyverse}, \emph{broom} y \emph{MuMIn}.

En este capítulo se explicará como generar modelos en R, el como obtener
información y tablas a partir de los modelos con el paquete \emph{Broom}
\citep{Robinson2018} y una leve introducción a la selección de modelos
con el paquete \emph{MuMIn} \citep{Barton2018}

Dado que este libro es un apoyo para el curso BIO4022, esta clase puede
también ser seguida en este
\href{https://derek-corcoran-barrios.github.io/Clase5/Modelos}{link}. El
video de la clase se encontrará disponible en este
\href{https://www.youtube.com/watch?v=rsRPVwd1_8k\&feature=youtu.be}{link}.

\hypertarget{modelos-estadisticos}{%
\section{Modelos estadísticos}\label{modelos-estadisticos}}

Un modelo estadístico intenta explicar las causas de un suceso basado en
un muestreo de la población total. El supuesto es que si la muestra que
obtenemos de la población es representativa de esta, podremos inferir
las causas de la variación de la población midiendo variables
explicativas. En general tenemos una variable respuesta (fenómeno que
queremos explicar), y una o varias variables explicativas que generarían
deterministamente parte de la variabilidad en la variable respuesta.

\hypertarget{ejemplo}{%
\subsection{Ejemplo}\label{ejemplo}}

Tomemos el ejemplo de la base de datos \emph{CO2} presente en R
\citep{potvin1990statistical}. Supongamos que nos interesa saber que
factores afectan la captación de \(CO_2\) en las plantas.

\label{tab:TablaCo2}Primeras 20 variables de la base de datos CO2.

Plant

Type

Treatment

conc

uptake

Qn1

Quebec

nonchilled

95

16.0

Qn1

Quebec

nonchilled

175

30.4

Qn1

Quebec

nonchilled

250

34.8

Qn1

Quebec

nonchilled

350

37.2

Qn1

Quebec

nonchilled

500

35.3

Qn1

Quebec

nonchilled

675

39.2

Qn1

Quebec

nonchilled

1000

39.7

Qn2

Quebec

nonchilled

95

13.6

Qn2

Quebec

nonchilled

175

27.3

Qn2

Quebec

nonchilled

250

37.1

Qn2

Quebec

nonchilled

350

41.8

Qn2

Quebec

nonchilled

500

40.6

Qn2

Quebec

nonchilled

675

41.4

Qn2

Quebec

nonchilled

1000

44.3

Qn3

Quebec

nonchilled

95

16.2

Qn3

Quebec

nonchilled

175

32.4

Qn3

Quebec

nonchilled

250

40.3

Qn3

Quebec

nonchilled

350

42.1

Qn3

Quebec

nonchilled

500

42.9

Qn3

Quebec

nonchilled

675

43.9

En la tabla \ref{tab:TablaCo2} vemos las primeras 20 observaciones de
esta base de datos. Vemos que dentro de los factores que tenemos para
explicar la captación de \(CO_2\) estan:

\begin{itemize}
\tightlist
\item
  \emph{Type:} Subespecie de la planta (Missisipi o Quebec)
\item
  \emph{Treatment:} Tratamiento de la plnata, enfriado (chilled) o no
  enfriado (nonchilled)
\item
  \emph{conc:} Concentración ambiental de \(CO_2\), en mL/L.
\end{itemize}

Una posible explicación que nos permitiría intentar explicar este
fenómeno, es que las plantas de distintas subespecies, tendrán distinta
captación de \(CO_2\), lo cual exlploramos en el gráfico
\ref{fig:Subespecie}:

\begin{figure}

{\centering \includegraphics[width=0.8\linewidth]{Libro_files/figure-latex/Subespecie-1} 

}

\caption{Captación de CO2 por plantas dependiente de su subespecie}\label{fig:Subespecie}
\end{figure}

Vemos que se observa una tendencia a que las plantas con origen en
Quebec capten más \(CO_2\) que las que estan en el Mississippi, pero
¿Podemos decir efectivamente que ambas poblaciónes tienen medias
distintas medias? Es ahí donde entran los modelos.

\hypertarget{representando-un-modelo-en-r}{%
\subsection{Representando un modelo en
R}\label{representando-un-modelo-en-r}}

En R la mayoría de los modelos se representan con el siguiente codigo:

\begin{Shaded}
\begin{Highlighting}[]
\KeywordTok{alguna_funcion}\NormalTok{(Y }\OperatorTok{~}\StringTok{ }\NormalTok{X1 }\OperatorTok{+}\StringTok{ }\NormalTok{X2 }\OperatorTok{+}\StringTok{ }\NormalTok{... }\OperatorTok{+}\StringTok{ }\NormalTok{Xn, }\DataTypeTok{data =}\NormalTok{ data.frame)}
\end{Highlighting}
\end{Shaded}

En este modelo, tenemos la variable respuesta \emph{Y}, la cual puede
estar explcada por una o multiples variables explicativas \emph{X}, es
por esto que el simbolo \texttt{\textasciitilde{}} se lee explicado por,
donde lo que esta a su izquerada es la variable respuesta y a la derecha
la variable explicativa. Los datos se encuentran en un data frame y
finalmente usaremos alguna función, que identificará algún modelo.
Algunas de estas funciones las encontramos en la tabla \ref{tab:Modelos}

\label{tab:Modelos}Algunos modelos que podemos generar en R

Modelos

Funcion

Prueba de t

t.test()

ANOVA

aov()

Modelo lineal simple

lm()

modelo lineal generalizado

glm()

Modelo aditivo

gam()

Modelo no lineal

nls()

modelos lineales mixtos

lmer()

Boosted regression trees

gbm()

\hypertarget{volvamos-al-ejemplo-de-las-plantas}{%
\subsection{Volvamos al ejemplo de las
plantas}\label{volvamos-al-ejemplo-de-las-plantas}}

Para este ejemplo usaremos un modelo lineal simple, para esto siguiendo
la tabla \ref{tab:Modelos} usaremos la función \texttt{lm}:

\begin{Shaded}
\begin{Highlighting}[]
\NormalTok{Fit1 <-}\StringTok{ }\KeywordTok{lm}\NormalTok{(uptake }\OperatorTok{~}\StringTok{ }\NormalTok{Type, }\DataTypeTok{data =}\NormalTok{ CO2)}
\end{Highlighting}
\end{Shaded}

\hypertarget{usando-broom-para-sacarle-mas-a-tu-modelo}{%
\subsubsection{usando broom para sacarle más a tu
modelo}\label{usando-broom-para-sacarle-mas-a-tu-modelo}}

El paquete broom \citep{Robinson2018} es un paquete adyacente al
tidyverse (por lo que debes cargarlo aparte del tidyverse), el cual nos
permite tomar información de modelos generados en formato tidy. Hoy
veremos 3 funciones de \emph{broom}, estas son \texttt{glance},
\texttt{tidy} y \texttt{augment}.

\hypertarget{glance}{%
\paragraph{glance}\label{glance}}

la función glance, nos entregará información general del modelo, como el
valor de p, el \(R^2\), log-likelihood, grados de libertad, y/o otros
parametros dependiendo del modelo a utilizar. Esta información nos es
entregada en un formato de data frame, como vemos en el código siguiente
y en la tabla \ref{tab:glance}

\begin{Shaded}
\begin{Highlighting}[]
\KeywordTok{library}\NormalTok{(broom)}
\KeywordTok{glance}\NormalTok{(Fit1)}
\end{Highlighting}
\end{Shaded}

\label{tab:glance}Información del modelo fi1 entregada por la función glance

r.squared

adj.r.squared

sigma

statistic

p.value

df

logLik

AIC

BIC

deviance

df.residual

0.346713

0.3387461

8.794012

43.5191

0

2

-300.8007

607.6014

614.8939

6341.441

82

\hypertarget{tidy}{%
\paragraph{tidy}\label{tidy}}

la función tidy, nos entregará información sobre los parametros del
modelo, esto es el intercepto, la pendiente y/o interacciones, como
vemos en el código siguiente y en la tabla \ref{tab:tidy}

\begin{Shaded}
\begin{Highlighting}[]
\KeywordTok{tidy}\NormalTok{(Fit1)}
\end{Highlighting}
\end{Shaded}

\label{tab:tidy}Información del modelo fi1 entregada por la función glance

term

estimate

std.error

statistic

p.value

(Intercept)

33.54286

1.356945

24.719384

0

TypeMississippi

-12.65952

1.919011

-6.596901

0

\hypertarget{augment}{%
\paragraph{augment}\label{augment}}

la función augment, nos entregará para cada observación de nuestro
modelo, varios parametros importantes como el valor predicho, los
residuales, el distancia de cook entre otros, esto nos sirve
principalmente para estudiar los supuestos de nuestro modelo. A
continuación vemos el uso de la función \texttt{augment} y 20 de sus
observaciones en la tabla \ref{tab:augment}

\begin{Shaded}
\begin{Highlighting}[]
\KeywordTok{augment}\NormalTok{(Fit1)}
\end{Highlighting}
\end{Shaded}

\label{tab:augment}Información del modelo fi1 entregada por la función
augment

uptake

Type

.fitted

.se.fit

.resid

.hat

.sigma

.cooksd

.std.resid

10.6

Mississippi

20.88333

1.356945

-10.283333

0.0238095

8.772231

0.0170823

-1.1835308

28.5

Mississippi

20.88333

1.356945

7.616667

0.0238095

8.806572

0.0093715

0.8766185

10.6

Mississippi

20.88333

1.356945

-10.283333

0.0238095

8.772231

0.0170823

-1.1835308

32.4

Quebec

33.54286

1.356945

-1.142857

0.0238095

8.847196

0.0002110

-0.1315339

27.9

Mississippi

20.88333

1.356945

7.016667

0.0238095

8.812874

0.0079531

0.8075632

17.9

Mississippi

20.88333

1.356945

-2.983333

0.0238095

8.841767

0.0014377

-0.3433582

35.3

Quebec

33.54286

1.356945

1.757143

0.0238095

8.845923

0.0004988

0.2022333

19.2

Mississippi

20.88333

1.356945

-1.683333

0.0238095

8.846104

0.0004577

-0.1937384

38.1

Quebec

33.54286

1.356945

4.557143

0.0238095

8.833275

0.0033548

0.5244913

14.9

Mississippi

20.88333

1.356945

-5.983333

0.0238095

8.822508

0.0057831

-0.6886346

38.6

Quebec

33.54286

1.356945

5.057143

0.0238095

8.829833

0.0041313

0.5820374

30.3

Quebec

33.54286

1.356945

-3.242857

0.0238095

8.840611

0.0016988

-0.3732274

31.8

Mississippi

20.88333

1.356945

10.916667

0.0238095

8.762547

0.0192512

1.2564225

13.7

Mississippi

20.88333

1.356945

-7.183333

0.0238095

8.811176

0.0083355

-0.8267452

16.0

Quebec

33.54286

1.356945

-17.542857

0.0238095

8.625388

0.0497139

-2.0190449

27.3

Quebec

33.54286

1.356945

-6.242857

0.0238095

8.820233

0.0062957

-0.7185038

12.0

Mississippi

20.88333

1.356945

-8.883333

0.0238095

8.791552

0.0127476

-1.0224018

41.4

Quebec

33.54286

1.356945

7.857143

0.0238095

8.803900

0.0099726

0.9042954

13.6

Quebec

33.54286

1.356945

-19.942857

0.0238095

8.559179

0.0642469

-2.2952661

7.7

Mississippi

20.88333

1.356945

-13.183333

0.0238095

8.723037

0.0280755

-1.5172980

\hypertarget{seleccion-de-modelos-usando-broom-y-el-aic}{%
\subsubsection{Selección de modelos usando broom y el
AIC}\label{seleccion-de-modelos-usando-broom-y-el-aic}}

El AIC, o Criterio de informacion de Akaike \citep{aho2014model}, es una
medida de cuanta información nos entrega un modelo dada su complejidad.
Esta última medida a partir del número de parámetros que tiene. Cuanto
más bajo sea el AIC, mejor comparativamente es un modelo, y en general,
un modelo que sea dos unidades de AIC menor que otro modelo, será
considerado un modelo que es significativamente mejor que otro.

La formula del criterio de selección de Akaike es la que vemos en la
ecuación \eqref{eq:AIC}.

\begin{equation} 
  AIC = 2 K - 2 \ln{(\hat{L})}
  \label{eq:AIC}
\end{equation}

Donde \(K\) es el número de parametros, lo cual podemos ver con tidy, si
vemos la tabla \ref{tab:tidy}, vemos que el modelo \emph{Fit1} tiene 2
parametros, esto es \(K\) es igual a 2.

El log-likelihood del modelo (\(\ln{(\hat{L})}\)) es el ajuste que este
tiene a los datos. Cuanto más positivo es este valor mejor se ajusta el
modelo a los datos, y cuanto mas negativo es, menos se ajusta a los
datos, en nuestro modelo, usando glance, podemos ver que el valor del
log-likelyhood del modelo es de -300.8 (ver tabla \ref{tab:tidy}).

Por lo tanto remplazando la ecuación \eqref{eq:AIC}, obtenemos 605.6, que
es un valor muy cercano a los 608, que aparecen en el glance del modelo
(tabla \ref{tab:tidy}).

\hypertarget{modelos-candidatos}{%
\paragraph{Modelos candidatos}\label{modelos-candidatos}}

Veamos la figura \ref{fig:CO2Mods}. para pensar cuales podrían ser
modelos interesantes a explorar.

\begin{Shaded}
\begin{Highlighting}[]
\KeywordTok{ggplot}\NormalTok{(CO2, }\KeywordTok{aes}\NormalTok{(}\DataTypeTok{x =}\NormalTok{ conc, }\DataTypeTok{y =}\NormalTok{ uptake)) }\OperatorTok{+}\StringTok{ }\KeywordTok{geom_point}\NormalTok{(}\KeywordTok{aes}\NormalTok{(}\DataTypeTok{color =}\NormalTok{ Type, }
    \DataTypeTok{shape =}\NormalTok{ Treatment), }\DataTypeTok{size =} \DecValTok{3}\NormalTok{)}
\end{Highlighting}
\end{Shaded}

\begin{figure}

{\centering \includegraphics[width=0.8\linewidth]{Libro_files/figure-latex/CO2Mods-1} 

}

\caption{Gráfico exploratorio para generar modelos de la base de datos CO2}\label{fig:CO2Mods}
\end{figure}

\hypertarget{loops}{%
\chapter{Loops (purrr) y bibliografía (rticles)}\label{loops}}

\hypertarget{paquetes-necesarios-para-este-capitulo-5}{%
\section{Paquetes necesarios para este
capítulo}\label{paquetes-necesarios-para-este-capitulo-5}}

Para este capítulo necesitas tener instalado el paquete
\emph{tidyverse}.

Probablemente uno de los puntos que marca la diferencia entre ser un
usuario de un lenguaje de programación y un alguién que realmente
programa. Es el momento en que una persona aprende a hacer loops. Los
loops son una acción repetitiva en la cual una misma acción es realizada
por el computador ahorrandonos mucho tiempo de escribir código y muchas
veces tiempo de computación tambien.

Existen varias formas de como realizar loops en R, los \emph{for} loops,
la familia de los \emph{apply} y más recientemente el uso del paquete
\emph{purrr} \citep{HenryPurrr} presente en el \emph{tidyverse}. En este
capítulo nos enfocaremos principalmente en el uso de este paquete, pero
también explicaremos levemente el caso de los for loops.

Dado que este libro es un apoyo para el curso BIO4022, esta clase puede
también ser seguida en este
\href{https://derek-corcoran-barrios.github.io/Clase6/Clase6Loopsybibliografia}{link}.
El video de la clase se encontrará disponible en este
\href{https://youtu.be/Oz_egH-sXZg}{link}.

\hypertarget{generando-una-receta}{%
\section{Generando una receta}\label{generando-una-receta}}

Como hacer un loop, es una repetición de un código multiples veces,
generalmente lo que más nos combiene es generar la receta tomando en
cuenta el primer elemento y luego repetirlo en un loop.

\hypertarget{dioxido-de-nitrogeno-en-madrid}{%
\subsection{Dioxido de nitrógeno en
Madrid}\label{dioxido-de-nitrogeno-en-madrid}}

Supongamos que queremos estudiar la concentración de dióxido de
Nitrógeno en madrid en distintas estaciones, la base de datos puede ser
encontrada en el siguiente
\href{https://www.kaggle.com/decide-soluciones/air-quality-madrid}{link}.
Dentro de esta base de datos tenemos una carpeta con la calidad de aire
de estaciones en Madrid, con un archivo para cada año. Supongamos que se
quiere hacer lo siguiente, limitandose a las estaciones de Cuatro
Caminos, El Pardo, Escuelas Aguirre, Moratalaz y Tres Olivos, calcular
los promedios de \(NO_2\) para cada mes y cada año en estas estaciones.

\hypertarget{generando-la-receta}{%
\subsubsection{Generando la receta}\label{generando-la-receta}}

Esto lo podemos hacer con un loop, pero antes generemos \emph{la receta}
tomando en cuenta solo el 2017.

Para esto hacemos lo siguiente:

\begin{itemize}
\tightlist
\item
  Tomemos la base de datos de calidad de aire de Madrid
\item
  Leeamos el año 2017
\item
  Limitemonos a las estaciones de Cuatro Caminos, El Pardo, Escuelas
  Aguirre, Moratalaz y Tres Olivos
\item
  Agreguemos una columna con el año y una con el mes
\item
  Calculemos los promedios de \(NO_2\) para cada mes
\item
  Eliminemos las columnas innecesarias para estudiar el efecto del
  \(NO_2\) en Madrid
\end{itemize}

Vamos paso a paso

\hypertarget{leyendo-la-base-de-datos}{%
\paragraph{leyendo la base de datos}\label{leyendo-la-base-de-datos}}

El primer paso es leer la base de datos, para esto usamos el
\emph{tidyverse} y cargamos además \emph{lubridate} por si tenemos que
trabajar con las fechas. En la tabla \ref{tab:Madrid2017a} vemos los
resultados del código a continuación.

\begin{Shaded}
\begin{Highlighting}[]
\KeywordTok{library}\NormalTok{(tidyverse)}
\KeywordTok{library}\NormalTok{(lubridate)}
\NormalTok{Madrid2017 <-}\StringTok{ }\KeywordTok{read_csv}\NormalTok{(}\StringTok{"csvs_per_year/madrid_2017.csv"}\NormalTok{)}
\end{Highlighting}
\end{Shaded}

\label{tab:Madrid2017a}Los primeros 20 datos de calidad de aire del 2017 en
Madrid para todas las estaciones.

date

BEN

CH4

CO

EBE

NMHC

NO

NO\_2

NOx

O\_3

PM10

PM25

SO\_2

TCH

TOL

station

2017-06-01 01:00:00

NA

NA

0.3

NA

NA

4

38

NA

NA

NA

NA

5

NA

NA

28079004

2017-06-01 01:00:00

0.6

NA

0.3

0.4

0.08

3

39

NA

71

22

9

7

1.40

2.9

28079008

2017-06-01 01:00:00

0.2

NA

NA

0.1

NA

1

14

NA

NA

NA

NA

NA

NA

0.9

28079011

2017-06-01 01:00:00

NA

NA

0.2

NA

NA

1

9

NA

91

NA

NA

NA

NA

NA

28079016

2017-06-01 01:00:00

NA

NA

NA

NA

NA

1

19

NA

69

NA

NA

2

NA

NA

28079017

2017-06-01 01:00:00

0.1

NA

0.3

0.2

NA

1

26

NA

70

26

NA

1

NA

0.3

28079018

2017-06-01 01:00:00

0.3

NA

0.2

0.1

0.17

1

19

NA

79

23

9

3

0.86

1.8

28079024

2017-06-01 01:00:00

NA

NA

NA

NA

NA

1

9

NA

87

NA

NA

NA

NA

NA

28079027

2017-06-01 01:00:00

NA

NA

0.3

NA

NA

3

30

NA

70

NA

NA

NA

NA

NA

28079035

2017-06-01 01:00:00

NA

NA

0.1

NA

NA

1

15

NA

NA

22

NA

10

NA

NA

28079036

2017-06-01 01:00:00

0.7

NA

NA

1.0

NA

1

25

NA

NA

21

10

2

NA

3.5

28079038

2017-06-01 01:00:00

NA

NA

0.2

NA

NA

1

21

NA

75

NA

NA

NA

NA

NA

28079039

2017-06-01 01:00:00

NA

NA

NA

NA

NA

2

17

NA

NA

24

NA

9

NA

NA

28079040

2017-06-01 01:00:00

NA

NA

NA

NA

NA

1

22

NA

NA

23

15

NA

NA

NA

28079047

2017-06-01 01:00:00

NA

NA

NA

NA

NA

2

30

NA

NA

17

9

NA

NA

NA

28079048

2017-06-01 01:00:00

NA

NA

NA

NA

NA

1

12

NA

74

NA

NA

NA

NA

NA

28079049

2017-06-01 01:00:00

NA

NA

NA

NA

NA

2

15

NA

NA

16

12

NA

NA

NA

28079050

2017-06-01 01:00:00

NA

NA

NA

NA

NA

3

12

NA

84

NA

NA

NA

NA

NA

28079054

2017-06-01 01:00:00

0.2

NA

NA

0.6

0.08

1

12

NA

NA

15

NA

NA

1.16

1.5

28079055

2017-06-01 01:00:00

NA

NA

0.1

NA

NA

9

47

NA

59

NA

NA

NA

NA

NA

28079056

\hypertarget{limitemonos-a-las-estaciones-seleccionadas}{%
\paragraph{Limitemonos a las estaciones
seleccionadas}\label{limitemonos-a-las-estaciones-seleccionadas}}

Revisando el archivo \emph{stations.csv}, podemos ver que el código de
estaciones que estudiaremos son 28079036, 28079008,28079058, 28079060 y
28079038, por lo que lo ponemos en un filter. El resultado de esto lo
podemos ver en la tabla \ref{tab:Madrid2017b}

\begin{Shaded}
\begin{Highlighting}[]
\NormalTok{Madrid2017 <-}\StringTok{ }\KeywordTok{read_csv}\NormalTok{(}\StringTok{"csvs_per_year/madrid_2017.csv"}\NormalTok{) }\OperatorTok\StringTok{ }\KeywordTok{filter}\NormalTok{(station }\OperatorTok\StringTok{ }
\StringTok{    }\KeywordTok{c}\NormalTok{(}\DecValTok{28079036}\NormalTok{, }\DecValTok{28079008}\NormalTok{, }\DecValTok{28079058}\NormalTok{, }\DecValTok{28079060}\NormalTok{, }\DecValTok{28079038}\NormalTok{))}
\end{Highlighting}
\end{Shaded}

\label{tab:Madrid2017b}Los primeros 20 datos de calidad de aire del 2017 en
Madrid después de filtrar según estación.

date

BEN

CH4

CO

EBE

NMHC

NO

NO\_2

NOx

O\_3

PM10

PM25

SO\_2

TCH

TOL

station

2017-06-01 01:00:00

0.6

NA

0.3

0.4

0.08

3

39

NA

71

22

9

7

1.40

2.9

28079008

2017-06-01 01:00:00

NA

NA

0.1

NA

NA

1

15

NA

NA

22

NA

10

NA

NA

28079036

2017-06-01 01:00:00

0.7

NA

NA

1.0

NA

1

25

NA

NA

21

10

2

NA

3.5

28079038

2017-06-01 01:00:00

NA

NA

NA

NA

NA

1

10

NA

66

NA

NA

NA

NA

NA

28079058

2017-06-01 01:00:00

NA

NA

NA

NA

NA

1

26

NA

79

86

NA

NA

NA

NA

28079060

2017-06-01 02:00:00

0.3

NA

0.3

0.2

0.07

2

27

NA

72

16

7

7

1.40

2.3

28079008

2017-06-01 02:00:00

NA

NA

0.1

NA

NA

1

13

NA

NA

17

NA

10

NA

NA

28079036

2017-06-01 02:00:00

0.2

NA

NA

0.5

NA

9

20

NA

NA

13

4

2

NA

1.3

28079038

2017-06-01 02:00:00

NA

NA

NA

NA

NA

1

11

NA

68

NA

NA

NA

NA

NA

28079058

2017-06-01 02:00:00

NA

NA

NA

NA

NA

1

9

NA

90

62

NA

NA

NA

NA

28079060

2017-06-01 03:00:00

0.3

NA

0.3

0.2

0.08

2

25

NA

73

14

7

7

1.40

2.0

28079008

2017-06-01 03:00:00

NA

NA

0.1

NA

NA

1

11

NA

NA

18

NA

10

NA

NA

28079036

2017-06-01 03:00:00

0.1

NA

NA

0.4

NA

6

20

NA

NA

11

6

2

NA

1.8

28079038

2017-06-01 03:00:00

NA

NA

NA

NA

NA

1

6

NA

68

NA

NA

NA

NA

NA

28079058

2017-06-01 03:00:00

NA

NA

NA

NA

NA

1

8

NA

86

19

NA

NA

NA

NA

28079060

2017-06-01 04:00:00

0.3

NA

0.2

0.2

0.08

2

22

NA

75

15

10

6

1.41

1.4

28079008

2017-06-01 04:00:00

NA

NA

0.1

NA

NA

1

14

NA

NA

12

NA

10

NA

NA

28079036

2017-06-01 04:00:00

0.2

NA

NA

0.5

NA

1

12

NA

NA

10

6

2

NA

1.7

28079038

2017-06-01 04:00:00

NA

NA

NA

NA

NA

1

6

NA

60

NA

NA

NA

NA

NA

28079058

2017-06-01 04:00:00

NA

NA

NA

NA

NA

1

11

NA

75

8

NA

NA

NA

NA

28079060

\hypertarget{agreguemos-aparte-el-mes-el-ano-y-el-nombre-de-la-estacion}{%
\paragraph{Agreguemos aparte el mes, el año y el nombre de la
estación}\label{agreguemos-aparte-el-mes-el-ano-y-el-nombre-de-la-estacion}}

Usando \texttt{mutate} y las funciones \texttt{month}y \texttt{year} de
lubridate podemos agregar el més y el año para cada observación, además
usando \texttt{left\_joint}, podemos agreagar el nombre de las
estaciones usando la base de datos \emph{stations.csv}. El resultado de
esto lo podemos ver en la tabla \ref{tab:Madrid2017c}

\begin{Shaded}
\begin{Highlighting}[]
\NormalTok{stations <-}\StringTok{ }\KeywordTok{read_csv}\NormalTok{(}\StringTok{"stations.csv"}\NormalTok{) }\OperatorTok\StringTok{ }\KeywordTok{rename}\NormalTok{(}\DataTypeTok{station =}\NormalTok{ id)}
\NormalTok{Madrid2017 <-}\StringTok{ }\KeywordTok{read_csv}\NormalTok{(}\StringTok{"csvs_per_year/madrid_2017.csv"}\NormalTok{) }\OperatorTok\StringTok{ }\KeywordTok{filter}\NormalTok{(station }\OperatorTok\StringTok{ }
\StringTok{    }\KeywordTok{c}\NormalTok{(}\DecValTok{28079036}\NormalTok{, }\DecValTok{28079008}\NormalTok{, }\DecValTok{28079058}\NormalTok{, }\DecValTok{28079060}\NormalTok{, }\DecValTok{28079038}\NormalTok{)) }\OperatorTok\StringTok{ }
\StringTok{    }\KeywordTok{mutate}\NormalTok{(}\DataTypeTok{month =} \KeywordTok{month}\NormalTok{(date), }\DataTypeTok{year =} \KeywordTok{year}\NormalTok{(date)) }\OperatorTok\StringTok{ }\KeywordTok{left_join}\NormalTok{(stations)}
\end{Highlighting}
\end{Shaded}

\label{tab:Madrid2017c}Los primeros 20 datos de calidad de aire del 2017 en
Madrid después de filtrar según estación, con mes, año y nombre.

date

BEN

CH4

CO

EBE

NMHC

NO

NO\_2

NOx

O\_3

PM10

PM25

SO\_2

TCH

TOL

station

month

year

name

address

lon

lat

elevation

2017-06-01 01:00:00

0.6

NA

0.3

0.4

0.08

3

39

NA

71

22

9

7

1.40

2.9

28079008

6

2017

Escuelas Aguirre

Entre C/ Alcalá y C/ O' Donell

-3.682319

40.42156

670

2017-06-01 01:00:00

NA

NA

0.1

NA

NA

1

15

NA

NA

22

NA

10

NA

NA

28079036

6

2017

Moratalaz

Avd. Moratalaz esq. Camino de los Vinateros

-3.645306

40.40795

685

2017-06-01 01:00:00

0.7

NA

NA

1.0

NA

1

25

NA

NA

21

10

2

NA

3.5

28079038

6

2017

Cuatro Caminos

Avda. Pablo Iglesias esq. C/ Marqués de Lema

-3.707128

40.44554

698

2017-06-01 01:00:00

NA

NA

NA

NA

NA

1

10

NA

66

NA

NA

NA

NA

NA

28079058

6

2017

El Pardo

Avda. La Guardia

-3.774611

40.51806

615

2017-06-01 01:00:00

NA

NA

NA

NA

NA

1

26

NA

79

86

NA

NA

NA

NA

28079060

6

2017

Tres Olivos

Plaza Tres Olivos

-3.689761

40.50059

715

2017-06-01 02:00:00

0.3

NA

0.3

0.2

0.07

2

27

NA

72

16

7

7

1.40

2.3

28079008

6

2017

Escuelas Aguirre

Entre C/ Alcalá y C/ O' Donell

-3.682319

40.42156

670

2017-06-01 02:00:00

NA

NA

0.1

NA

NA

1

13

NA

NA

17

NA

10

NA

NA

28079036

6

2017

Moratalaz

Avd. Moratalaz esq. Camino de los Vinateros

-3.645306

40.40795

685

2017-06-01 02:00:00

0.2

NA

NA

0.5

NA

9

20

NA

NA

13

4

2

NA

1.3

28079038

6

2017

Cuatro Caminos

Avda. Pablo Iglesias esq. C/ Marqués de Lema

-3.707128

40.44554

698

2017-06-01 02:00:00

NA

NA

NA

NA

NA

1

11

NA

68

NA

NA

NA

NA

NA

28079058

6

2017

El Pardo

Avda. La Guardia

-3.774611

40.51806

615

2017-06-01 02:00:00

NA

NA

NA

NA

NA

1

9

NA

90

62

NA

NA

NA

NA

28079060

6

2017

Tres Olivos

Plaza Tres Olivos

-3.689761

40.50059

715

2017-06-01 03:00:00

0.3

NA

0.3

0.2

0.08

2

25

NA

73

14

7

7

1.40

2.0

28079008

6

2017

Escuelas Aguirre

Entre C/ Alcalá y C/ O' Donell

-3.682319

40.42156

670

2017-06-01 03:00:00

NA

NA

0.1

NA

NA

1

11

NA

NA

18

NA

10

NA

NA

28079036

6

2017

Moratalaz

Avd. Moratalaz esq. Camino de los Vinateros

-3.645306

40.40795

685

2017-06-01 03:00:00

0.1

NA

NA

0.4

NA

6

20

NA

NA

11

6

2

NA

1.8

28079038

6

2017

Cuatro Caminos

Avda. Pablo Iglesias esq. C/ Marqués de Lema

-3.707128

40.44554

698

2017-06-01 03:00:00

NA

NA

NA

NA

NA

1

6

NA

68

NA

NA

NA

NA

NA

28079058

6

2017

El Pardo

Avda. La Guardia

-3.774611

40.51806

615

2017-06-01 03:00:00

NA

NA

NA

NA

NA

1

8

NA

86

19

NA

NA

NA

NA

28079060

6

2017

Tres Olivos

Plaza Tres Olivos

-3.689761

40.50059

715

2017-06-01 04:00:00

0.3

NA

0.2

0.2

0.08

2

22

NA

75

15

10

6

1.41

1.4

28079008

6

2017

Escuelas Aguirre

Entre C/ Alcalá y C/ O' Donell

-3.682319

40.42156

670

2017-06-01 04:00:00

NA

NA

0.1

NA

NA

1

14

NA

NA

12

NA

10

NA

NA

28079036

6

2017

Moratalaz

Avd. Moratalaz esq. Camino de los Vinateros

-3.645306

40.40795

685

2017-06-01 04:00:00

0.2

NA

NA

0.5

NA

1

12

NA

NA

10

6

2

NA

1.7

28079038

6

2017

Cuatro Caminos

Avda. Pablo Iglesias esq. C/ Marqués de Lema

-3.707128

40.44554

698

2017-06-01 04:00:00

NA

NA

NA

NA

NA

1

6

NA

60

NA

NA

NA

NA

NA

28079058

6

2017

El Pardo

Avda. La Guardia

-3.774611

40.51806

615

2017-06-01 04:00:00

NA

NA

NA

NA

NA

1

11

NA

75

8

NA

NA

NA

NA

28079060

6

2017

Tres Olivos

Plaza Tres Olivos

-3.689761

40.50059

715

Finalmente, agrupamos sacamos el promedio por mes y sacamos las columnas
sobrantes al mismo tiempo, como vemos en la tabla \ref{tab:Madrid2017d}

\begin{Shaded}
\begin{Highlighting}[]
\KeywordTok{library}\NormalTok{(lubridate)}
\NormalTok{stations <-}\StringTok{ }\KeywordTok{read_csv}\NormalTok{(}\StringTok{"stations.csv"}\NormalTok{) }\OperatorTok\StringTok{ }\KeywordTok{rename}\NormalTok{(}\DataTypeTok{station =}\NormalTok{ id)}
\end{Highlighting}
\end{Shaded}

\begin{verbatim}
## Parsed with column specification:
## cols(
##   id = col_integer(),
##   name = col_character(),
##   address = col_character(),
##   lon = col_double(),
##   lat = col_double(),
##   elevation = col_integer()
## )
\end{verbatim}

\begin{Shaded}
\begin{Highlighting}[]
\NormalTok{Madrid2017 <-}\StringTok{ }\KeywordTok{read_csv}\NormalTok{(}\StringTok{"csvs_per_year/madrid_2017.csv"}\NormalTok{) }\OperatorTok\StringTok{ }\KeywordTok{filter}\NormalTok{(station }\OperatorTok\StringTok{ }
\StringTok{    }\KeywordTok{c}\NormalTok{(}\DecValTok{28079036}\NormalTok{, }\DecValTok{28079008}\NormalTok{, }\DecValTok{28079058}\NormalTok{, }\DecValTok{28079060}\NormalTok{, }\DecValTok{28079038}\NormalTok{)) }\OperatorTok\StringTok{ }
\StringTok{    }\KeywordTok{mutate}\NormalTok{(}\DataTypeTok{month =} \KeywordTok{month}\NormalTok{(date), }\DataTypeTok{year =} \KeywordTok{year}\NormalTok{(date)) }\OperatorTok\StringTok{ }\KeywordTok{left_join}\NormalTok{(stations) }\OperatorTok\StringTok{ }
\StringTok{    }\KeywordTok{group_by}\NormalTok{(month, name, year) }\OperatorTok\StringTok{ }\KeywordTok{summarise}\NormalTok{(}\DataTypeTok{NO_2 =} \KeywordTok{mean}\NormalTok{(NO_}\DecValTok{2}\NormalTok{, }
    \DataTypeTok{na.rm =} \OtherTok{TRUE}\NormalTok{))}
\end{Highlighting}
\end{Shaded}

\begin{verbatim}
## Parsed with column specification:
## cols(
##   date = col_datetime(format = ""),
##   BEN = col_double(),
##   CH4 = col_character(),
##   CO = col_double(),
##   EBE = col_double(),
##   NMHC = col_double(),
##   NO = col_double(),
##   NO_2 = col_double(),
##   NOx = col_character(),
##   O_3 = col_double(),
##   PM10 = col_double(),
##   PM25 = col_double(),
##   SO_2 = col_double(),
##   TCH = col_double(),
##   TOL = col_double(),
##   station = col_integer()
## )
\end{verbatim}

\begin{verbatim}
## Joining, by = "station"
\end{verbatim}

\label{tab:Madrid2017d}Los primeros 20 datos de calidad de aire del 2017 en
Madrid después de filtrar según estación, con mes, año y nombre.

month

name

year

NO\_2

1

Cuatro Caminos

2017

59.88124

1

Cuatro Caminos

2018

5.00000

1

El Pardo

2017

27.89892

1

El Pardo

2018

1.00000

1

Escuelas Aguirre

2017

69.43666

1

Escuelas Aguirre

2018

20.00000

1

Moratalaz

2017

54.17004

1

Moratalaz

2018

10.00000

1

Tres Olivos

2017

55.28706

1

Tres Olivos

2018

5.00000

2

Cuatro Caminos

2017

46.01045

2

El Pardo

2017

17.18955

2

Escuelas Aguirre

2017

59.89686

2

Moratalaz

2017

44.19581

2

Tres Olivos

2017

39.38209

3

Cuatro Caminos

2017

43.71833

3

El Pardo

2017

15.07962

3

Escuelas Aguirre

2017

61.74798

3

Moratalaz

2017

43.01353

3

Tres Olivos

2017

36.68329

\hypertarget{ultimos-detalles}{%
\paragraph{Últimos detalles}\label{ultimos-detalles}}

Vemos que hay algunos valores del 2018, esto parece raro, ya que leimos
los archivos del 2017. Al revisar mas con \texttt{summarize}, vemos que
en realidad son tan solo unas pocas observaciones las que generan esta
anomalía debido a algunas medidas del 1 de enero del 2018.

Para eliminarlas agregamos el siguiente código.

\begin{Shaded}
\begin{Highlighting}[]
\KeywordTok{library}\NormalTok{(lubridate)}
\NormalTok{stations <-}\StringTok{ }\KeywordTok{read_csv}\NormalTok{(}\StringTok{"stations.csv"}\NormalTok{) }\OperatorTok\StringTok{ }\KeywordTok{rename}\NormalTok{(}\DataTypeTok{station =}\NormalTok{ id)}
\end{Highlighting}
\end{Shaded}

\begin{verbatim}
## Parsed with column specification:
## cols(
##   id = col_integer(),
##   name = col_character(),
##   address = col_character(),
##   lon = col_double(),
##   lat = col_double(),
##   elevation = col_integer()
## )
\end{verbatim}

\begin{Shaded}
\begin{Highlighting}[]
\NormalTok{Madrid2017 <-}\StringTok{ }\KeywordTok{read_csv}\NormalTok{(}\StringTok{"csvs_per_year/madrid_2017.csv"}\NormalTok{) }\OperatorTok\StringTok{ }\KeywordTok{filter}\NormalTok{(station }\OperatorTok\StringTok{ }
\StringTok{    }\KeywordTok{c}\NormalTok{(}\DecValTok{28079036}\NormalTok{, }\DecValTok{28079008}\NormalTok{, }\DecValTok{28079058}\NormalTok{, }\DecValTok{28079060}\NormalTok{, }\DecValTok{28079038}\NormalTok{)) }\OperatorTok\StringTok{ }
\StringTok{    }\KeywordTok{mutate}\NormalTok{(}\DataTypeTok{month =} \KeywordTok{month}\NormalTok{(date), }\DataTypeTok{year =} \KeywordTok{year}\NormalTok{(date)) }\OperatorTok\StringTok{ }\KeywordTok{left_join}\NormalTok{(stations) }\OperatorTok\StringTok{ }
\StringTok{    }\KeywordTok{group_by}\NormalTok{(month, name, year) }\OperatorTok\StringTok{ }\KeywordTok{summarise}\NormalTok{(}\DataTypeTok{NO_2 =} \KeywordTok{mean}\NormalTok{(NO_}\DecValTok{2}\NormalTok{, }
    \DataTypeTok{na.rm =} \OtherTok{TRUE}\NormalTok{), }\DataTypeTok{n =} \KeywordTok{n}\NormalTok{()) }\OperatorTok\StringTok{ }\KeywordTok{filter}\NormalTok{(n }\OperatorTok{>}\StringTok{ }\DecValTok{500}\NormalTok{)}
\end{Highlighting}
\end{Shaded}

\begin{verbatim}
## Parsed with column specification:
## cols(
##   date = col_datetime(format = ""),
##   BEN = col_double(),
##   CH4 = col_character(),
##   CO = col_double(),
##   EBE = col_double(),
##   NMHC = col_double(),
##   NO = col_double(),
##   NO_2 = col_double(),
##   NOx = col_character(),
##   O_3 = col_double(),
##   PM10 = col_double(),
##   PM25 = col_double(),
##   SO_2 = col_double(),
##   TCH = col_double(),
##   TOL = col_double(),
##   station = col_integer()
## )
\end{verbatim}

\begin{verbatim}
## Joining, by = "station"
\end{verbatim}

Esto nos dá al fin la receta final que usaremos en el loop.

\hypertarget{empezando-el-loop}{%
\section{Empezando el loop}\label{empezando-el-loop}}

En este capíulo usaremos principalmente la función \texttt{map} del
paquete \emph{purrr} para generar loops, en esta función los dos
argumentos generales que necesitamos es un vector o lista
(\emph{argumento .x}) de los elementos que pasarán por una función, y
una funcion (\emph{argumento .f}) que se aplicará a toda esta lista. Es
importante establecer que el resultado de map siempre será una lista.

\hypertarget{volvamos-a-nuestra-receta}{%
\subsection{Volvamos a nuestra receta}\label{volvamos-a-nuestra-receta}}

Veamos el código que usamos para el año 2017

\begin{Shaded}
\begin{Highlighting}[]
\KeywordTok{library}\NormalTok{(lubridate)}
\NormalTok{stations <-}\StringTok{ }\KeywordTok{read_csv}\NormalTok{(}\StringTok{"stations.csv"}\NormalTok{) }\OperatorTok\StringTok{ }\KeywordTok{rename}\NormalTok{(}\DataTypeTok{station =}\NormalTok{ id)}
\NormalTok{Madrid2017 <-}\StringTok{ }\KeywordTok{read_csv}\NormalTok{(}\StringTok{"csvs_per_year/madrid_2017.csv"}\NormalTok{) }\OperatorTok\StringTok{ }\KeywordTok{filter}\NormalTok{(station }\OperatorTok\StringTok{ }
\StringTok{    }\KeywordTok{c}\NormalTok{(}\DecValTok{28079036}\NormalTok{, }\DecValTok{28079008}\NormalTok{, }\DecValTok{28079058}\NormalTok{, }\DecValTok{28079060}\NormalTok{, }\DecValTok{28079038}\NormalTok{)) }\OperatorTok\StringTok{ }
\StringTok{    }\KeywordTok{mutate}\NormalTok{(}\DataTypeTok{month =} \KeywordTok{month}\NormalTok{(date), }\DataTypeTok{year =} \KeywordTok{year}\NormalTok{(date)) }\OperatorTok\StringTok{ }\KeywordTok{left_join}\NormalTok{(stations) }\OperatorTok\StringTok{ }
\StringTok{    }\KeywordTok{group_by}\NormalTok{(month, name, year) }\OperatorTok\StringTok{ }\KeywordTok{summarise}\NormalTok{(}\DataTypeTok{NO_2 =} \KeywordTok{mean}\NormalTok{(NO_}\DecValTok{2}\NormalTok{, }
    \DataTypeTok{na.rm =} \OtherTok{TRUE}\NormalTok{), }\DataTypeTok{n =} \KeywordTok{n}\NormalTok{()) }\OperatorTok\StringTok{ }\KeywordTok{filter}\NormalTok{(n }\OperatorTok{>}\StringTok{ }\DecValTok{500}\NormalTok{)}
\end{Highlighting}
\end{Shaded}

La primera parte del código es la lectura del archivo

\begin{Shaded}
\begin{Highlighting}[]
\NormalTok{Madrid2017 <-}\StringTok{ }\KeywordTok{read_csv}\NormalTok{(}\StringTok{"csvs_per_year/madrid_2017.csv"}\NormalTok{)}
\end{Highlighting}
\end{Shaded}

Para hacer esto por todos los archivos de la base de datos requeriríamos
de una lista o vector con los nombres de cada uno de los archivos. ¡Si
solo hubiera una función en R que nos permitiera leer los archivos de
una carpeta! La función \texttt{list.files} hace eso.

Entonces el código que vemos abajo genera un vector con todos los
nombres de los archivos que queremos incorporar:

\begin{Shaded}
\begin{Highlighting}[]
\NormalTok{Archivos <-}\StringTok{ }\KeywordTok{list.files}\NormalTok{(}\StringTok{"csvs_per_year"}\NormalTok{, }\DataTypeTok{full.names =} \OtherTok{TRUE}\NormalTok{)}
\NormalTok{Archivos}
\end{Highlighting}
\end{Shaded}

\begin{verbatim}
##  [1] "csvs_per_year/madrid_2001.csv" "csvs_per_year/madrid_2002.csv"
##  [3] "csvs_per_year/madrid_2003.csv" "csvs_per_year/madrid_2004.csv"
##  [5] "csvs_per_year/madrid_2005.csv" "csvs_per_year/madrid_2006.csv"
##  [7] "csvs_per_year/madrid_2007.csv" "csvs_per_year/madrid_2008.csv"
##  [9] "csvs_per_year/madrid_2009.csv" "csvs_per_year/madrid_2010.csv"
## [11] "csvs_per_year/madrid_2011.csv" "csvs_per_year/madrid_2012.csv"
## [13] "csvs_per_year/madrid_2013.csv" "csvs_per_year/madrid_2014.csv"
## [15] "csvs_per_year/madrid_2015.csv" "csvs_per_year/madrid_2016.csv"
## [17] "csvs_per_year/madrid_2017.csv" "csvs_per_year/madrid_2018.csv"
\end{verbatim}

Entonces poner dentro de map, un vector con el nombre de los archivos
(\emph{Archivos}), y una función para leer los archivos
(\emph{read\_csv}). Esto es el siguiente código

\begin{Shaded}
\begin{Highlighting}[]
\NormalTok{Madrid <-}\StringTok{ }\KeywordTok{map}\NormalTok{(Archivos, read_csv)}
\end{Highlighting}
\end{Shaded}

Genera una lista donde cada elemento es un data frame de un año de
mediciones.

Cuando se agregan otras funciones mas complejas en un loop usando map.
Como por ejemplo \texttt{filter}, ponemos un simbolo
\texttt{\textasciitilde{}} dentro de map, y un \texttt{.x} dentro de
filter para representar a cada dataframe que usaremos.

\begin{Shaded}
\begin{Highlighting}[]
\NormalTok{Madrid <-}\StringTok{ }\KeywordTok{map}\NormalTok{(Archivos, read_csv) }\OperatorTok\StringTok{ }\KeywordTok{map}\NormalTok{(}\OperatorTok{~}\KeywordTok{filter}\NormalTok{(.x, station }\OperatorTok\StringTok{ }
\StringTok{    }\KeywordTok{c}\NormalTok{(}\DecValTok{28079036}\NormalTok{, }\DecValTok{28079008}\NormalTok{, }\DecValTok{28079058}\NormalTok{, }\DecValTok{28079060}\NormalTok{, }\DecValTok{28079038}\NormalTok{)))}
\end{Highlighting}
\end{Shaded}

De esta forma podemos seguir la receta creada anteriormente sin ningún
problema.

\begin{Shaded}
\begin{Highlighting}[]
\NormalTok{Madrid <-}\StringTok{ }\KeywordTok{map}\NormalTok{(Archivos, read_csv) }\OperatorTok\StringTok{ }\KeywordTok{map}\NormalTok{(}\OperatorTok{~}\KeywordTok{filter}\NormalTok{(.x, station }\OperatorTok\StringTok{ }
\StringTok{    }\KeywordTok{c}\NormalTok{(}\DecValTok{28079036}\NormalTok{, }\DecValTok{28079008}\NormalTok{, }\DecValTok{28079058}\NormalTok{, }\DecValTok{28079060}\NormalTok{, }\DecValTok{28079038}\NormalTok{))) }\OperatorTok\StringTok{ }
\StringTok{    }\KeywordTok{map}\NormalTok{(}\OperatorTok{~}\KeywordTok{mutate}\NormalTok{(.x, }\DataTypeTok{month =} \KeywordTok{month}\NormalTok{(date), }\DataTypeTok{year =} \KeywordTok{year}\NormalTok{(date))) }\OperatorTok\StringTok{ }
\StringTok{    }\KeywordTok{map}\NormalTok{(}\OperatorTok{~}\KeywordTok{left_join}\NormalTok{(.x, stations)) }\OperatorTok\StringTok{ }\KeywordTok{map}\NormalTok{(}\OperatorTok{~}\KeywordTok{group_by}\NormalTok{(.x, month, }
\NormalTok{    name, year)) }\OperatorTok\StringTok{ }\KeywordTok{map}\NormalTok{(}\OperatorTok{~}\KeywordTok{summarise}\NormalTok{(.x, }\DataTypeTok{NO_2 =} \KeywordTok{mean}\NormalTok{(NO_}\DecValTok{2}\NormalTok{, }\DataTypeTok{na.rm =} \OtherTok{TRUE}\NormalTok{), }
    \DataTypeTok{n =} \KeywordTok{n}\NormalTok{())) }\OperatorTok\StringTok{ }\KeywordTok{map}\NormalTok{(}\OperatorTok{~}\KeywordTok{filter}\NormalTok{(.x, n }\OperatorTok{>}\StringTok{ }\DecValTok{500}\NormalTok{))}
\end{Highlighting}
\end{Shaded}

Pero en este momento tenemos una lista con 17 data frames, en vez de un
gran data frame con todos los datos. Para esto debenos unir esta lista
usando la función \texttt{reduce}, lo cual nos genera el siguiente
código y la tabla \ref{tab:Madrid2017e}

\begin{Shaded}
\begin{Highlighting}[]
\KeywordTok{library}\NormalTok{(lubridate)}
\NormalTok{Madrid <-}\StringTok{ }\KeywordTok{map}\NormalTok{(Archivos, read_csv) }\OperatorTok\StringTok{ }\KeywordTok{map}\NormalTok{(}\OperatorTok{~}\KeywordTok{filter}\NormalTok{(.x, station }\OperatorTok\StringTok{ }
\StringTok{    }\KeywordTok{c}\NormalTok{(}\DecValTok{28079036}\NormalTok{, }\DecValTok{28079008}\NormalTok{, }\DecValTok{28079058}\NormalTok{, }\DecValTok{28079060}\NormalTok{, }\DecValTok{28079038}\NormalTok{))) }\OperatorTok\StringTok{ }
\StringTok{    }\KeywordTok{map}\NormalTok{(}\OperatorTok{~}\KeywordTok{mutate}\NormalTok{(.x, }\DataTypeTok{month =} \KeywordTok{month}\NormalTok{(date), }\DataTypeTok{year =} \KeywordTok{year}\NormalTok{(date))) }\OperatorTok\StringTok{ }
\StringTok{    }\KeywordTok{map}\NormalTok{(}\OperatorTok{~}\KeywordTok{left_join}\NormalTok{(.x, stations)) }\OperatorTok\StringTok{ }\KeywordTok{map}\NormalTok{(}\OperatorTok{~}\KeywordTok{group_by}\NormalTok{(.x, month, }
\NormalTok{    name, year)) }\OperatorTok\StringTok{ }\KeywordTok{map}\NormalTok{(}\OperatorTok{~}\KeywordTok{summarise}\NormalTok{(.x, }\DataTypeTok{NO_2 =} \KeywordTok{mean}\NormalTok{(NO_}\DecValTok{2}\NormalTok{, }\DataTypeTok{na.rm =} \OtherTok{TRUE}\NormalTok{), }
    \DataTypeTok{n =} \KeywordTok{n}\NormalTok{())) }\OperatorTok\StringTok{ }\KeywordTok{map}\NormalTok{(}\OperatorTok{~}\KeywordTok{filter}\NormalTok{(.x, n }\OperatorTok{>}\StringTok{ }\DecValTok{500}\NormalTok{)) }\OperatorTok\StringTok{ }\KeywordTok{reduce}\NormalTok{(bind_rows)}
\end{Highlighting}
\end{Shaded}

\label{tab:Madrid2017e}Los primeros 20 datos de calidad de aire del todos
los años en Madrid después de filtrar según estación, con mes, año y
nombre.

month

name

year

NO\_2

n

1

Cuatro Caminos

2017

59.88124

743

1

El Pardo

2017

27.89892

743

1

Escuelas Aguirre

2017

69.43666

743

1

Moratalaz

2017

54.17004

743

1

Tres Olivos

2017

55.28706

743

2

Cuatro Caminos

2017

46.01045

672

2

El Pardo

2017

17.18955

672

2

Escuelas Aguirre

2017

59.89686

672

2

Moratalaz

2017

44.19581

672

2

Tres Olivos

2017

39.38209

672

3

Cuatro Caminos

2017

43.71833

744

3

El Pardo

2017

15.07962

744

3

Escuelas Aguirre

2017

61.74798

744

3

Moratalaz

2017

43.01353

744

3

Tres Olivos

2017

36.68329

744

4

Cuatro Caminos

2017

34.35139

720

4

El Pardo

2017

11.48122

720

4

Escuelas Aguirre

2017

51.56485

720

4

Moratalaz

2017

30.07232

720

4

Tres Olivos

2017

25.51253

720

\hypertarget{presentacion}{%
\chapter{Presentaciones en R}\label{presentacion}}

\hypertarget{paquetes-necesarios-para-este-capitulo-6}{%
\section{Paquetes necesarios para este
capítulo}\label{paquetes-necesarios-para-este-capitulo-6}}

Para este capítulo necesitas tener instalado el paquete
\emph{rmarkdown}.

En este capítulo aprenderemos dos formas distintas de hacer
presentaciones en R

Dado que este libro es un apoyo para el curso BIO4022, esta clase puede
también ser seguida en este
\href{https://derek-corcoran-barrios.github.io/Clase7/Clase7PresentacionesEnR}{link}.
El video de la clase se encontrará disponible en este
\href{https://youtu.be/fgX-FV7fuxA}{link}.

\hypertarget{soluciones}{%
\chapter{Soluciones a problemas}\label{soluciones}}

Todos los problemas en programación tienen más de una forma de llegar a
ellos, es por esto que las soluciones acá mostradas deben tomarse solo
como una referencia, y revisar si el resultado final de tu código
(aunque sea distinto de este), sea igual al que presentamos.

\hypertarget{capitulo-1}{%
\section{Capítulo 1}\label{capitulo-1}}

\hypertarget{ejercicio-1-3}{%
\subsection{Ejercicio 1}\label{ejercicio-1-3}}

Algunas posibles soluciones:

\begin{Shaded}
\begin{Highlighting}[]
\NormalTok{storms }\OperatorTok\StringTok{ }\KeywordTok{filter}\NormalTok{(status }\OperatorTok{==}\StringTok{ "hurricane"}\NormalTok{) }\OperatorTok\StringTok{ }\KeywordTok{select}\NormalTok{(year, wind, }
\NormalTok{    hu_diameter) }\OperatorTok\StringTok{ }\KeywordTok{group_by}\NormalTok{(year) }\OperatorTok\StringTok{ }\KeywordTok{summarize_all}\NormalTok{(mean)}
\end{Highlighting}
\end{Shaded}

\begin{Shaded}
\begin{Highlighting}[]
\NormalTok{storms }\OperatorTok\StringTok{ }\KeywordTok{filter}\NormalTok{(status }\OperatorTok{==}\StringTok{ "hurricane"} \OperatorTok{&}\StringTok{ }\OperatorTok{!}\KeywordTok{is.na}\NormalTok{(hu_diameter)) }\OperatorTok\StringTok{ }
\StringTok{    }\KeywordTok{select}\NormalTok{(year, wind, hu_diameter) }\OperatorTok\StringTok{ }\KeywordTok{group_by}\NormalTok{(year) }\OperatorTok\StringTok{ }\KeywordTok{summarize_all}\NormalTok{(mean)}
\end{Highlighting}
\end{Shaded}

\begin{Shaded}
\begin{Highlighting}[]
\NormalTok{storms }\OperatorTok\StringTok{ }\KeywordTok{filter}\NormalTok{(status }\OperatorTok{==}\StringTok{ "hurricane"}\NormalTok{) }\OperatorTok\StringTok{ }\KeywordTok{select}\NormalTok{(year, wind, }
\NormalTok{    hu_diameter) }\OperatorTok\StringTok{ }\KeywordTok{group_by}\NormalTok{(year) }\OperatorTok\StringTok{ }\KeywordTok{summarize_all}\NormalTok{(}\KeywordTok{funs}\NormalTok{(mean), }
    \DataTypeTok{na.rm =} \OtherTok{TRUE}\NormalTok{)}
\end{Highlighting}
\end{Shaded}

\hypertarget{ejercicio-2-2}{%
\subsection{Ejercicio 2}\label{ejercicio-2-2}}

Una de las soluciones posibles:

\begin{Shaded}
\begin{Highlighting}[]
\NormalTok{Solution <-}\StringTok{ }\NormalTok{mpg }\OperatorTok\StringTok{ }\KeywordTok{filter}\NormalTok{(year }\OperatorTok{>}\StringTok{ }\DecValTok{2004} \OperatorTok{&}\StringTok{ }\NormalTok{class }\OperatorTok{==}\StringTok{ "compact"}\NormalTok{) }\OperatorTok\StringTok{ }
\StringTok{    }\KeywordTok{mutate}\NormalTok{(}\DataTypeTok{kpl =}\NormalTok{ (cty }\OperatorTok{*}\StringTok{ }\FloatTok{1.609}\NormalTok{)}\OperatorTok{/}\FloatTok{3.78541}\NormalTok{)}
\end{Highlighting}
\end{Shaded}

\hypertarget{capitulo-2}{%
\section{Capítulo 2}\label{capitulo-2}}

\hypertarget{ejercicio-1-4}{%
\subsection{Ejercicio 1}\label{ejercicio-1-4}}

Una posible solución a este problema sería:

\texttt{\textasciigrave{}r\ mean((iris\ \%\textgreater{}\%\ filter(Species\ ==\ "virginica"))\$Petal.Length)\textasciigrave{}}

\hypertarget{capitulo-3}{%
\section{Capítulo 3}\label{capitulo-3}}

\hypertarget{ejercicio-1-5}{%
\subsection{Ejercicio 1}\label{ejercicio-1-5}}

\hypertarget{a}{%
\subsubsection{a}\label{a}}

\begin{Shaded}
\begin{Highlighting}[]
\NormalTok{Sola <-}\StringTok{ }\NormalTok{Huemul }\OperatorTok\StringTok{ }\NormalTok{dplyr}\OperatorTok{::}\KeywordTok{select}\NormalTok{(lon, lat, basisOfRecord) }\OperatorTok\StringTok{ }
\StringTok{    }\KeywordTok{filter}\NormalTok{(}\OperatorTok{!}\KeywordTok{is.na}\NormalTok{(lat) }\OperatorTok{&}\StringTok{ }\OperatorTok{!}\KeywordTok{is.na}\NormalTok{(lon))}
\end{Highlighting}
\end{Shaded}

lon

lat

basisOfRecord

-72.97587

-51.18353

HUMAN\_OBSERVATION

-72.93940

-49.37483

HUMAN\_OBSERVATION

-72.97712

-51.01511

HUMAN\_OBSERVATION

-71.87026

-46.08686

HUMAN\_OBSERVATION

-72.43751

-47.20485

HUMAN\_OBSERVATION

-73.01456

-51.03635

HUMAN\_OBSERVATION

-73.03190

-51.17531

HUMAN\_OBSERVATION

-72.72944

-46.25602

HUMAN\_OBSERVATION

-71.31538

-41.30110

PRESERVED\_SPECIMEN

-71.31538

-41.30110

PRESERVED\_SPECIMEN

-71.71667

-44.86667

PRESERVED\_SPECIMEN

-71.71667

-44.86667

PRESERVED\_SPECIMEN

-71.30989

-40.81978

PRESERVED\_SPECIMEN

-71.31538

-41.30110

PRESERVED\_SPECIMEN

-73.02467

-50.46476

PRESERVED\_SPECIMEN

-71.33186

-41.26523

PRESERVED\_SPECIMEN

-73.01764

-50.46747

PRESERVED\_SPECIMEN

-71.70000

-45.26667

PRESERVED\_SPECIMEN

-71.70000

-45.26667

PRESERVED\_SPECIMEN

-71.70000

-45.26667

PRESERVED\_SPECIMEN

-72.08000

-47.25000

PRESERVED\_SPECIMEN

-72.00000

-41.50000

PRESERVED\_SPECIMEN

-71.36714

-41.13574

PRESERVED\_SPECIMEN

-71.71094

-42.75692

HUMAN\_OBSERVATION

-71.64718

-40.22605

PRESERVED\_SPECIMEN

-67.88534

-43.99376

PRESERVED\_SPECIMEN

\hypertarget{b}{%
\subsubsection{b}\label{b}}

\begin{Shaded}
\begin{Highlighting}[]
\NormalTok{Solb <-}\StringTok{ }\NormalTok{Huemul }\OperatorTok\StringTok{ }\KeywordTok{group_by}\NormalTok{(basisOfRecord) }\OperatorTok\StringTok{ }\KeywordTok{summarize}\NormalTok{(}\DataTypeTok{N =} \KeywordTok{n}\NormalTok{())}
\end{Highlighting}
\end{Shaded}

basisOfRecord

N

HUMAN\_OBSERVATION

103

PRESERVED\_SPECIMEN

65

\hypertarget{ejercicio-2-3}{%
\subsection{Ejercicio 2}\label{ejercicio-2-3}}

\hypertarget{a-1}{%
\subsubsection{a}\label{a-1}}

Primero bajamos la base de datos, lo cual se puede hacer de forma manual
o como en el código siguiente utilizando la función
\texttt{download.file}

\begin{Shaded}
\begin{Highlighting}[]
\KeywordTok{download.file}\NormalTok{(}\StringTok{"http://www.ine.cl/docs/default-source/medioambiente-(micrositio)/variables-b%C3%A1sicas-ambientales-(vba)/aire/dimensi%C3%B3n-aire-factor-estado.xlsx?sfvrsn=4"}\NormalTok{, }
    \DataTypeTok{destfile =} \StringTok{"test.xlsx"}\NormalTok{)}
\end{Highlighting}
\end{Shaded}

Una vez bajada esta base de datos utilizaremos los paquetes
\emph{readxl} para leer los archivos excel, \emph{tidyverse} para
manipular los datos y \emph{stringr} para trabajar con texto.

\begin{Shaded}
\begin{Highlighting}[]
\KeywordTok{library}\NormalTok{(readxl)}
\KeywordTok{library}\NormalTok{(tidyverse)}
\KeywordTok{library}\NormalTok{(stringr)}
\end{Highlighting}
\end{Shaded}

Partimos leyendo la pestaña que contiene las estaciones meteorológicas
con su código:

\begin{Shaded}
\begin{Highlighting}[]
\NormalTok{EM <-}\StringTok{ }\KeywordTok{read_excel}\NormalTok{(}\StringTok{"test.xlsx"}\NormalTok{, }\DataTypeTok{sheet =} \StringTok{"T001"}\NormalTok{)}
\end{Highlighting}
\end{Shaded}

Luego para poder más adelante unir esta base de datos con otras,
cambiamos el nombre de la columna \emph{Codigo\_Est\_Meteoro} a
\emph{Est\_Meteoro} como aparece en las otras bases de datos.

\begin{Shaded}
\begin{Highlighting}[]
\NormalTok{EM <-}\StringTok{ }\NormalTok{EM }\OperatorTok\StringTok{ }\KeywordTok{rename}\NormalTok{(}\DataTypeTok{Est_Meteoro =}\NormalTok{ Codigo_Est_Meteoro)}
\end{Highlighting}
\end{Shaded}

Luego empezamos a trabajar con la base de datos de temperatura media,
para esto leemos la pestaña \emph{E10000003}

\begin{Shaded}
\begin{Highlighting}[]
\NormalTok{TempMedia <-}\StringTok{ }\KeywordTok{read_excel}\NormalTok{(}\StringTok{"test.xlsx"}\NormalTok{, }\DataTypeTok{sheet =} \StringTok{"E10000003"}\NormalTok{)}
\end{Highlighting}
\end{Shaded}

Existen varias variables que no utilizaremos, por ejemplo el código de
la variable, y la unidad de medida. Además vemos que la variable día,
siempre tiene valor 0, por lo cuál podemos eliminarla también.

\begin{Shaded}
\begin{Highlighting}[]
\NormalTok{TempMedia <-}\StringTok{ }\NormalTok{TempMedia }\OperatorTok\StringTok{ }\KeywordTok{select}\NormalTok{(}\OperatorTok{-}\NormalTok{Codigo_variable, }\OperatorTok{-}\NormalTok{Unidad_medida, }
    \OperatorTok{-}\NormalTok{Día)}
\end{Highlighting}
\end{Shaded}

Además podemos cambiar los nombres de la columna \emph{ValorF} que no
tiene ningún significado a \emph{TempMedia} y \emph{Año} a \emph{Year},
esta última variable es cambiada solo por que la letra Ñ puede no ser
leída por todos los computadores.

\begin{Shaded}
\begin{Highlighting}[]
\NormalTok{TempMedia <-}\StringTok{ }\NormalTok{TempMedia }\OperatorTok\StringTok{ }\KeywordTok{rename}\NormalTok{(}\DataTypeTok{TempMedia =}\NormalTok{ ValorF, }\DataTypeTok{Year =}\NormalTok{ Año)}
\end{Highlighting}
\end{Shaded}

Si nos fijamos, hay algunos años, en los cuales todos los meses aparecen
como 13, esto nos indica que en estos años no se registró en que mes se
realizó la medición, por lo cual se eliminarán esas obsevaciones.

\begin{Shaded}
\begin{Highlighting}[]
\NormalTok{TempMedia <-}\StringTok{ }\NormalTok{TempMedia }\OperatorTok\StringTok{ }\KeywordTok{filter}\NormalTok{(Mes }\OperatorTok{!=}\StringTok{ }\DecValTok{13}\NormalTok{)}
\end{Highlighting}
\end{Shaded}

Posterior a esto, unumos la base de datos \emph{TempMedia} con la base
de datos \emph{EM} y seleccionamos tan solo las columnas que nos
interesan y finalmente transformamos el mes en una variable numérica:

\begin{verbatim}
## Joining, by = "Est_Meteoro"
\end{verbatim}

\begin{Shaded}
\begin{Highlighting}[]
\NormalTok{TempMedia <-}\StringTok{ }\KeywordTok{left_join}\NormalTok{(TempMedia, EM) }\OperatorTok\StringTok{ }\KeywordTok{select}\NormalTok{(Mes, Year, TempMedia, }
\NormalTok{    Ciudad_localidad) }\OperatorTok\StringTok{ }\KeywordTok{mutate}\NormalTok{(}\DataTypeTok{Mes =} \KeywordTok{as.numeric}\NormalTok{(Mes))}
\end{Highlighting}
\end{Shaded}

Si hicieramos todo esto en un comando tendriamos el siguiente código

\begin{Shaded}
\begin{Highlighting}[]
\NormalTok{TempMedia <-}\StringTok{ }\KeywordTok{read_excel}\NormalTok{(}\StringTok{"test.xlsx"}\NormalTok{, }\DataTypeTok{sheet =} \StringTok{"E10000003"}\NormalTok{) }\OperatorTok\StringTok{ }
\StringTok{    }\KeywordTok{select}\NormalTok{(}\OperatorTok{-}\NormalTok{Codigo_variable, }\OperatorTok{-}\NormalTok{Unidad_medida, }\OperatorTok{-}\NormalTok{Día) }\OperatorTok\StringTok{ }\KeywordTok{rename}\NormalTok{(}\DataTypeTok{TempMedia =}\NormalTok{ ValorF, }
    \DataTypeTok{Year =}\NormalTok{ Año) }\OperatorTok\StringTok{ }\KeywordTok{filter}\NormalTok{(Mes }\OperatorTok{!=}\StringTok{ }\DecValTok{13}\NormalTok{) }\OperatorTok\StringTok{ }\KeywordTok{left_join}\NormalTok{(EM) }\OperatorTok\StringTok{ }
\StringTok{    }\KeywordTok{select}\NormalTok{(Mes, Year, TempMedia, Ciudad_localidad) }\OperatorTok\StringTok{ }\KeywordTok{mutate}\NormalTok{(}\DataTypeTok{Mes =} \KeywordTok{as.numeric}\NormalTok{(Mes))}
\end{Highlighting}
\end{Shaded}

De la misma manera modificamos el código de arriba para la humedad con
la salvedad que la columna de día no tiene tilde en esta pestaña a la
fecha de 19 de Agosto del 2018:

\begin{verbatim}
## Joining, by = "Est_Meteoro"
\end{verbatim}

\begin{Shaded}
\begin{Highlighting}[]
\NormalTok{HumMedia <-}\StringTok{ }\KeywordTok{read_excel}\NormalTok{(}\StringTok{"test.xlsx"}\NormalTok{, }\DataTypeTok{sheet =} \StringTok{"E10000006"}\NormalTok{) }\OperatorTok\StringTok{ }
\StringTok{    }\NormalTok{dplyr}\OperatorTok{::}\KeywordTok{select}\NormalTok{(}\OperatorTok{-}\NormalTok{Codigo_variable, }\OperatorTok{-}\NormalTok{Unidad_medida, }\OperatorTok{-}\NormalTok{Dia) }\OperatorTok\StringTok{ }
\StringTok{    }\KeywordTok{rename}\NormalTok{(}\DataTypeTok{HumMedia =}\NormalTok{ ValorF, }\DataTypeTok{Year =}\NormalTok{ Año) }\OperatorTok\StringTok{ }\KeywordTok{filter}\NormalTok{(Mes }\OperatorTok{!=}\StringTok{ }
\StringTok{    }\DecValTok{13}\NormalTok{) }\OperatorTok\StringTok{ }\KeywordTok{left_join}\NormalTok{(EM) }\OperatorTok\StringTok{ }\NormalTok{dplyr}\OperatorTok{::}\KeywordTok{select}\NormalTok{(Mes, Year, HumMedia, }
\NormalTok{    Ciudad_localidad) }\OperatorTok\StringTok{ }\KeywordTok{mutate}\NormalTok{(}\DataTypeTok{Mes =} \KeywordTok{as.numeric}\NormalTok{(Mes))}
\end{Highlighting}
\end{Shaded}

En el siguiente código unimos las dos bases de datos, vemos las primeras
20 observaciones de la base de datos resultante en la tabla
\ref{tab:TempHum}

\begin{Shaded}
\begin{Highlighting}[]
\NormalTok{TempHum <-}\StringTok{ }\KeywordTok{full_join}\NormalTok{(TempMedia, HumMedia)}
\end{Highlighting}
\end{Shaded}

\begin{verbatim}
## Joining, by = c("Mes", "Year", "Ciudad_localidad")
\end{verbatim}

\label{tab:TempHum}Las primeras 20 observaciones de temperatura y humedad
unidas

Mes

Year

TempMedia

Ciudad\_localidad

HumMedia

1

1981

22.0

Arica

NA

2

1981

22.2

Arica

NA

3

1981

22.1

Arica

NA

4

1981

20.3

Arica

NA

5

1981

18.2

Arica

NA

6

1981

17.0

Arica

NA

7

1981

15.0

Arica

NA

8

1981

16.0

Arica

NA

9

1981

16.6

Arica

NA

10

1981

15.9

Arica

NA

11

1981

19.1

Arica

NA

12

1981

21.1

Arica

NA

1

1981

19.7

Iquique

NA

2

1981

21.1

Iquique

NA

3

1981

20.9

Iquique

NA

4

1981

19.9

Iquique

NA

5

1981

17.6

Iquique

NA

6

1981

15.9

Iquique

NA

7

1981

14.6

Iquique

NA

8

1981

15.6

Iquique

NA

Con esto vemos que la humedad media no es medida en los mismos años ni
en todos los lugares que se mide la temperatura media, pero como ambas
variables nos interesan por igual, la mantenemos de todas maneras con
sus valores \emph{NA}

\hypertarget{b-1}{%
\subsubsection{b}\label{b-1}}

El segundo ejercicio es mucho mas simple, donde solo tenemos que agrupar
por ciudad y mes, y usar \texttt{summarize\_all} para las funciones
\texttt{mean} y \texttt{sd} como vemos en la tabla
\ref{tab:TempHumMonthly}

\begin{Shaded}
\begin{Highlighting}[]
\NormalTok{TempHumMonthly <-}\StringTok{ }\NormalTok{TempHum }\OperatorTok\StringTok{ }\KeywordTok{select}\NormalTok{(}\OperatorTok{-}\NormalTok{Year) }\OperatorTok\StringTok{ }\KeywordTok{group_by}\NormalTok{(Mes, }
\NormalTok{    Ciudad_localidad) }\OperatorTok\StringTok{ }\KeywordTok{summarize_all}\NormalTok{(}\KeywordTok{funs}\NormalTok{(mean, sd), }\DataTypeTok{na.rm =} \OtherTok{TRUE}\NormalTok{)}
\end{Highlighting}
\end{Shaded}

\label{tab:TempHumMonthly}Las primeras 20 observaciones de temperatura y
humedad agrupadas por mes y localidad

Mes

Ciudad\_localidad

TempMedia\_mean

HumMedia\_mean

TempMedia\_sd

HumMedia\_sd

1

Antártica Chilena

1.388889

87.35000

0.6319031

3.483772

1

Antofagasta

20.125000

69.70000

0.8378118

1.589549

1

Arica

22.375000

62.72500

0.9391105

2.394960

1

Balmaceda

12.358823

56.37500

1.2200640

2.487804

1

Calama

17.000000

25.00000

NA

NA

1

Cerrillos

20.996000

NaN

0.7855359

NaN

1

Chillán

19.747059

63.05000

0.7054916

3.750111

1

Concepción

16.683333

73.75000

0.6222080

6.727308

1

Copiapó

19.604348

NaN

0.7449700

NaN

1

Coyhaique

13.980556

58.27500

1.2537531

1.543535

1

Curicó

20.632353

59.65000

0.7293503

6.310573

1

Graneros

21.480000

NaN

0.4661330

NaN

1

Iquique

21.791667

61.72500

1.0332680

3.981101

1

Isla Juan Fernández

18.513889

71.22500

0.5111068

3.044531

1

La Serena

17.311429

75.97500

0.7275145

3.187868

1

Osorno

15.807407

74.10000

0.9388725

2.265686

1

Pudahuel

20.652778

47.53333

0.7268272

5.852635

1

Puerto Montt

14.451429

78.00000

0.7184016

2.499333

1

Punta Arenas

10.852778

62.55000

0.7443064

3.349129

1

Quinta Normal

21.261111

53.35000

0.5530579

6.507688

\bibliography{book.bib,packages.bib}


\end{document}
